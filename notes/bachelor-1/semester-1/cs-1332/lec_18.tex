\lecture{18}{Fri 06 Oct 2023 15:22}{AVL Continued}

\subsubsection{Right Left Rotation}
This type of rotation is used when:
\begin{itemize}
	\item The node is leaning right (balance factor is -2)
	\item The right child is leaning left (balance factor is 1)
\end{itemize}
We rotate the right child to the right. Then, we rotate the root to the left. Even though this is a combination of two rotations, it is still considered a single operation.

\subsubsection{Left Right Rotation}
This type of rotation is used when:
\begin{itemize}
	\item The node is leaning left (balance factor is 2)
	\item The left child is leaning right (balance factor is -1)
\end{itemize}
This is just the mirror of the previous operation: we rotate the left child to the left, and the root to the right.

\subsection{Runtime}
How long do operations on an AVL tree take?
\begin{itemize}
	\item When you add data into an AVL tree, you do at most one rotation, so the runtime is \( O(\log (n)) \).
	\item When you remove from an AVL tree, you do at most \( \log (n) \) rotations. Even still, the runtime is still \( O(\log (n)) \).
	\item Remember that each rotation is \( O(1) \).
\end{itemize}

\begin{note}
	An AVL can be at most \( 1.44\log (n) =\log (n)\) tall. This is derived from \( \frac{1}{\log_2(\phi )} \).
\end{note}

\exercise{1}
What does the following AVL tree look like after these operations?
\begin{itemize}
	\item add(56, 75, 61, 88, 93, 77)
	\item remove(61)
\end{itemize}
