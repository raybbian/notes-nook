\lecture{27}{Mon 30 Oct 2023 14:00}{Partially Ordered Sets}

One last bit about graphs:

\begin{note}
	If \( G \) is a graph with chromatic number \( k \), then \( G \) contains a subgraph \( H \) such that \( \chi (H)=k \) and \( \chi (H-v) < k \) for all \( v \in  V(H) \). Such graphs are called \textbf{critical graphs}.
\end{note}

\section{Partially Ordered Sets}

\subsection{Relations}

Finally, a new section!

\begin{definition}
	A \textbf{binary relation} on a set \( X \) is a subset \( \mathcal{R} \subseteq X^2 = X \times X \).
\end{definition}

In other words, \( R \) is a set of some ordered pairs \( (x,y) \) with \( x,y \in  X \).

\begin{eg}
	Let \( X=\{a,b,c\}   \). Then, one such relation is \( \mathcal{R}=\{(a,a),(a,b),(b,a),(b,c),(c,c)\}   \).
\end{eg}

\begin{note}
	These relations can be related to a directed graph where there can be loops and multiple edges between vertices.
\end{note}

\begin{eg}
	The empty set \( \varnothing \) is a relation.
\end{eg}

We use the word relation because it is a set of pairs \( (x,y) \) where \( x \) is ``related'' to \( y \) in some sense. We say that \( (x,y) \in \mathcal{R} \) if \( y \) is \( \mathcal{R} \)-related to \( x \).

\begin{eg}
	Let \( X=\{1,2,3,4,5\}   \). \( \mathcal{R}=\{ (x,y) \in X^2 \colon x < y\} \) is another way of writing \( \{(1,2),(1,3),(1,4),(1,5),(2,3),(2,4),(2,5),(3,4),(3,5),(4,5)\}    \)
\end{eg}

Note that relations can exist on infinite sets as well!

\begin{eg}
	\( \{(n,m) \in \mathbb{N}^2 \colon n \le m\}   \) is an ordering relation of the natural numbers.
\end{eg}

\begin{eg}
	\( \{(x,y) \in \mathbb{R}^2 \times \mathbb{R}^2 \colon x \text{ and } y \text{ are orthogonal}\} \) is another relation containing all pairs of orthogonal vectors.
\end{eg}

\begin{remark}
	In this course, we will focus on relations on two elements in the same set. However, more generally, relations can exist between a set \( X \) and another set \( Y \) (subsets of \( X \times  Y \)).
\end{remark}

\begin{eg}
	Let \( X \) be the set of GT students enrolled in Fall 2023, let \( Y \) be the set of classes offered at GT in Fall 2023. Then, \[
		\{(S,C) \in X \times Y \colon S \text{ is registered for } C\}
	.\] is one such valid, real-life relation.
\end{eg}

\begin{notation}
	When \( \mathcal{R} \) is a relation, we often write \( x \mathcal{R}y \) to mean \( (x,y) \in \mathcal{R} \)
\end{notation}

\begin{eg}
	We write \( x < y \) instead of \( (x,y) \in < \).
\end{eg}

\begin{eg}
	If \( X \) is a set of size \( n \), how many binary relations on \( X \) are there?
\end{eg}

There are \( n \cdot n = n^2\) elements in \( X \times X \), such that there are \( 2^{n^2} \) relations. Note that this value is \( |\mathcal{P}(X^2)| \).

\subsubsection{Properties of Relations}

\begin{property}
	Let \( \mathcal{R} \subseteq X^2 \) be a relation on \( X \). \( R \) is:
	\begin{enumerate}
		\item reflexive: for all \( x \in X \), \( x \mathcal{R}x \).
		\item irreflexive: for all \( x \in X \), not \( x \mathcal{R}x \).
		\item symmetric: for all \( x,y \in X \), if \( x\mathcal{R}y \), then \(y\mathcal{R}x \).
		\item asymmetric: for all \( x,y \in X \), if \( x \mathcal{R}y \), then not \( y\mathcal{R}x \). Note that all asymmetric relations are irreflexive.
		\item antisymmetric: for all \( x,y \in X \), if \( x \mathcal{R}y \) and \( y\mathcal{R}x \), then \( x=y \).
		\item transitive: for all \( x,y,z \in X \), if \( x \mathcal{R}y \) and \( y\mathcal{R}z \), then \( x\mathcal{R}z \).
	\end{enumerate}
\end{property}

\begin{eg}
	Let \( X=\{1,2,3\}   \), \( R=\{(1,1),(1,3),(2,2),(3,3)\}   \). What properties does this relation fall under?
\end{eg}

This relation is reflexive, not irreflexive, not symmetric, not asymmetric, antisymmetric.

\begin{note}
	The only relation on the empty set \( \varnothing \) is the empty set \( \varnothing \).
\end{note}

\exercise{1}
What properties are the relations on the lecture notes?
