\lecture{37}{Mon 27 Nov 2023 14:00}{Generating Functions}

\section{Generating Functions}

Solving combinatorial problems with algebra.

\begin{definition}
	Given a sequence of numbers \( a_{0},a_{1},a_{2},\ldots  \), its \textbf{generating function} is the series \[ a_{0}+a_{1}x+a_{2}x^2+a_{3}x^3 + \ldots = \sum_{k=0}^{\infty} a_kx^k .\]
\end{definition}

\begin{eg}
	Consider the case where \( a_k = \binom{n}{k} \), where \( \binom{n}{k}=0 \) if \( k>n \).
\end{eg}

The generating function for this sequence is \[
	\sum_{k=0}^{\infty} \binom{n}{k}x^k = \sum_{k=0}^{n} \binom{n}{k}x^k = (1+x)^n
.\] 

\begin{eg}
	Consider the sequence \( 1,1,1,1,1,1,\ldots  \) (\( a_k =1\) for all \( k \)).
\end{eg}

The generating function is then \[
	1 + x + x^2 + x^3 + \ldots = \sum_{k=0}^{\infty}x^k = \frac{1}{1-x}
.\] 

\begin{note}
	This is because
	\begin{align*}
		\left( \sum_{k=0}^{\infty}x^k  \right)(1-x) &= (1 + x + x^2 + \ldots )(1 - x) \\
		&= 1 + x + x^2 + \ldots - x - x^2 - x^3 - \ldots \\
		&= 1
	.\end{align*}
	for \( |x|<1 \) (We need to make sure the series actually converges, first).
\end{note}

There are two ways to deal with these convergence issues:
\begin{enumerate}
	\item When dealing with generating functions, all equalities are assumed to hold for \( x \) sufficiently close to 0 (when all the relevant series do converge). This is sufficient for the purposes of this course.
	\item Restrict ourselves to ``formal manipulations'' of algebraic expressions. This is more rigorous, but more complicated and needs a background in Abstract Algebra. Roughly, this means we can add, multiply, etc. generating functions term-by-term without worrying about convergence.
\end{enumerate}

\begin{eg}
	\[
		\sum_{k=0}^{\infty} a_kx^k + \sum_{k=0}^{\infty} b_kx^k = \sum_{k=0}^{\infty} (a_k + b_k)x^k
	.\] 
\end{eg}

\begin{eg}
	\[
		\sum_{k=0}^{\infty} a_kx^k \cdot \sum_{k=0}^{\infty} b_kx^k = \sum_{k=0}^{\infty} \left( \sum_{j=0}^{k}a_j b_{k-j}  \right) x^k
	.\] 
\end{eg}
