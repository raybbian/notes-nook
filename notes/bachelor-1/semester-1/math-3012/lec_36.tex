\lecture{36}{Mon 20 Nov 2023 14:04}{Inclusion Exclusion Continued}

\begin{eg}
	How many strings of length \( n \) can be formed using \( m \) symbols so that each symbol appears at least once?
\end{eg}

For concreteness, we will label the symbols as \( \{1,2,3,\ldots ,m\}   \).

Let us charge the Principle of Inclusion Exlusion Cannon!

Let \[
	X = \{\text{all strings of length \( n \) using symbols \( 1,2,\ldots ,m\)}\} =[m]^n  
.\] Note that \( |X| = m^n \). For \( i \in [m] \), let \( A_i = \{\text{all strings in \( X \) that miss the symbol i}\} \). Then, \( |A_i| = (m-1)^n \). Remember that we want the number of of strings that don't miss any symbol, which is \[
	|X \setminus (A_{1}\cup A_{2}\cup \ldots \cup A_m)|
.\] 
Also, remember for a set \( S \subseteq [m] \), \( N(S) = \left| \bigcap_{i \in S}A_i \right|  \) = the number of strings in \( X \) that miss every symbol \( i \) such that \( i \) in S, which is just \( (m-|S|)^n \). Then, from PIE, our answer is 
\begin{align*}
	\sum_{S \subseteq [m]}(-1)^{|S|} N(S) &= \sum_{S \subseteq [m]}(-1)^{|S|}(m-|S|)^n \\
																				&= \sum_{k=0}^{m}(-1)^{k}\binom{m}{k}(m-k)^n \\
.\end{align*}

We can simplify slightly if \( m \) is even:
\begin{align*}
	\sum_{k=0}^{m}\binom{m}{k}(-1)^k(m-k)^n &= \sum_{k=0}^{m}\binom{m}{m-k}(-1)^{m-k}(m-k)^n \\
															&= \sum_{k=0}^{m}\binom{m}{k}(-1)^{k}(k)^n \tag{\( k \) is \( m-k \)}
.\end{align*}

\begin{corollary}
	For any integer \( n\ge 1 \):\[
		\sum_{k=0}^{n} \binom{n}{k} (-1)^k k^n = \begin{cases}
			n! & \text{if \( n \) is even} \\
			-n! & \text{if \( n \) is odd}
		\end{cases}
	.\] 
\end{corollary}

Why? When \( m=n \), we are counting strings of length \( n \) formed using \( n \) symbols, where each symbol appears at least once. This is just \( n! \). 

\subsection{Fixed Points}

Let \( S_n  = \{\text{all permutations of \( [n] \)}\}  \), \( |S_n| = n! \)
\begin{definition}
	A \textbf{fixed point of a permutation \( \pi =\{x_{1},x_{2},\ldots x_n\} \in S_n  \)} is \( i \in [n] \) such that \( x_i = i\).
\end{definition}

\begin{eg}
	For \( n=3 \), \( (3,2,1) \) has fixed point 2.
\end{eg}

\begin{eg}
	For \( n=3 \), \( (1,2,3) \) has all 3 fixed points.
\end{eg}

What is the average number of fixed points in a permutation \( \pi \in S_n \)?

We can calculate this value with the average: 
\begin{align*}
	\frac{\sum_{\pi  \in S_n} \text{num of fixed points in \( \pi  \)}}{n!} &= \frac{1}{n!}\sum_{\pi  \in S_n} \sum_{i = 1}^{n} 1[\text{\( i \) is a fixed point of \( \pi  \)}] \\
																																						 &= \frac{1}{n!}\sum_{i=1}^{n} \underbrace{\sum_{\pi  \in S_n} 1[\text{\( i \) is a fixed point of \( \pi  \)}]}_{\text{num \( \pi  \in S_n \) such that \( i \) is a fixed point of \( \pi  = (n-1)!\)}} \\
																																						 &= \frac{1}{n!}\sum_{i=1}^{n}(n-1)! = \frac{n\cdot (n-1)!}{n!} = 1
.\end{align*}

\begin{definition}
	A permutation \( \pi  \in S_n \) is a \textbf{derangement} if it has no fixed points. Denote \( D_n \) to be the number of derangements of \( [n] \).
\end{definition}

What is \( \frac{D_n}{n!} \) (i.e. the probability that a randomly chosen permutation of \( [n] \) is a derangement)?

We will set up the PIE cannon: Let \( X = S_n \). For \( i \in [n] \), let \( A_i =\{\pi  \in S_n : \text{\( i \) is fixed point of \( \pi  \)}\}  \). We wish to find \[
	D_n = |X \setminus\left( A_1 \cup A_2 \cup \ldots \cup A_n \right)|
.\] For a set \( S \subseteq[n] \), 
\begin{align*}
	N(S) &= |\bigcap_{i \in S}A_i| \\
			 &= \text{the number of perms \( \pi  \in S_n \) such that every \( i \in S \) is a fixed point of \( \pi  \)} \\
				&= (n-|S|)!
.\end{align*}

Then, by PIE, we have: \[
	D_n = \sum_{S \subseteq [n]}(-1)^{|S|}N(S) = \sum_{k=0}^{n}(-1)^k\binom{n}{k}(n-k)! = \sum_{k=0}^{n}(-1)^k\frac{n!}{k!}
.\]

\begin{eg}
	For \( n=3 \), \[
		D_3 = \sum_{k=0}^{3}(-1)^k \frac{3!}{k!} = \frac{6}{1} - \frac{6}{1} + \frac{6}{2} - \frac{6}{6} = 2
	.\] 
\end{eg}

Also note that we have \( \frac{D_n}{n!} = \sum_{k=0}^{n} = \frac{(-1)^k}{k!} \). Taking the limit of this as \( n \) goes to infinity, 
\begin{align*}
	\sum_{k=0}^{\infty}\frac{(-1)^k}{k!} = \frac{1}{e} \tag{\( e^x=\sum_{k=0}^{\infty} \frac{x^k}{k!} \)}
.\end{align*}

\exercise{1}
Can we find a simplification to the first example if \( m \) is odd?
