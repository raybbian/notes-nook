\lecture{20}{Wed 11 Oct 2023 14:03}{Euler's Theorem}

Continuing with the proof of Euler's Theorem:

\begin{proof}
	Let \( G \) be a conneced even graph. Let \( T = (x_{0}, x_{1}, x_{2}, \ldots , x_L = x_{0})\) be a closed trail in \( G \) of maximum length. We want to show that \( T \) is an Euler circuit. Assume, for the sake of contradiction, that \( T \) is not an Euler circuit. Then some edges are not used in \( T \). Let \( U \) be the set of all unused edges. Note that
	\[
		U = E(G) \setminus \{x_{0}x_{1}, x_{1}x_{2}, \ldots , x_{L-1}x_L\} \neq \varnothing
	.\] 
	Let \( X \coloneq \{x_{0}, x_{1}, x_{2}, \ldots , x_{L-1}\}   \) be the vertices used in \( T \) and \( Y \coloneqq V(G) \setminus X \) be the unused vertices. Note that every edge used by \( T \) has both endpoints in \( X \). We claim that there is an edge in \( U \) incident to a vertex in \( X \).
	\begin{description}
		\item[Case 1:]\( Y = \varnothing \). In this case, every vertex is in \( X \). Therefore, every edge in \( U \) is incident to two vertices in \( X \).
		\item[Case 2:] \( Y \neq \varnothing \). In this case, because \( G \) is connected, there must be at least one edge that connects a vertex in \( X \) to a vertex in \( Y \). This edge is incident to a vertex in \( X \), and is in \( U \) (as all edges not in \( U \) are only incident to vertices in \( X \)).
	\end{description}
	In both cases, we can find an edge in \( U \) incident to two vertices in \( X \), which was what we wanted. \par
	So, let \( x_i \in X \) be a vertex incident to an edge in \( U \). Let \( G' \) be the spanning subgraph of \( G \) with edge set \( U \) (we only keep the unused edges). Note that \( \deg_{G'}(x_i) > 0 \) because \( x_i \) is incident to an edge in \( U \). Also note that \( G' \) is an even graph (exercise). \par
	Then, by the lemma, there exists a closed trail \[
		(x_i=z_{0}, z_{1}, z_{2}, \ldots , z_k = x_i)
	\]  in \( G' \) starting and ending at \( x_i \) of length \( k > 0 \). Note that this closed trail only uses edges in \( U \). But then there would exist closed trail \[
	(x_{0}, x_{1}, \ldots , \underbrace{x_i = z_{0}, z_{1}, \ldots , z_k = x_i}_{\text{only added unused edges}}, x_{i+1}, \ldots , x_{L-1}, x_L = x_{0})
	.\] in \( G \) of length \( L + k > L \), which is impossible \contra. Therefore, our assumpion is false, and \( T \) is an Euler circuit.
\end{proof}

Note that this proof actually gives you an easy way of finding an Euler circuit in a connected, even graph. \par

Unfortunately, there is no similar, simple way to tell if a graph has a Hamilton cycle.

\begin{definition}
	The \textbf{minimum degree} of \( G \), denoted by \( \delta(G) \), is the minimum of the degrees of the vertices of \( G \).
\end{definition}


\begin{property}
	Let \( G \) be a graph with \( n \ge 3 \) vertices.
	\begin{itemize}
		\item If \( \delta(G) \ge n - 1 \), then \( G \) has a Hamilton cycle (it is a complete graph).
		\item If \( n \) is even, then we can find \( G \) with \( \delta(G) = \frac{n}{2} - 1 \) and no Hamiltonian cycle.
		\item If \( n \) is odd, then we can find \( G \) with \( \delta(G) = \frac{n-1}{2} \) and no Hamiltonian cycle (exercise).
	\end{itemize}
\end{property}

\begin{theorem}
	If \( G \) is a graph with \( n \ge 3 \) vertices and \( \delta(G) \ge \frac{n}{2} \), then \( G \) is Hamiltonian (Dirac).
\end{theorem}
\begin{proof}
	Theorem 5.18 in the book (clever use of longest paths).
\end{proof}

\exercise{1}
Write a computer program that finds euler circuits in connected even graphs.
\exercise{2}
If \( G \) is an even graph, then for any set \( X \subseteq V(G) \), the number of edges joining \( X \) to \( V(G)\setminus X \) is even.
