\lecture{21}{Fri 13 Oct 2023 14:03}{Graph Coloring}

\subsection{Graph Coloring}

Another property of graphs.

\begin{definition}
	A proper \( k \)\textbf{-coloring} of a graph \( G \) is an assignment of labels ("colors") from an \( k \)-element set to the vertices of \( G \) such that adjacent vertices are assigned different labels.
\end{definition}

\begin{definition}
	The \textbf{chromatic number} of \( G \), denoted \( \chi(G) \), is the minimum \( k \) such that \( G \) has a proper \( k \)-coloring.
\end{definition}

\begin{note}
	If \( G \) has \( n \) vertices, then \( \chi(G) \le n \).
\end{note}

\begin{notation}
	\( C_n \) denotes the \( n \)-cycle.
\end{notation}

\begin{eg}
	\( \chi(C_n) = 2\) if \( n \) is even, and \( \chi(C_n) = 3 \) if \( n \) is odd.
\end{eg}

\begin{eg}
	\( \chi(\text{tree}) = 2 \) if there are at least two vertices.
\end{eg}

\begin{note}
	In a proper coloring, vertices of the same color form an independent set. In other words, \( \chi(G) \) is the minimum \( k \) such that it is possible to partition \( V(G) \) into \( k \) independent sets.
\end{note}

Why is coloring useful?
\begin{itemize}
	\item It's fun. 
	\item Practical uses, e.g. scheduling problems and register allocation (in compilers), radio bandwidth allocation, etc.
	\item It is a useful auxiliary tool for other problems, e.g. an algorithm that may process one independent set in a graph at a time.
	\item It can capture in a single number some complex structural information about a graph.
\end{itemize}

\begin{eg}
	You are trying to assign a set of jobs \( J_{1}, J_2, \ldots , J_n \) into time slots, where some jobs conflict with each other and can't be assigned to the same time slot (if they use the same equipment). Define a graph \( G \colon V(G) = \{J_{1}, J_{2},\ldots , J_n \}   \) where edges are inserted between every conflicting job. Then, we know that every valid time slot assignment is a proper coloring of \( G \). Note that \( \chi (G) \) is the minimum number of time slots required to complete all jobs.
\end{eg}

\begin{definition}
	\( G \) is \textbf{bipartite} if \( \chi(G) \le 2 \).
\end{definition}

\begin{theorem}
	\( G \) is bipartite if and only if \( G \) has no odd cycles.
\end{theorem}

If \( \chi(G) \ge 3 \) is because there are cycles of odd length, then what makes \( \chi (G) \) large?

\begin{definition}
	\( \omega(G) \), the \textbf{clique number} of \( G \), is the maxmimum size of a clique in \( G \).
\end{definition}

\begin{property}
	\( \chi(G) \ge \omega (G) \)
\end{property}

\begin{note}
	We can also have \( \chi(G) > \omega (G) \): \( \chi (C_5) = 3\), but \( \omega (C_5) = 2\) (works for any cycle of odd length).
\end{note}

\begin{definition}
	A graph \( G \) is \textbf{triangle-free} if there are no cliques of size 3 (which look like triangles) in \( G \).
\end{definition}

\begin{theorem}
	For any \( k \in \mathbb{N} \), there is a graph \( G \) such that \( \chi (G) \ge k \) and \( \omega (G) = 2 \).
\end{theorem}

There are many proofs for this theorem. We will use the Blanche Descartes construction.

\begin{note}
	Blanche Descartes is the pen name of 4 undergrads at Cambridge in 1935. One of them was W. T. Tutte, who went on to become a founder of modern discrete math. He was also a codebreaker in World War 2.
\end{note}

\begin{proof}
	Our plan is to start with a triangle-free graph \( G \) with \( \chi(G) = k \), and construct a triangle-free graph \( BD(G) \) with \( \chi (BG(G))= k+1 \). One way we could do this is by adding a vertex adjacent to every vertex in \( G \). However, a problem occurs: adding this vertex creates lots of triangles. \par
	Instead, we can connect all vertices in \( G \) to separate other vertices, where all of those vertices need to have the same color. How do we do this? We use many copies of \( G \). \par
	Let \( n \coloneq |V(G)| \), and \( k \coloneq \chi (G) \). Define \( N \coloneq k\cdot (n - 1) + 1 \). Define \( r \coloneq \binom{N}{n} \). Take a set of vertices \( X \) of size \( N \). List all \( n \)-element subsets of \( X \) as \( S_{1}, S_{2}, \ldots , S_r \). Let \( G_{1}, G_{2}, \ldots , G_r \) be copies of \( G \), disjoint from each other and from \( X \). Note that \( |V(G_i)| = n = |S_i| \) such that we can connect vertices in \( G_i \) to \( S_i \) by \( n \) disjoint edges. The resulting graph is \( BD(G) \).
\end{proof}

\exercise{1}
What is \( BD(K_2) \)?
