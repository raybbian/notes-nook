\lecture{19}{Thu 02 Nov 2023 14:10}{Modular Arithmetic Continued}

How do we solve \( \overline{a}\overline{x}=\overline{b}    \) in \( \mathbb{Z}_m \)?

It suffices to solve the cases where \( \overline{b}=\overline{0}   \) and \( \overline{b}=\overline{1}   \). We first check \( \gcd(a,m)=d \).

\begin{eg}
	Find \( \gcd(30,12) \).
\end{eg}

By the Euclidean algorithm, we have
\begin{align*}
	& \gcd(30, 18) \tag{\( 30=18\cdot 1 + 12 \)} \\
	&= \gcd(18, 12) \tag{\( 18=12\cdot 1 + 6 \)} \\
	&= \gcd(12, 6) \tag{\( 12=6\cdot 2 + 0 \)} \\
	&= \gcd(6,0) = 6
.\end{align*}

Then, let us go back to the spot where we have a remainder of 6. Then, we get a linear combination \[
	6 = 18 \cdot 1 + 12 \cdot (-1)
.\] We also get from one step above that: \[
	6 = 18 \cdot 1 + (30 \cdot 1 + 18 \cdot -2) \cdot (-1) = 30 \cdot (-1) + 18 \cdot 3
.\] In other words, we have now found a method for finding the coefficients of the linear combination that represents the gcd in terms of the two numbers we started with. 

\begin{definition}	
	Let \( d=\gcd(a,b) \). Then, there exists numbers \( s,t \in \mathbb{Z} \) such that \( d=as+bt \). These are known as \textbf{Bezout coefficients}.
\end{definition}

We can use the Extended Euclidean Algorithm to find one such coefficient.

\begin{algorithm}
	\caption{Extended Euclidean Algorithm}
	\KwIn{\( (a,b), a \ge b \)}
	\KwOut{\( (d,s,t) \) such that \( d=\gcd(a,b) \), \( d=as+bt \)}
	\( r_0=a \)\;
	\( r_1=b \)\;
	\( s_0=1 \)\;
	\( s_1=0 \)\;
	\( t_0=0 \)\;
	\( t_1=1 \)\;
	\While{\( r_k\neq 0 \)}{
		\( r_{k+1}=r_{k-1} \pmod {r_k} \)\;
		\( s_{k+1}=s_{k-1} - (r_{k-1} \text{ div } r_k)\cdot s_k \)\;
		\( t_{k+1}=t_{k-1} - (r_{k-1} \text{ div } r_k) \cdot t_k \)\;
	}
	return \( (r_{k-1}, s_{k-1}, t_{k-1}) \)\;
\end{algorithm}

\begin{remark}
	If \( d=1 \), then \( s \) is the inverse of \( a \pmod b \).
\end{remark}

We can also solve systems of linear equations in \( \mathbb{Z}_m \). Let's say we wish to find a number such that we can satisfy \[
	\begin{pmatrix}
		x &\equiv a_1 \pmod {m_1} \\
		x &\equiv a_2 \pmod {m_2} \\
			& \hdots \\
		x &\equiv a_k \pmod {m_k} \\
	\end{pmatrix}
.\] 

\begin{theorem}
	(Chinese Remainder Theorem) Let \( m_{1},m_{2},\ldots m_k \in \mathbb{N}\) be numbers such that \( \gcd(m_i,m_j)=1 \) for all \( 1 \le  i < j \le  k \). Then, the system has a unique solution in \( \mathbb{Z}_M \) where \( M=m_{1}m_{2}\ldots m_k \).
\end{theorem}

\begin{proof}
	First, we will prove the solution's existence. Set \( M_i =\frac{M}{m_i}\). Because all \( m \)'s don't share a factor, \( \gcd(m_i, M_i) = 1\). Let \( y_i \) be the inverse of \( M_i \pmod {m_i}\). Consider: \[
		x = M_1 \cdot y_{1} \cdot  a_{1} + M_{2} \cdot y_{2}\cdot a_{2}+ \ldots + M_{k} \cdot y_{k} \cdot a_{k}
	.\] Then, we check: 
	\begin{align*}
		x &\equiv M_{1}\cdot y_{1}\cdot a_{1}+ \cancelto{0}{M_{2}\cdot y_{2}\cdot a_{2}} + \ldots  + \cancelto{0}{M_k \cdot  y_k \cdot a_k} \pmod {m_1} \\
			& \equiv \cancelto{1}{M_{1}\cdot y_{1}}\cdot a_{1} \equiv a_{1} \pmod {m_1}
.\end{align*}
  We can reprove this for each \( 2\le i\le k \).
\end{proof}

\begin{eg}
	Solve \[
		\begin{matrix}
			x & \equiv 3 \pmod 5 \\
			x & \equiv 2 \pmod 7
		\end{matrix}
	.\] 
\end{eg}

We first check that \( \gcd(5,7)=1 \), which it is.
Then, we continue as follows:
\begin{align*}
	M&=5\cdot 7=35 \\
	M_1 &= \frac{35}{5} = 7 \\
	M_2 &= \frac{35}{7} = 5 \\
	y_{1} &= 3 \tag{Inverse of 7 mod 5} \\
	y_{2} &= 3 \tag{Inverse of 5 mod 7} \\
.\end{align*}
Therefore, the solution is as follows: 
\begin{align*} 
	x &= M_{1} \cdot  y_{1} \cdot  a_{1} + M_{2} \cdot  y_{2} \cdot  a_{2} \\
		&= 7 \cdot  3 \cdot 3 + 5 \cdot 3 \cdot  2 = 93\\
		& \equiv 23 \pmod {35}
.\end{align*}

\begin{note}
	This solution (\( x \equiv 93 \pmod {35} \)) is the only solution, which we will prove in our homework.
\end{note}
