\lecture{24}{Tue 28 Nov 2023 14:01}{}

\subsection{With Repetition}

\begin{eg}
	Let \( n \) be the number of objects, and \( t_{1},t_{2},\ldots ,t_k \) denote the \( k \) types. We want to find a permutation of these objects such that  \( t_{1} + t_{2} + \ldots  + t_k = n \). How should we do this?
\end{eg}

We can use the division rule, dividing out every permutation between identical objects. There are \( t_{i} \) ways to permute objects of type \( i \). Therefore, our answer is \[
	\frac{n!}{t_{1}!t_{2}!\ldots t_k!}
.\] 

\begin{eg}
	What if we don't want to find a permutation of these objects, but instead a combination?
\end{eg}

We visualize this combination as a permutation of stars and bars. There are total of \( n+k-1 \) stars and bars, of which \( n \) are stars and \( k-1 \) are bars. Therefore, our answer is then \[
	\frac{(n+k-1)!}{n!(k-1)!} = \binom{n+k-1}{k-1}
.\] 

TODO: Complete the rest of these notes.
