\lecture{1}{Tue 09 Jan 2024 15:30}{Introduction}

\section{Big Ideas}

\begin{definition}
	Big Idea 1: All computers can compute the same kinds of things. We call this \textbf{turing-completeness}.
\end{definition}

\begin{definition}
	Big Idea 2: \textbf{Abstraction} represents the layers that make the electorns work.
\end{definition}

\begin{definition}
	Big Idea 3: \textbf{Binary numbers} are better than decimal for electronic computing. Saves energy and is easier to distinguish.
\end{definition}

\begin{definition}
	Big Idea 4: Computers store \textbf{finite-sized representations} of data and information.
\end{definition}

\section{Data Types}

What does the number 5 mean to us? Does it mean the character 5, the integer 5, or the floating point number 5.0? It could also be represented by roman numberals, etc. These are all many different representations of the idea of ``5''.

\begin{definition}
	A \textbf{data representation} is the set of values from which something can take its value. It also includes the meaning of those values
\end{definition}

\begin{eg}
	These representations can be integers, floats, instructions, pointers, addresses, etc. All of these types have various flavors depending on the size, etc.
\end{eg}

You can usually expect the processor to deal with data representations gracefully.

\begin{definition}
	A \textbf{bit} quite literally, is a charge on the capacitor in the DRAM. It is short for ``binary digit'' and takes on the value of 0 or 1.
\end{definition}

\begin{eg}
	How many bits would we need to store 5 different items?
\end{eg}
\begin{explanation}
	3, as \( 2^2 \le 5\le 2^3 \).
\end{explanation}

\begin{eg}
	How many different numbers can be represented by 7 bits?
\end{eg}
\begin{explanation}
	\( 2^7=128 \) numbers.
\end{explanation}

\begin{definition}
	\textbf{Unsigned integers} are represented in bits as a base 2 number mathematically.
\end{definition}

For binary numbers, addition and subtraction are defined in exactly the same way. However, you have to carry the bits more often.
