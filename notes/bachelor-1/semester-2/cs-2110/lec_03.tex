\lecture{3}{Tue 16 Jan 2024 15:32}{Hexadecimal and Octal}

Octal is just like binary, but in base 8 as well. However, grouping bits in 3 is kind of weird. So instead, we use hexadecimal, which is base 16 (1...F). This way, we can group bits into 4.

We can define constant hex numbers with 0x, oct numbers with 0, and binary numbers (sometimes) with 0b.

\subsection{Bit Vectors}

To clear a bit vector, we can \& it will all 0s. To set a bit vector, we can | it with 1s. To test a certain bit, you can mask the other bits away, and check the singular bit.

To put 1 in any bit position \( n \) in a mask, shift left by \( n \).

\begin{eg}
	\( 1 << 2 = 0100_2 \)
\end{eg}

\subsection{ASCII}

\begin{eg}
	How do we transform `P' to `p'?
\end{eg}
\begin{explanation}
	We can do `P' + 32, `P' + 0x20, `P' | 0b00010000, etc.
\end{explanation}

\subsection{Unicode}

Now, we give other languages characters as well. Each character get a number ("code point"), and there are now 1,112,064 code points. It is also backwards compatible with ASCII.
