\lecture{2}{Thu 11 Jan 2024 15:39}{Two's Complement}

Multiplication in binary can be done like normal multiplication, but it is a lot faster. Note that we don't even need a microprocessor to add numbers. It can be done purely with transistors.

\begin{definition}
	The \textbf{not} operator, denoted \( \neg \), toggles the value of a variable.
\end{definition}

\subsection{Two's Complement}

How do we represent negative numbers? We could have signed magnitude, where 1 bit is used on the left. We could also use 1's complement, in which the negative number is just the negated bit value of the positive.

\begin{definition}
	\textbf{Two's Complement} wraps back around after the maxmimum number to the least negative number. You can find the negative value of a number by negating it and adding 1.
\end{definition}

To extend the side of a number, we fill in to the left copies of the leading bit.

\begin{eg}
	Sum of \( 101111_2 \) and \( 001010_2 \) is \( 111001_2 \).
\end{eg}

What should we do when we overflow?
