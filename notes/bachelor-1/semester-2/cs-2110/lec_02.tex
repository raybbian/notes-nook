\lecture{2}{Thu 11 Jan 2024 15:39}{Two's Complement}

Multiplication in binary can be done like normal multiplication, but it is a lot faster. Note that we don't even need a microprocessor to add numbers. It can be done purely with transistors.

\begin{definition}
	The \textbf{not} operator, denoted \( \neg \), toggles the value of a variable.
\end{definition}

\subsection{Two's Complement}

How do we represent negative numbers? We could have signed magnitude, where 1 bit is used on the left. We could also use 1's complement, in which the negative number is just the negated bit value of the positive.

\begin{definition}
	\textbf{Two's Complement} wraps back around after the maxmimum number to the least negative number. You can find the negative value of a number by negating it and adding 1.
\end{definition}

\begin{eg}
	Sum of \( 101111_2 \) and \( 001010_2 \) is \( 111001_2 \).
\end{eg}

What should we do when we overflow? When we add two positive numbers, and get a negative result, we have probably overflowed. When we add two negative numbers, and get a positive number, we have probably underflowed.

\subsubsection{Sign extension}

To extend the side of a number, we fill in to the left copies of the leading bit. If the number is negative, this bit will be 1.

\subsubsection{Doubling}

When we double a number, we just shift the number to the left. When quadrupling, we can shift the number twice to the left. The operator for this in many languages is <<.

\subsection{Fractional Binary Numbers}

Fractional binary numbers can be denoted by a place for the decimal point, and bits for the number itself otherwise.

\section{Logical Operators}

There are many binary logical operators, including AND, OR, NOT, XOR, NAND, NOR, and the shift operators, etc.

\begin{definition}
	The \textbf{AND} operator results in 1 only if both are 1.
\end{definition}

\begin{definition}
	The \textbf{OR} operator results in 1 if either are 1.
\end{definition}

\begin{definition}
	The \textbf{XOR} operator results in 1 if exactly 1 input bit is 1.
\end{definition}

\begin{definition}
	The \textbf{NOT} operator results in 0 if the single input bit is 1, and vice versa.
\end{definition}

Other logical operators can be done by adding the not operator to previous operators.
