\lecture{7}{Tue 30 Jan 2024 15:33}{Clock Cycles and State Machines}

We instead need to use two latches to copy data into the leader, then copy to the follower. This is known as \textbf{write-enable}, which is sort of like a flip flop between two D-latches. It's sort of like a landing bay in a space ship.

\begin{note}
	We must build our own D-latches in circuitsim.
\end{note}

\subsection{State Machines}

We will now create state machines!

\begin{definition}
	A \textbf{state machine} includes states and transitions.
\end{definition}

\begin{eg}
	A garage door state machine includes a Motor Up, Stop (Down Next), Motor Down, and Stop (Up Next) state.
\end{eg}

We need something to hold state, combinational logic, a clock, a reset button, limit switches, and push button. We also need to handle being in no state. We should also add dummy states for when the button is pressed down.

To implement the state machine logic, we need to create a truth table for every state. We consider the circuit inputs, the current state, next state, and circuit outputs.
