\lecture{5}{Tue 23 Jan 2024 15:52}{Combinational Logic}

The process for making circuits is this: Once we have a working circuit, we use logic gates to implement it. Once that is finished, we can use boolean algebra to simplify our logic gate.

Here are some non-traditional rules for simplification where \( + \) is OR and \( \times  \) is AND:
\begin{enumerate}
	\item \( A + AB = A \)
	\item \( A + \overline{A}B = A + B \)
	\item \( (A + B)(A + C) = A + BC \)
\end{enumerate}

DeMorgan's Law can be combined with double negation to transform circuits into some with less transistors.

To simplify a circuit, OR together all cases which result in 1. Then, simplify it with boolean algebra.
