\lecture{6}{Thu 25 Jan 2024 17:02}{More on Spaces}

If the dimension of \( V = n \) and you have linearly independent vectors, then you have a basis for \( V \).

Most of the time, we will look at subspaces of \( \mathbb{R}^{n}  \).

\begin{note}
	If we have a subset \( V \) of \( \mathbb{R}^{n}  \), to show that \( V \) is a subspace all we must do is show that if \( x,y \in V \), \( \alpha ,\beta \in \mathbb{R} \), then \( \alpha x+\beta y \in V \).
\end{note}

\begin{eg}
	Some example with string/sine wave, insert from lecture notes later. The \( N \) vectors \( f_i \) for \( i=1\ldots N \) for a basis for \( \mathbb{R}^{N}  \), of which the proof is left as an exercise. This is also known as the Fourier basis.
\end{eg}

The definition of vector spaces can be given with \( \mathbb{C} \) instead of \( \mathbb{R} \). This allows us to talk about vector spaces over \( \mathbb{C} \).

\begin{eg}
	The typical example is \( \mathbb{C}^{n}  \). A vector \( x \in \mathbb{C}^{n}  \) is given by \[
		x = \begin{bmatrix}
			x_{1} \\ x_{2} \\ \vdots \\ x_n
		\end{bmatrix} \quad x_i \in \mathbb{C}
	.\] 
\end{eg}

Consider a subspace of \( \mathbb{R}^{n}  \). The simplest example is a plane, or a hyperplane (if \( n>3 \)). Let \( A \) be a \( 1\times n \) matrix such that \( A\neq 0 \), and \( V \) the set of solutions to \( Ax=0 \). Then, \( V \) is a subspace of \( \mathbb{R}^{n}  \). What is the dimension of \( V \)?
