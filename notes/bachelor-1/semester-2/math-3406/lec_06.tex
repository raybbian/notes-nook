\lecture{6}{Thu 25 Jan 2024 17:02}{More on Spaces}

If the dimension of \( V = n \) and you have linearly independent vectors, then you have a basis for \( V \).

Most of the time, we will look at subspaces of \( \mathbb{R}^{n}  \).

\begin{note}
	If we have a subset \( V \) of \( \mathbb{R}^{n}  \), to show that \( V \) is a subspace all we must do is show that if \( x,y \in V \), \( \alpha ,\beta \in \mathbb{R} \), then \( \alpha x+\beta y \in V \).
\end{note}

\begin{eg}
	Some example with string/sine wave, insert from lecture notes later. The \( N \) vectors \( f_i \) for \( i=1\ldots N \) for a basis for \( \mathbb{R}^{N}  \), of which the proof is left as an exercise. This is also known as the Fourier basis.
\end{eg}

The definition of vector spaces can be given with \( \mathbb{C} \) instead of \( \mathbb{R} \). This allows us to talk about vector spaces over \( \mathbb{C} \).

\begin{eg}
	The typical example is \( \mathbb{C}^{n}  \). A vector \( x \in \mathbb{C}^{n}  \) is given by \[
		x = \begin{bmatrix}
			x_{1} \\ x_{2} \\ \vdots \\ x_n
		\end{bmatrix} \quad x_i \in \mathbb{C}
	.\] 
\end{eg}

Consider a subspace of \( \mathbb{R}^{n}  \). The simplest example is a plane, or a hyperplane (if \( n>3 \)). Let \( A \) be a \( 1\times n \) matrix such that \( A\neq 0 \), and \( V \) the set of solutions to \( Ax=0 \). Then, \( V \) is a subspace of \( \mathbb{R}^{n}  \). What is the dimension of \( V \)?

The dimension of \( V \) should be \( n-1 \). Let \( x_{2}=1 \), and \( x_{3},\ldots ,x_n =0\). Let \( x_{1} = -\frac{a_{2}}{a_{1}} \). Then, \[
	\sum_{i=1}^{n} a_i x_i = a_1 \left( -\frac{a_{2}}{a_{1}} \right) + a_{2} = 0
.\] such that this is a solution. We can apply the same thing, instead setting \( x_i=1 \) to find solutions \( f_i \) such that \( Af_i=0 \) for \( i=1\ldots n-1 \). These vectors are linearly independent as the only values for \( \alpha  \) such that \[
	\alpha_2 f_{2} + \alpha _3f_{3} + \ldots  + \alpha _n f_n = 0
.\] is that \[
	\alpha _2 = \alpha _3 = \ldots  = \alpha _n = 0
.\] This tells us that \[
	\text{dim} V \ge n-1
.\] Note that if \( V \subset \mathbb{R}^{n}  \), then \( \text{dim}V < n \). Therefore, \[
	\text{dim}V = n-1
.\] 

\begin{note}
	Also note that the column vector \( A^{T}  \) is not in \( V \). This process also might not work for complex numbers, as \( A A^{T}  \) is not necessarily non-negative.
\end{note}

\begin{definition}
	The thing we constructed, \( V \), is called the \textbf{nullspace} of \( A \), denoted \( N(A) \). Similarly, \( \text{dim}N(A) = n-1  \).
\end{definition}

Now, let us consider \( A^{T}  \). 

\begin{definition}
	\( W = \Span A^{T}  \). \( \text{dim}A^{T} = 1  \).
\end{definition}

\begin{definition}
	If \( A \) is an \( n\times m \) matrix (\( m \) columns and \( n \) rows) in \( \mathbb{R}^{n}  \). Then, \( C(A) \) is the \textbf{column space} of \( A \), and is defined as \[
		\Span \{A_{1},A_{2},\ldots ,A_n\}  
	.\] 
\end{definition}

\begin{prop}
	If we have two subspaces of \( \mathbb{R}^{n}  \), \( V,W \), then \[
		\Span \{V, W\} = \{x + y : x \in V, y \in W\}  
	.\] We then say that \( \mathbb{R}^{n} = V \bigoplus W \) (called the direct sum).
\end{prop}
\begin{proof}
	Assume that \( V \cup W = \{0\}   \). Then, we can show that every vector in \( \Span \{V, W\}   \) can be written in a unique way as \( x+y \) where \( x \in V \), \( y \in W \).
\end{proof}

Back to our equation, let \( A \) be a \( 1\times n \) matrix and \( V = \{x : Ax=0\}   \). Let \( W = \Span A^{T} = C(A^{T} )  \). Then, we have \[
	V \cap W = \{0\} \implies V \bigoplus W = \mathbb{R}^{n} 
.\] In other words,

\begin{prop}
	We have shown so far that for a row matrix,
	\[
		N(A) \bigoplus C(A^{T} ) = \mathbb{R}^{n} 
	.\] 
\end{prop}

\begin{prop}
	If \( V \bigoplus W = \mathbb{R}^{n}  \), then \[
		\text{dim}V + \text{dim}W = n
	.\] 
\end{prop}
