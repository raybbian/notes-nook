\lecture{3}{Tue 16 Jan 2024 14:01}{Bayes Theorem and Random Variables}

Continuing on with conditional probabiltiy from last time,

\begin{eg}
	A total of \( n \) balls are sequentially and randomly chosen without replacement from an urn containing \( r \) red balls and \( b \) blue balls (\( n\le r+b \)). Given that \( k \) of the \( n \) balls are blue, what is the conditional probability that the first ball chosen is blue?
\end{eg}
\begin{explanation}
	We can use Lemma 2. Then, we have
	\begin{align*}
		\mathbb{P}(\text{first blue}) &= \frac{b}{r+b} \\
		\mathbb{P}(\text{first \( k \) are blue}) &= \frac{\binom{n}{k} P(b,k) P(r,n-k)}{P(r+b, n)}
	.\end{align*}
	\( \mathbb{P}(\text{\( k-1 \) of remaining \( n-1 \) slots are blue.}) \) = \[
		\frac{\binom{n-1}{k-1}P(b-1,k-1)\cdot P(r,n-k)}{P(r+b-1,n-1)}
	.\] \[
		\mathbb{P}(\text{\( k-1 \) of rest \( n-1 \) are blue}) \cdot \frac{\mathbb{P}(\text{first blue})}{\mathbb{P}(\text{first \( k \) are blue})}
	.\] will then be our answer.
\end{explanation}

\subsection{Bayes Theorem}

\begin{definition}
	A \textbf{partition} of \( \Omega  \) is a collection \( \{ B_i : i \in I \} \) of disjoint events with union \( \bigcup_i B_i = \Omega  \). 
\end{definition}

\begin{theorem}
	Let \( (\Omega, \mathcal{F}, \mathbb{P}) \) be a probability space. If \( \{B_{1},B_{2},\ldots \}   \) is such a parition with \( \mathbb{P}(B_i) > 0\), then \[
		\mathbb{P}(A) = \sum_i \mathbb{P}(A \mid B_i) \mathbb{P}(B_i) \quad \text{for } A \in \mathcal{F}
	.\] 
\end{theorem}

\begin{eg}
	Flip a fair coin. If heads, roll a 6-sided fair die. If tails, roll two 4-sided dice and sum the total. What is the overall likelihood of an outcome of 3?
\end{eg}
\begin{explanation}
	Look at the event tree, and count the probabilities. The heads case is \( \frac{1}{2} \cdot \frac{1}{6} \) and the tails case is \( \frac{1}{2}\cdot \frac{1}{8} \). This is an informal Bayes Theorem.
\end{explanation}

\begin{theorem}
	We can also rearrange Bayes' Theorem to yield
	\[
		\mathbb{P}(B_j \mid A) = \frac{\mathbb{P}(A \mid  B_j)\mathbb{P}(B_j)}{\sum_i \mathbb{P}(A\mid B_i) \mathbb{P}(B_i)}
	.\] 
\end{theorem}

\section{Random Variables}

\begin{definition}
	A \textbf{random variable} on \( (\Omega, \mathbb{P}) \), \( |\Omega | < \infty \) is a function \( X: \Omega  \to  \mathbb{R} \).
\end{definition}

\begin{notation}
	\( \{X = a\} = \{ \omega  \in \Omega : X(\omega ) = a \} = \ldots = X^{-1}(a) \).
\end{notation}

\begin{eg}
	3 balls are to be selected without replacement from an urn containing 20 balls numbered 1 to 20. What is the probability that at least one of the balls that are drawn has a number as large or larger than 17?
\end{eg}
\begin{explanation}
	\( \Omega =\{1,2,3,\ldots ,20\}   \). \( |\Omega | = \binom{20}{3} \). Let our random variable \( X : \Omega \to \mathbb{R} \), \( X=\text{largest of the three values} \). Let \( E=\{X\ge 17\}   \). Then, \( \mathbb{P}(E)=1 = \mathbb{P}(E^{c}) = \mathbb{P}(\text{all } < 17) \).
	\begin{align*}
		\mathbb{P}(\text{all } < 17) &= \frac{|E^{c}|}{|\Omega |} = \frac{\binom{16}{3}}{\binom{20}{3}} \\
		\mathbb{P}(E) &= 1 - \mathbb{P}(E^{c}) = 1 - \frac{\binom{16}{3}}{\binom{20}{3}}
	.\end{align*}
\end{explanation}

\begin{definition}
	\( X \) is called \textbf{discrete} if \( \exists  \) a countable set \( S \subset \mathbb{R} \) such that \( \mathbb{P}(X \in S) = 1 \).
\end{definition}

\begin{definition}
	The \textbf{probability mass function} \( p(a)=P(X = a) \) is positive for most a countable number of values of \( a \).
\end{definition}

\begin{eg}
	The pmf of a random variable \( X \) is given by \( p_X(i) = \frac{c\lambda ^i}{i!} \), \( i=0,1,2\ldots  \) where \( \lambda  \) is some positive value. What is \( \mathbb{P}(X=0) \) and \( \mathbb{P}(X>2) \)?
\end{eg}
\begin{explanation}
	\begin{align*}
		\sum p_X(i) &= \sum_{i=0}^{\infty} \frac{c\lambda ^i}{i!} = 1 \\
								&\implies X = \frac{1}{\sum_{k=0}^{\infty} \frac{\lambda ^i}{i!}} = e^{-\lambda } 
	.\end{align*}
	Then, \( \mathbb{P}(X=0)= p_X(0) = \frac{c\lambda ^{0} }{0!} = c = e^{-\lambda } \). Also, \( \mathbb{P}(X>2) = 1 - \mathbb{P}(X\le 2) \). 
	\begin{align*}
		\mathbb{P}(X\le 2) &= \mathbb{P}(X=0) + \mathbb{P}(X=1) + \mathbb{P}(X=2) \\
											 &= e^{-\lambda } \left( 1 + \lambda  + \frac{\lambda ^{2} }{2} \right) 
	.\end{align*}
\end{explanation}

\begin{definition}
	If \( X \) is a discrete random variable, the \textbf{expectation} of \( X \) is denoted by \( \mathbb{E}(X) \) and is defined by \[
		\mathbb{E}(X) = \sum_{x \in \text{Im} X} x\mathbb{P}(X=x)
	.\] 
\end{definition}

\begin{eg}
	We say that \( I \) is an indicator variable for the event \( A \) if \[
		I = \begin{cases}
			1 &\text{ if \( A \) occurs}\\
			0 &\text{ if \( A^{c} \) occurs}
		\end{cases}
	.\] Find \( \mathbb{E}(I) \).
\end{eg}
\begin{explanation}
	\begin{align*}
		\mathbb{E}(I_A) &= 0 \cdot \mathbb{P}(I_A=0) + 1 \cdot  \mathbb{P}(I_A = 1) \\
		&= \mathbb{P}(I_A=1) \\
		&= \mathbb{P}(\{\omega  \in \Omega : I_A(\omega ) = 1\}  ) \\
		&= \mathbb{P}(\{\omega  \in A\}  ) \\
		&= \mathbb{P}(A)
	.\end{align*}
\end{explanation}

\begin{eg}
	A class of 120 students is driven in 3 buses to a performance, with 36, 40, and 44 students in the busees. Let \( X \) denote the number of students on the bus of a randomly chosen student. Find \( \mathbb{E}(X) \).
\end{eg}
\begin{explanation}
	Note that \( \mathbb{P}(B_{1}) = \frac{36}{120} \), \( \mathbb{P}(B_{2}) = \frac{40}{120} \) and \( \mathbb{P}(B_{3})=\frac{44}{120} \). Then, 
	\begin{align*}
		\mathbb{E}(X) &= 36 \cdot \frac{36}{120} + 40 \cdot \frac{40}{120} + 44 \cdot \frac{44}{120} \\
		&= \frac{36^{2}+40^{2}+44^{2}}{120} \\
	.\end{align*}
\end{explanation}
