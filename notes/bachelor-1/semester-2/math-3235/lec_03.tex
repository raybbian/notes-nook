\lecture{3}{Tue 16 Jan 2024 14:01}{}

Continuing on with conditional probabiltiy from last time,

\begin{eg}
	A total of \( n \) balls are sequentially and randomly chosen without replacement from an urn containing \( r \) red balls and \( b \) blue balls (\( n\le r+b \)). Given that \( k \) of the \( n \) balls are blue, what is the conditional probability that the first ball chosen is blue?
\end{eg}
\begin{explanation}
	We can use Lemma 2. Then, we have
	\begin{align*}
		\mathbb{P}(\text{first blue}) &= \frac{b}{r+b} \\
		\mathbb{P}(\text{first \( k \) are blue}) &= \frac{\binom{n}{k} P(b,k) P(r,n-k)}{P(r+b, n)}
	.\end{align*}
	\( \mathbb{P}(\text{\( k-1 \) of remaining \( n-1 \) slots are blue.}) \) = \[
		\frac{\binom{n-1}{k-1}P(b-1,k-1)\cdot P(r,n-k)}{P(r+b-1,n-1)}
	.\] \[
		\mathbb{P}(\text{\( k-1 \) of rest \( n-1 \) are blue}) \cdot \frac{\mathbb{P}(\text{first blue})}{\mathbb{P}(\text{first \( k \) are blue})}
	.\] will then be our answer.
\end{explanation}
