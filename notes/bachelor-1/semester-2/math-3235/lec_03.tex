\lecture{3}{Tue 16 Jan 2024 14:01}{}

Continuing on with conditional probabiltiy from last time,

\begin{eg}
	A total of \( n \) balls are sequentially and randomly chosen without replacement from an urn containing \( r \) red balls and \( b \) blue balls (\( n\le r+b \)). Given that \( k \) of the \( n \) balls are blue, what is the conditional probability that the first ball chosen is blue?
\end{eg}
\begin{explanation}
	We can use Lemma 2. Then, we have
	\begin{align*}
		\mathbb{P}(\text{first blue}) &= \frac{b}{r+b} \\
		\mathbb{P}(\text{first \( k \) are blue}) &= \frac{\binom{n}{k} P(b,k) P(r,n-k)}{P(r+b, n)}
	.\end{align*}
	\( \mathbb{P}(\text{\( k-1 \) of remaining \( n-1 \) slots are blue.}) \) = \[
		\frac{\binom{n-1}{k-1}P(b-1,k-1)\cdot P(r,n-k)}{P(r+b-1,n-1)}
	.\] \[
		\mathbb{P}(\text{\( k-1 \) of rest \( n-1 \) are blue}) \cdot \frac{\mathbb{P}(\text{first blue})}{\mathbb{P}(\text{first \( k \) are blue})}
	.\] will then be our answer.
\end{explanation}

\subsection{Bayes Theorem}

\begin{definition}
	A \textbf{partition} of \( \Omega  \) is a collection \( \{ B_i : i \in I \} \) of disjoint events with union \( \bigcup_i B_i = \Omega  \). 
\end{definition}

\begin{theorem}
	Let \( (\Omega, \mathcal{F}, \mathbb{P}) \) be a probability space. If \( \{B_{1},B_{2},\ldots \}   \) is such a parition with \( \mathbb{P}(B_i) > 0\), then \[
		\mathbb{P}(A) = \sum_i \mathbb{P}(A \mid B_i) \mathbb{P}(B_i) \quad \text{for } A \in \mathcal{F}
	.\] 
\end{theorem}

\begin{eg}
	Flip a fair coin. If heads, roll a 6-sided fair die. If tails, roll two 4-sided dice and sum the total. What is the overall likelihood of an outcome of 3?
\end{eg}
\begin{explanation}
	Look at the event tree, and count the probabilities. The heads case is \( \frac{1}{2} \cdot \frac{1}{6} \) and the tails case is \( \frac{1}{2}\cdot \frac{1}{8} \). This is an informal Bayes Theorem.
\end{explanation}

\begin{theorem}
	We can also rearrange Bayes' Theorem to yield
	\[
		\mathbb{P}(B_j \mid A) = \frac{\mathbb{P}(A \mid  B_j)\mathbb{P}(B_j)}{\sum_i \mathbb{P}(A\mid B_i) \mathbb{P}(B_i)}
	.\] 
\end{theorem}

\section{Random Variables}

\begin{definition}
	A \textbf{random variable} on \( (\Omega, \mathbb{P}) \), \( |\Omega | < \infty \) is a function \( X: \Omega  \to  \mathbb{R} \).
\end{definition}

\begin{notation}
	\( \{X = a\} = \{ \omega  \in \Omega : X(\omega ) = a \} = \ldots = X-1(a) \).
\end{notation}
