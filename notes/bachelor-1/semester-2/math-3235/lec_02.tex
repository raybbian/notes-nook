\lecture{2}{Thu 11 Jan 2024 13:56}{More Probabiliy}

\subsection{Properties of Event Spaces}

\begin{definition}
	A collection \( \mathcal{F} \) of subsets of the sample space \( \Omega  \) is called an \textbf{event space} if 
	\begin{itemize}
		\item \( \mathcal{F} \) is non-empty.
		\item if \( A \in \mathcal{F} \) then \( \Omega \setminus A \in \mathcal{F} \).
		\item if \( A_{1},A_{2},\ldots \in \mathcal{F} \) then \( \bigcup_{i=1}^{\infty}A_i \in \mathcal{F} \).
	\end{itemize}
\end{definition}

\begin{theorem}
	If \( A \in \mathcal{F} \), then \( \mathbb{P}(A) + \mathbb{P}(\Omega \setminus A) \) = 1
\end{theorem}
\begin{proof}
	Notice that \( A \) and \( \Omega  \setminus  A \) are disjoint. And, that \( A \cup (\Omega  \setminus A) = \Omega \). Then,	\[
		\mathbb{P}(A \cup (\Omega \setminus A)) = \mathbb{P}(\Omega ) = 1
	.\] 
\end{proof}

\begin{theorem}
	If \( A,B \in \mathcal{F} \) then \( \mathbb{P}(A \cup B) + \mathbb{P}(A \cap B) = \mathbb{P}(A) + \mathbb{P}(B)\).
\end{theorem}
\begin{proof}
	Note that \( A \cup B = (A \setminus B) \cup (A \cap B) \cup (B \setminus A) \). This is a union of disjoint sets, such that \( \mathbb{P}(A \cup B) = \mathbb{P}(A \setminus B) + \mathbb{P}(A \cap B) + \mathbb{P}(B \setminus A) \). Then, we have \( \mathbb{P}(A \cup B) + \mathbb{P}(A \cap B) = \mathbb{P}(A \setminus B) + \mathbb{P}(A \cap B) + \mathbb{P}(B \setminus A) + \mathbb{P}(A \cap B)\), of which the RHS simplifies to \( \mathbb{P}(A) + \mathbb{P}(B) \).
\end{proof}

\begin{theorem}
	If \( A, B \in \mathcal{F} \), and \( A \subseteq B \), then \( \mathbb{P}(A) \le  \mathbb{P}(B) \).
\end{theorem}
\begin{proof}
	We wish to show \( A \subseteq B \implies \mathbb{P}(A) \le \mathbb{P}(B) \). Then, \( B = (B \setminus A) \cup  (B \cap A) = (B \setminus A) \cup  A \), such that \( \mathbb{P}(B) = \mathbb{P}(B \setminus A) + \mathbb{P}(A) \ge \mathbb{P}(A)\) because \( \mathbb{P}(B \setminus A) \ge  0 \).
\end{proof}

\subsection{Examples}

\begin{eg}
	What is the probability that one is dealt a full house?
\end{eg}
\begin{explanation}
	This is the number of ways one can get a full house, divided by the total number of poker hands (5 card). The total number of poker hands is \( \binom{52}{5} \). The number of full houses is \( \frac{52 \cdot \binom{3}{2} \cdot 48 \cdot 3}{2!3!} \). Another way we can count the number of full houses is \( \binom{13}{1} \cdot \binom{4}{3} \cdot \binom{12}{1} \cdot \binom{4}{2}\). The result of the division is our answer.
\end{explanation}

\begin{eg}
	A box contains 3 marbles, 1 red 1 green and 1 blue. Consider an experiment that cnsists of us taking 1 marble, replacing it, and drawing another marble. What is the sample space?
\end{eg}
\begin{explanation}
	\begin{align*}
		\Omega = & \{(r,r),(r,b),(r,g), \\ & (b,r),(b,g),(b,b), \\ & (g,r),(g,g),(g,b)\}
	.\end{align*}
\end{explanation}

\begin{eg}
	What about if we don't replace the first marble?
\end{eg}
\begin{explanation}
	Everything without \( (r,r),(b,b),(g,g) \).
\end{explanation}

\begin{eg}
	What is the probability of being dealt a flush?
\end{eg}
\begin{explanation}
	This is just number of flushses divided by number of poker hands. The number of flushes is \( \binom{4}{1} \cdot \binom{13}{5} \).
\end{explanation}

\begin{eg}
	What is the probability of being dealt a straight?
\end{eg}
\begin{explanation}
	We can do the probability of any straight, minus probability of straight flush. The number of straights is 10 number-wise. Therefore, the number of straights is \( 10 \cdot (4^5 - 4) \). The probability can be then calculated.
\end{explanation}

\begin{eg}
	An urn contains \( n \) balls. If \( k \) balls are withdrawn one at a time, what is the probability that a special ball is chosen?
\end{eg}
\begin{explanation}
	\( \mathbb{P}(\text{special})  = 1 - \mathbb{P}(\text{special}^{c})\). If the special ball is not chosen, it would be \( \frac{(n-1)!}{(n-k-1)!} \). The total number of withdrawings is \( \frac{n!}{k!} \). Then, the total probability is \( 1-\frac{n-k}{n} \).
\end{explanation}

\begin{eg}
	If \( n \) people are present in a room, what is the prob that no two celebrate their birthday on the same date? How large must \( n \) be such that this probability is \( <\frac{1}{2} \).
\end{eg}
\begin{explanation}
	\( \mathbb{P}(\text{no people with same birthday}) \) is the number of no same birthday situations divided by the number of possibilities. Total possibilities is \( 365^n \). No same birthday situatons is \( \mathbb{P}(365,n) = \frac{365!}{(365-n)!} \). For the second question, \( n=23 \).
\end{explanation}

\subsection{Conditional Probability}

\begin{definition}
	If \( A,B \in \mathcal{F} \) and \( \mathbb{P}(B) > 0 \) then the \textbf{conditional probability} if \( A \) given \( B \) is denoted by \( \mathbb{P}(A\mid B) \) and defined by \[
		\mathbb{P}(A \mid B) = \frac{\mathbb{P}(A \cap B)}{\mathbb{P}(B)}
	.\] 
\end{definition}

\begin{theorem}
	If \( B \in \mathcal{F} \) and \( \mathbb{P}(B) > 0 \) then \( (\Omega, \mathcal{F}, \mathbb{Q}) \) is a pobability space where \( \mathbb{Q}: \mathcal{F} \to \mathbb{R} \) is defined by \( \mathbb{Q}(A) = \mathbb{P}(A \mid B) \)
\end{theorem}

\begin{eg}
	Let's say a coin is flipped twice. What is the conditional probability that both flips land on heads, given that the first flip lands on heads?
\end{eg}
\begin{explanation}
	\( \frac{\mathbb{P}(\text{two heads } \cap \text{ first heads})}{\mathbb{P}(\text{first heads})} = \frac{\mathbb{P}(\text{two heads})}{\mathbb{P}(\text{first heads})}\). This is just \( \frac{1}{2} \).
\end{explanation}

\begin{eg}
	What if given at least one lands on heads?
\end{eg}
\begin{explanation}
	\( \frac{\mathbb{P}(\text{two heads } \cap \text{ at least one head})}{\mathbb{P}(\text{at least one head})} = \frac{\mathbb{P}(\text{two heads})}{\mathbb{P}(\text{at least one head})} = \frac{2}{3}\). 
\end{explanation}

\begin{eg}
	In the card game bridge, the 52 cards are dealt equally. If North and South have a total of 8 spades among them, what is the probability that East has 3 of the 5 remaining spades?
\end{eg}
\begin{explanation}
	No rule: \( \mathbb{P}(\text{E has 3 spades}) = \frac{\binom{5}{3} \cdot \binom{21}{10}}{\binom{26}{13}} \).
\end{explanation}

\begin{theorem}
	Probability of intersection of three sets (insert from canvas).
\end{theorem}

\begin{definition}
	We call two events \( A,B \) \textbf{independent} if the occurence of one does not affect the other. Formally, \[
		\mathbb{P}(A \mid B) = \mathbb{P}(A) \text{ and } \mathbb{P}(B \mid A) = \mathbb{P}(B)
	.\] We can also check that \( \mathbb{P}(A \cap B) = \mathbb{P}(A)\cdot \mathbb{P}(B) \).
\end{definition}

\begin{eg}
	Flip three fair coins. What is likelihood that all three come up heads?
\end{eg}
\begin{explanation}
	With the sample space approach: \( \Omega =\{H,T\}^3  \). Of 8 total elementary events, 1 has three heads, so the probability is \( \frac{1}{8} \).

	With independence: we know that each event is independent, and all events are \( \frac{1}{2} \), so the probability is \( \left( \frac{1}{2} \right) ^3 =\frac{1}{8}\).
\end{explanation}

\begin{definition}
	Independence can be expanded to more than just two events (insert from canvas). However, note that events can be pairwise independent, but may not be all together independent.
\end{definition}

\begin{lemma}
	\[
		\mathbb{P}(B\mid A)=\mathbb{P}(A\mid B)\frac{\mathbb{P}(B)}{\mathbb{P}(A)}
	.\] 
\end{lemma}
\begin{proof}
	The RHS is the same as \( \frac{\mathbb{P}(A \cap B)}{\mathbb{P}(B)} \cdot \frac{\mathbb{P}(B)}{\mathbb{P}(A)} = \frac{\mathbb{P}(A \cap B)}{\mathbb{P}(A)} = \mathbb{P}(B\mid A) \).
\end{proof}

\begin{eg}
	There are \( n \) balls that are sequentially chosen without replacement from \( r \) red balls and \( b \) blue balls. Given that \( k \) of the \( n \) balls are blue, what is the conditional probability that the first chosen is blue?
\end{eg}
\begin{explanation}
	\begin{align*}
		& \mathbb{P}(\text{first is blue} \mid k\text{ are blue}) \\
	  &= \mathbb{P}(k\text{ are blue}\mid \text{first is blue}) \\
	  &\cdot \frac{\mathbb{P}(\text{first is blue})}{\mathbb{P}(k\text{ are blue})}\ldots  \\
	\end{align*}
\end{explanation}

