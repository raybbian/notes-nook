\lecture{7}{Tue 30 Jan 2024 14:02}{}

\begin{eg}
	Suppose that \( n+m \) independent trials with probability of success \( p \) are performed. If \( X \) is the number of successes of the first \( n \), and \( Y \) is the number of successes of the last \( m \), then \( X \) and \( Y \) are independent.
\end{eg}
\begin{explanation}
	Look at \( p_{x,y}(X=x,Y=y) \). We wish to show that this equals \( p_X(x) \cdot p_Y(y) \). Let 1 be success, 0 be failure. Then, \( \Omega = \{0,1\} ^{n+m} = (a = \{0,1\} ^{n}  , b=\{0,1\} ^{m}  )  \). Then 
	\begin{align*}
		\mathbb{P}((a, b)) &= p^{x+y} \cdot (1-p)^{m + n - (x + y)}  \\
		&= p^{x}\cdot (1-p)^{n-x} \cdot p^{y}\cdot (1-p)^{m-y}
	.\end{align*}
	Therefore, 
	\begin{align*}
		\mathbb{P}(X=x,Y=y) = \sum_{(a,b) \in \{X=x,Y=y\}  } \mathbb{P}((a,b))
	.\end{align*}
\end{explanation}
