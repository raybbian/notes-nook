\lecture{11}{Thu 15 Feb 2024 14:01}{}

\begin{definition}
	The \textbf{hazard rate}/failure rate is given by \[
		\lambda (t) = \frac{f(t)}{\overline{F}(t)}
	.\] where \( \overline{F} = 1 - F \). Suppose that an item has survived for a time \( t \) and we wish to find the probability that it will not survive for a longer \( dt \). Then \( \lambda (t) \) is the probability intensity that a \( t \) old item will die/fail.
\end{definition}

\begin{eg}
	What is the hazard rate of the exponential function?
\end{eg}
\begin{explanation}
	We have \[
		f_x(t) = \gamma e^{-\gamma t}  \quad t \ge 0
	.\] and that \[
		F_X(t) = 1 - e^{-\gamma t} 
	.\] from a previous class. Then \[
		\lambda (t) = \frac{\gamma e^{-\gamma t} }{e^{-\gamma t} } = \gamma \quad  t > 0
	.\] It doesn't depend on \( t \) (memoryless)! Also note that working backwards, \[
		\overline{F}_X = Ce^{-\lambda t} 
	.\] 
\end{explanation}

\begin{eg}
	A distribution function of positive continuous random variable can be specified by giving its hazard rate function. If a random variable has linear hazard rate function, \[
		\lambda (t) = a + bt
	.\] then \[
		\overline{F}_X = Ce^{-at - \frac{bt ^{2} }{2}}  
	.\] Noting that \( F_X(0) = 0 \), then we can solve for \( C \)), which is 1.
\end{eg}

\begin{definition}
	Expectation of a continuous random variable is given by \[
		\mathbb{E}(X) = \int_{-\infty}^{\infty}x\cdot f_X(x)dx 
	.\] 
\end{definition}

\begin{theorem}
	(Law of the subconcious statistician) If \( X \) is a continuous random variable, then \[
		\mathbb{E}(g(X)) = \int_{-\infty}^{\infty}g(x)f_X(x)dx 
	.\] whenever this integral converges.
\end{theorem}

\begin{eg}
	If \( X \) has the exponential distribution with parameter \( \lambda  \), then \( \mathbb{E}(X) = \frac{1}{\lambda } \)
\end{eg}

\begin{eg}
	A stick of length 1 is split at a point \( U \) uniformly distributed over \( (0,1) \). Determine the expected length of the piece that contains the point \( p \), \( 0\le p\le 1 \).
\end{eg}
\begin{explanation}
	The answer is \( p - p^{2} + \frac{1}{2}  \).
\end{explanation}

\begin{lemma}
	For nonnegative random variable \( Y \), \[
		\mathbb{E}(Y) = \int_{0}^{\infty}\mathbb{P}(Y>y)dy 
	.\] 
\end{lemma}
\begin{proof}
	\begin{align*}
		\mathbb{E}(Y) &= \int_{0}^{\infty}\mathbb{P}(Y>y)  dy\\
		\mathbb{P}(Y>y) &=  \int_{y}^{\infty}f_Y(x)dx  \\
		\int_{0}^{\infty} \mathbb{P}(Y>y) dy &= \int_{0}^{\infty}\int_{y}^{\infty}f_Y(x)dxdy   \\
		&= \int_{0}^{\infty}\int_{0}^{x} f_Y(x)dy dx   \\
		&= \int_{0}^{\infty} \left(\int_{0}^{x} dy \right) f_Y(x) dx  \\
		&= \int_{0}^{\infty} x f_Y(x) dx 
	.\end{align*}
\end{proof}

\begin{eg}
	If \( X \) is continuous with density function \( f_X \), and \( g(x) = ax + b \) when \( a>0 \), then \( Y=g(X) = aX + b \) has distribution function given by
\end{eg}
\begin{explanation}
	Note that 
	\begin{align*}
		F_Y(t) &= \mathbb{P}(y\le t) \\
		&= \mathbb{P}(aX + b \le t) \\
		&= \mathbb{P}\left(X \le \frac{t-b}{a}\right) \\
		&= F_X\left(\frac{t-b}{a}\right) 
	.\end{align*}
	Note that \[
		f_Y(t) = \frac{d}{dt} F_X \left( \frac{t-b}{a} \right)  = \frac{1}{a} \cdot f_X\left(\frac{t-b}{a}\right)
	.\] 
\end{explanation}

\begin{theorem}
	If \( X  \) is a CRV with density function \( f_X \), and \( g \) is a strictly increasing and differentiable function from \( \mathbb{R} \) to \( \mathbb{R} \), then \( Y = g(X) \) has density function \[
		f_Y(y) = f_X(g^{-1}(y))\frac{d}{dy}[g^{-1}(y)]
	.\] 
\end{theorem}
