\lecture{4}{Thu 18 Jan 2024 14:03}{Random Variables}

\begin{prop}
	If \( X \) is a discrete random variables that takes on one of the values \( x_i \), \( I\ge 1 \), with respective proabilities \( p(x_i) \), then, for any real valued function \( g \), \[
		\mathbb{E}(g(X)) = \sum_{i} g(x_i)p(x_i)
	.\] In other words, \( g(X) \) is also a random variable.
\end{prop}
\begin{proof}
	Done with a change of variables.
\end{proof}

\begin{eg}
	Suppose \( t \) units of a product are ordered, and \( X = \text{number of units sold} \) is a random variable. Assume a net profit of 6 per unit and a net loss of \( l \) per unit left unsold. Compute \textit{expected profit}.
\end{eg}
