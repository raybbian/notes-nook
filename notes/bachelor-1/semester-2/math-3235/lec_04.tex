\lecture{4}{Thu 18 Jan 2024 14:03}{Random Variables and Expected Values}

\begin{prop}
	If \( X \) is a discrete random variables that takes on one of the values \( x_i \), \( I\ge 1 \), with respective proabilities \( p(x_i) \), then, for any real valued function \( g \), \[
		\mathbb{E}(g(X)) = \sum_{i} g(x_i)p(x_i)
	.\] In other words, \( g(X) \) is also a random variable.
\end{prop}
\begin{proof}
	Done with a change of variables.
\end{proof}

\begin{eg}
	Suppose \( t \) units of a product are ordered, and \( X = \text{number of units sold} \) is a random variable. Assume a net profit of \( b \) per unit and a net loss of \( l \) per unit left unsold. Compute \textit{expected profit}.
\end{eg}
\begin{explanation}
	Our profit function is then \( \gamma = bX - l(t-X) \). Then, \( \mathbb{E}(\gamma ) = \mathbb{E}(g(X)) \) where \( g(X) = (b+l)X - lt \). Then we have
	\begin{align*}
		\mathbb{E}(g(X)) &= \sum_{x \in \text{Im}X} g(x) \cdot \mathbb{P}_X(x) \\
										 &= (b+l)\underbrace{\sum_{x \in \text{Im}X} x \cdot \mathbb{P}_X(x)}_{\mathbb{E}(X)} \\ & \qquad - lt \underbrace{\sum_{x \in \text{Im}X} \mathbb{P}_X(x)}_{1}\\ 
		&= (b+l)\mathbb{E}(X) - lt
	.\end{align*}
\end{explanation}

\begin{definition}
	If \( X \) is a random variable with mean \( \mu  \), then the \textbf{variance} of \( X \), denoted by \( \Var(X) \) is defined by \[
		\Var(X) = \mathbb{E}((X - \mu )^{2} )
	.\] 
\end{definition}

\begin{prop}
	\( \Var(X) = \mathbb{E}(X^{2} ) + (\mathbb{E}(X))^{2} \)
\end{prop}
\begin{proof}
	We have 
	\begin{align*}
		\Var(X) &= \mathbb{E}((X - \mathbb{E}(X)^{2})) \\
						&= \mathbb{E}(X^{2} - 2x\cdot \mathbb{E}(X) + (\mathbb{E}(X))^{2}   ) \\
						&= \mathbb{E}(X^{2} ) - 2 \mathbb{E}(X) + (\mathbb{E}(X))^{2} \tag{\( \mathbb{E}(c) = c \) for constant \( c \)} \\
						&= \mathbb{E}(X^{2} ) - (\mathbb{E}(X))^{2}
	.\end{align*}
\end{proof}

\begin{prop}
	If \( X \) is a discrete random variable with finitely many values, then \( \Var(x) = 0 \iff X \equiv \mathbb{E}(X) \).
\end{prop}
\begin{proof}
	(\( \impliedby \)) Suppose \( X=\mathbb{E}(X) \) Then,
	\begin{align*}
		\mathbb{E}(X^{2} ) &= \sum_{i=1}^{n} x_{i}^{2}\mathbb{P}_X(x_i)  \\
		&= c^{2}\cdot \sum_{i=1}^{n} \mathbb{P}_X(x_i)  \\
		&= c^{2}
	.\end{align*}
	Plugging both sides back into \( \Var(X) \), we have \( \Var(X) = c^{2} - c^{2} = 0   \). (\( \implies \)) Suppose \( \Var(X) = 0 \). Then, 
	\begin{align*}
		\Var(X) &= \mathbb{E}[(X - \mathbb{E}[X] )^{2} ] = 0\\
						&= \underbrace{\sum_{i}(x_i - c)^{2} \cdot \mathbb{P}_X(i)}_{\text{every term \( \ge 0 \)}}  \\
						&\implies (x_i = c \quad \forall i) \implies \mathbb{E}(X) = c
	.\end{align*}
\end{proof}

Note that \( \Var(X) \) is very similar to standard deviation, and it measures the spread of how far apart data is from the mean.

\begin{definition}
	Let \( X:\Omega \to \mathbb{R} \) be a random variable. The \textbf{cumulative distributino function} (CDF) is defined as \[
		F_X(a) = \mathbb{P}(X \le a) = \mathbb{P}(X(\omega ) \in (-\infty, a])
	.\] 
\end{definition}

\begin{definition}
	We say \( X \sim \text{Bernoulli}(p) \) if \[
		\mathbb{P}(X=1) = p \quad \mathbb{P}(X=0) = 1-p \quad (p \in (0,1))
	.\] 
\end{definition}

\begin{eg}
	It is known that screws produced will be defective with probability 0.1. The company sells screws in packages of 10 and gives a refund if more than 1 screw is defective. What propotion of packages must the company refund?
\end{eg}
\begin{explanation}
	Let \( X \) represent the number of defective screws. We wish to find \( 1-\mathbb{P}(X\le 1) \). This is just \( 1 - \mathbb{P}(X=0) - \mathbb{P}(X = 1) \). Just apply the binomial formula to get your answer.
\end{explanation}

\begin{definition}
	A random variable \( X \) that takes on one of the values \(0,1,2,\ldots \) is said to be a \textbf{Poisson} random variable with parameter \( \lambda  \) if for some \( \lambda >0 \)
	\[
		p(i) = \mathbb{P}(X=i) = e^{-\lambda } \left( \frac{\lambda ^{i}}{i!}  \right) 
	.\] 
\end{definition}
