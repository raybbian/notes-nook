\lecture{6}{Thu 25 Jan 2024 14:02}{Multivariate Probability}

\section{Multivariate Probability}

Our objective is to treat random vectors \( (X,Y_{0} \in \mathbb{R}^{2} ) \) together as \[
	(X,Y) : \Omega ^{2} \to \mathbb{R}^{2} 
.\] 

\begin{definition}
	If \( X \) and \( Y \) are discrete random variables on \( (\Omega , \mathcal{F}, \mathbb{P}) \), the \textbf{joint probability mass function} \( P_{X,Y}(x,y) \) of \( X \) and \( Y \) is the function \[
		p_{X,Y} : \mathbb{R}^{2} \to [0, 1]
	.\] defined by 
	\begin{align*}
		&p_{X,Y}(x, y) \\ &= \mathbb{P}(\{\omega  \in \Omega : X(\omega ) = x \text{ and } Y(\omega )=y\}  )
	.\end{align*}
	and abbreviated \[
		p_{X,Y}(x, y) = \mathbb{P}(X=x, Y=y)
	.\] 
\end{definition}

\begin{note}
	The sum of all options still remains one.
\end{note}

\begin{eg}
	Two cards are drawn at random from a dech of 52 cards. If \( X \) denotes the number of aces drawn and \( Y \) denotes the number of kings, display the join mass function of \( X \), and \( Y \) in tabular form.
\end{eg}
\begin{explanation}
	Note that \( X = \{0,1,2\}, Y=\{0,1,2\}     \). Then, we have \begin{vmatrix}
		
	\end{vmatrix}
\end{explanation}
