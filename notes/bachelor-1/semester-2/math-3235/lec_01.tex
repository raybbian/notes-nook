\lecture{1}{Tue 09 Jan 2024 14:01}{Intro to Probability}

\section{Basics of Probability}

What data do you need to specify probability? You need the \textbf{set of all outcomes}, a list of everything that could possibly occur as a consequence, and the likelihood of each event.


\begin{eg}
	For a roll of a dice, the set of all outcomes would be \( \{1,2,3,4,5,6\}   \). The list could include things like ``the result is 3'', or ``the result is \( \ge 4 \)'', and the likelihood would be \( \frac{1}{6} \) for each of the results.
\end{eg}

\subsection{Basics of Set Theory}

\begin{definition}
	A \textbf{set} is an unordered collection of elements. \textbf{Elements} are objects within sets.
\end{definition}

\begin{definition}
	A set \( A \) is a \textbf{subset} of a set \( B \) if \( a \in A \implies a \in B \)
\end{definition}

\begin{definition}
	The \textbf{union} of two sets \( A \) and \( B \) is the collection of elements that are in \( A \) or \( B \).
\end{definition}

\begin{definition}
	The \textbf{intersection} of two sets \( A \) and \( B \) is the collection of elements that are in both \( A \) and \( B \).
\end{definition}

\begin{definition}
	The \textbf{complement} of a set \( A \) is everything not in \( A \).
\end{definition}

\begin{definition}
	A \textbf{finite set} is a set with finite number of elements.
\end{definition}

\begin{definition}
	The \textbf{cartesian} product of two sets \( A \) and \( B \) denoted \( A \times B \) is \[
		\{(a,b) \colon a \in A \land b \in B\}  
	.\] 
	Then, \( |A \times B| = |A| \cdot |B| \).
\end{definition}

\subsection{Back to Probability}

\begin{definition}
	A \textbf{sample space} is the set of al possible outcomes in an experiment.
\end{definition}

\begin{eg}
	The sample space \( \Omega  \) for a coin flip is \( \{H,T\}   \).
\end{eg}

Note that \textbf{events} are just subsets of the sample space, and \textbf{elementary events} are just elements of the sample space.

\begin{eg}
	For a dice roll:	\( \Omega =\{1,2,3,4,5,6\}   \), some events could be \( \{1,2\}   \), \( \{3,6\}   \), \( \{3\}   \). There are a total of \( 2^6 \) events.
\end{eg}
\begin{definition}
	If \( \Omega  \) is a finite set, a probability \( P \) on \( \Omega  \) is a function: \( P \colon 2^{\Omega } \to [0,1]    \) such that \( P(\varnothing) = 0 \) and \( P(\Omega )=1 \).
\end{definition}

\begin{lemma}
	If \( A_{1}, \ldots, A_{\alpha } \subset \Omega  \) are disjoint, \( P(\bigcup_{i}A_i) \) = \( \sum_{i} P(A_i) \).
\end{lemma}

\begin{prop}
	Let \( A = \{a_{1},a_{2},\ldots a_l\}   \) such that \( a_i \) are elementary events. Then, \[
		P(A) = \sum_{i=1}^{l} P(\{a_i\}  )
	.\] 
\end{prop}

\begin{eg}
	For the dice roll, if \( A=\{1,3,5\}   \), then \( P(A) = 3\cdot  \frac{1}{6}= \frac{1}{2} \).
\end{eg}

\begin{definition}
	\textbf{Equiprobable outcomes}: Let's say we have the set \( \Omega =\{\omega_1, \ldots , \omega_N\}   \) and \( P(\omega_i)=P(\omega_j) \) for all \( i \) and \( j \). Then, \( P(\omega) = \frac{1}{N} \) for all \( \omega  \in \Omega  \) and \( P(A) = \frac{|A|}{N} \). In other words, when outcomes are probable, \[ P(\text{event}) = \frac{\text{number of outcomes for that event}}{\text{number of possible outcomes}}.\]
\end{definition}

\subsection{Counting}

Suppose 2 experiments are being performed. Let's say that experiment 1 has \( m \) possible outcommes, and experiment 2 has \( n \) possible outcomes. Then together, there are total of \( n\cdot m \) total outcomes.

\begin{eg}
	Rolling a dice and then flipping a coin, how many possible outcomes are there?
\end{eg}
\begin{explanation}
	You have \( 6 \cdot 2 = 12  \) outcomes.
\end{explanation}

\begin{eg}
	Let's say you have a college planning committee that consists of 3 freshman, 4 sophomores, 5 juniors, and 2 seniors. How many ways are there to select a subcommittee of 4 with one person from each grade?
\end{eg}
\begin{explanation}
	There are 4 events with 3, 4, 5, and 2 possible outcomes for each. Therefore, there are \( 3\cdot 4\cdot 5\cdot 2=120 \) total subcommittees.
\end{explanation}

\begin{eg}
	How many 7-place license plates are there if the first 3 are letters and the last 4 are numbers?
\end{eg}
\begin{explanation}
	There are \( 26^3 \cdot 10^4 \) license plates.
\end{explanation}
