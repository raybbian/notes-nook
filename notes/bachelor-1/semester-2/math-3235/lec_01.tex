\lecture{1}{Tue 09 Jan 2024 14:01}{Intro to Probability}

\section{Basics of Probability}

What data do you need to specify probability? You need the \textbf{set of all outcomes}, a list of everything that could possibly occur as a consequence, and the likelihood of each event.


\begin{eg}
	For a roll of a dice, the set of all outcomes would be \( \{1,2,3,4,5,6\}   \). The list could include things like ``the result is 3'', or ``the result is \( \ge 4 \)'', and the likelihood would be \( \frac{1}{6} \) for each of the results.
\end{eg}

\subsection{Basics of Set Theory}

\begin{definition}
	A \textbf{set} is an unordered collection of elements. \textbf{Elements} are objects within sets.
\end{definition}

\begin{definition}
	A set \( A \) is a \textbf{subset} of a set \( B \) if \( a \in A \implies a \in B \)
\end{definition}

\begin{definition}
	The \textbf{union} of two sets \( A \) and \( B \) is the collection of elements that are in \( A \) or \( B \).
\end{definition}

\begin{definition}
	The \textbf{intersection} of two sets \( A \) and \( B \) is the collection of elements that are in both \( A \) and \( B \).
\end{definition}

\begin{definition}
	The \textbf{complement} of a set \( A \) is everything not in \( A \).
\end{definition}

\begin{definition}
	A \textbf{finite set} is a set with finite number of elements.
\end{definition}

\begin{definition}
	The \textbf{cartesian} product of two sets \( A \) and \( B \) denoted \( A \times B \) is \[
		\{(a,b) \colon a \in A \land b \in B\}  
	.\] 
	Then, \( |A \times B| = |A| \cdot |B| \).
\end{definition}

\subsection{Back to Probability}

\begin{definition}
	A \textbf{sample space} is the set of al possible outcomes in an experiment.
\end{definition}

\begin{eg}
	The sample space \( \Omega  \) for a coin flip is \( \{H,T\}   \).
\end{eg}

Note that \textbf{events} are just subsets of the sample space, and \textbf{elementary events} are just elements of the sample space.

\begin{eg}
	For a dice roll:	\( \Omega =\{1,2,3,4,5,6\}   \), some events could be \( \{1,2\}   \), \( \{3,6\}   \), \( \{3\}   \). There are a total of \( 2^6 \) events.
\end{eg}
\begin{definition}
	If \( \Omega  \) is a finite set, a probability \( P \) on \( \Omega  \) is a function: \( P \colon 2^{\Omega } \to [0,1]    \) such that \( \mathbb{P}(\varnothing) = 0 \) and \( \mathbb{P}(\Omega )=1 \).
\end{definition}

\begin{lemma}
	If \( A_{1}, \ldots, A_{\alpha } \subset \Omega  \) are disjoint, \( \mathbb{P}(\bigcup_{i}A_i) \) = \( \sum_{i} \mathbb{P}(A_i) \).
\end{lemma}

\begin{prop}
	Let \( A = \{a_{1},a_{2},\ldots a_l\}   \) such that \( a_i \) are elementary events. Then, \[
		\mathbb{P}(A) = \sum_{i=1}^{l} \mathbb{P}(\{a_i\}  )
	.\] 
\end{prop}

\begin{eg}
	For the dice roll, if \( A=\{1,3,5\}   \), then \( \mathbb{P}(A) = 3\cdot  \frac{1}{6}= \frac{1}{2} \).
\end{eg}

\begin{definition}
	\textbf{Equiprobable outcomes}: Let's say we have the set \( \Omega =\{\omega_1, \ldots , \omega_N\}   \) and \( \mathbb{P}(\omega_i)=\mathbb{P}(\omega_j) \) for all \( i \) and \( j \). Then, \( \mathbb{P}(\omega) = \frac{1}{N} \) for all \( \omega  \in \Omega  \) and \( \mathbb{P}(A) = \frac{|A|}{N} \). In other words, when outcomes are probable, \[ \mathbb{P}(\text{event}) = \frac{\text{number of outcomes for that event}}{\text{number of possible outcomes}}.\]
\end{definition}

\subsection{Counting}

Suppose 2 experiments are being performed. Let's say that experiment 1 has \( m \) possible outcommes, and experiment 2 has \( n \) possible outcomes. Then together, there are total of \( n\cdot m \) total outcomes.

\begin{eg}
	Rolling a dice and then flipping a coin, how many possible outcomes are there?
\end{eg}
\begin{explanation}
	You have \( 6 \cdot 2 = 12  \) outcomes.
\end{explanation}

\begin{eg}
	Let's say you have a college planning committee that consists of 3 freshman, 4 sophomores, 5 juniors, and 2 seniors. How many ways are there to select a subcommittee of 4 with one person from each grade?
\end{eg}
\begin{explanation}
	There are 4 events with 3, 4, 5, and 2 possible outcomes for each. Therefore, there are \( 3\cdot 4\cdot 5\cdot 2=120 \) total subcommittees.
\end{explanation}

\begin{eg}
	How many 7-place license plates are there if the first 3 are letters and the last 4 are numbers?
\end{eg}
\begin{explanation}
	There are \( 26^3 \cdot 10^4 \) license plates.
\end{explanation}

\begin{definition}
	A \textbf{permutation} is an ordering of elements in a set. The number of ways to order \( n \) elements is given by \( n! \).
\end{definition}

\begin{eg}
	Alex has a bunny ranch with 10 bunnies. They are going to run an obstacle course and ranked 1-10 based on completion time. How many possible rankings are there (no ties)?
\end{eg}
\begin{explanation}
	There are \( 10! \) possible rankings.
\end{explanation}

\begin{eg}
	Assume 6 bunnies have straight ears and 4 have floppy ears. We rank the bunnies separately. How many possible rankings are there?
\end{eg}
\begin{explanation}
	There are \( 6! \cdot 4! \) possible outcomes.
\end{explanation}

\begin{definition}
	A \textbf{combination} denotes the number of ways to choose \( k \) elements from \( n \) total elements (counting subsets).
\end{definition}

\begin{eg}
	How many ways are there to pick a 2 person team from a set of 5 people?
\end{eg}
\begin{explanation}
	There are \( C(5,2) = \binom{5}{2} = \frac{5!}{2!\cdot 3!} = 10 \) ways.
\end{explanation}

\begin{eg}
	How many committees consisiting of 2 women and 3 men can be formed from a group of 5 women and 7 men?
\end{eg}
\begin{explanation}
	We have \( C(5,2) \cdot C(7,3) \) possible committees.
\end{explanation}

\begin{eg}
	What if two of the men do not want to serve on the committee together?
\end{eg}
\begin{explanation}
	The number of ways to choose the women stays the same. However, for the men we must subtract the number of committees that have both men. Therefore, we have \( C(5,2) \cdot (C(7,3) - C(5,1)) \) possible committees.
\end{explanation}

\begin{eg}
	How many ways can we divide a 10 person class into 3 groups, sizes 3, 3, and 4?
\end{eg}
\begin{explanation}
	We just have 3 events, multiplying: \( C(10,3)\cdot C(7,3)\cdot C(4,4) \).
\end{explanation}

\begin{definition}
	This is known as a \textbf{multinomial}, and is given by \[
		\binom{n}{n_{1}, n_{2}, \ldots , n_r} = \frac{n!}{n_{1}! \cdot n_{2}! \cdot \ldots \cdot n_r!}
	.\] It counts the number of ways to partition a set of size \( n \) into sets of sizes \( n_{1},n_{2}\ldots ,n_r \).
\end{definition}

\subsection{Back to Probability Again}

\begin{eg}
	Flip 10 fair coins. What is the likelihood of flipping 3 heads?
\end{eg}
\begin{explanation}
	Number of events of 3 heads is \( C(10,3) \). Total number of events is \( 2^{10} \). Therefore, \[
		\mathbb{P}(\text{10 heads}) = \frac{C(10,3)}{2^{10} }
	.\] 
\end{explanation}

In general, we have \( \sum_{k=0}^{n} \mathbb{P}(\text{\( k \) heads}) = 1 \). In other words, \[
	\frac{1}{2^{10}  } \cdot \sum_{k=0}^{10} \binom{10}{k} = 1
.\] such that \[
	\sum_{k=0}^{10} \binom{10}{k} = 2^{10}
.\] More generally,

\begin{definition}
	The \textbf{binomial theorem} states that for all \( x,y \in \mathbb{R} \), \( n\ge 1 \), \( n \in \mathbb{N} \), \[ (x+y)^n = \sum_{k=0}^{n} \binom{n}{k} x^k y^{n-k}  .\]
\end{definition}

\begin{eg}
	Rolling 10 dice, what is the likelihood of exactly 2 outcomes each of 1,2,3,4, 1 outcome of 6, and 1 outcome of 5.
\end{eg}
\begin{explanation}
	There are total \( 6^{10}  \) outcomes, and there are \( \binom{10}{2,2,2,2,1,1} \) desired outcomes. Therefore, the probability of this event is \( \frac{\binom{10}{2,2,2,2,1,1}}{6^{10} } \).
\end{explanation}

\begin{definition}
	The \textbf{multinomial theorem} states that \( (x_{1}+\ldots +x_r)^n = \)
		\[ \sum_{n_{1}+\ldots +n_r=n} \binom{n}{n_{1},\ldots ,n_r} x_{1}^{n_{1}} x_{2}^{n_{2}} \ldots x_r^{n_r} 
	.\] 
\end{definition}

\subsection{Measure Theory}

This is just a generalization of what we have seen before.

\begin{definition}
	Let \( \mathcal{F} \subset 2^{\Omega }  \) be an ``event space''. A mapping \( P : \mathcal{F} \to  \mathbb{R} \) is a \textbf{probability measure} on \( (\Omega , \mathcal{F}) \) if 
	\begin{itemize}
		\item \( \mathbb{P}(A) \ge 0 \quad \forall A \in \mathcal{F} \)
		\item \( \mathbb{P}(\varnothing) = 0, \mathbb{P}(\Omega) = 1\)
		\item If \( A_{1},A_{2},\ldots  \) are disjoint, \[
				\mathbb{P}(\bigcup_{i = 1}^{\infty}A_i ) = \sum_{i=1}^{\infty} \mathbb{P}(A_i)
		.\] 
	\end{itemize}
\end{definition}
