\lecture{10}{Tue 13 Feb 2024 14:01}{CDFs and PDFs}

\begin{definition}
	The CDF is everything under the graph of a PDF to the left of some point \( t \).
\end{definition}

\begin{theorem}
	We have for PDFs
	\[ \mathbb{P}(a\le X\le b) = \int_{a}^{b}f_X(t) dt = F_x(b) - F_x(a)\]
	where \( F \) is a CDF.
\end{theorem}

\begin{eg}
	If \( X \) is uniformly distributed over \( (0, 10) \), calculate the probability that \( X<3 \), \( X>6 \), and \( 3<X<8 \).
\end{eg}
\begin{explanation}
	Note that \[
		f_X(t) = \begin{cases}
			0, &\text{ if }t<0\\
			\frac{1}{10} &\text{ if } 0\le t\le 10 \\
			1 &\text{ if } t > 10
		\end{cases}
	.\] Then for \( X<3 \),
	\begin{align*}
		\mathbb{P}(X < 3) &= \sum_{-\infty}^{3} f_X(t) dt \\
		&= \int_{-\infty}^{0}0 dt + \int_{0}^{3}\frac{1}{10}dt  \\
		&= \frac{3}{10}
	.\end{align*}
	Similarly for \( X>6 \), we have \( \frac{4}{10} \).
\end{explanation}

\begin{eg}
	Suppose \( X \) is a continuous random variable whose probability density function is given by \[
		f(x) = \begin{cases}
			C(4x-2x^{2} ), &\text{ if }0<x<2\\
			0 &\text{ otherwise}
		\end{cases}
	.\] 
	What is \( C \)? What is \( \mathbb{P}(X>1) \)?
\end{eg}
\begin{explanation}
	We have that \[
		\int_{-\infty}^{\infty}f(x)dx = 1 
	.\] such that 
	\begin{align*}
		1 &= \int_{0}^{2}C(4x-2x^{2} ) dx\\
		&= 4C\left(\frac{x^{2} }{2} |_0^{2} \right) - 2C \left(\frac{x^{3} }{3} |_0^{2}\right)  \\
		&= \frac{8}{3}C 
	.\end{align*}
	Therefore, \( C = \frac{3}{8} \). Now calculating \( \mathbb{P}(X>1) \), 
	\begin{align*}
		\mathbb{P}(X>1) &= \int_{1}^{\infty} f(x)dx \\
										&= \int_{1}^{2}\frac{3}{8}(4x-2x^{2} )dx  \\
										&= \frac{1}{2} 
	.\end{align*}
\end{explanation}

\begin{eg}
	The amount of time in hours that a computer functions before breaking down is a continuous random variable with PDF given by \[
		f(x) = \begin{cases}
			\lambda e^{\frac{-x}{100}} , &\text{ if } x\ge 0\\
			0 &\text{ otherwise}
		\end{cases}
	.\] What is \( \mathbb{P}(50 < X < 150) \)? \( \mathbb{P}(X<100) \)?
\end{eg}
\begin{explanation}
	We have that 
	\begin{align*}
		\int_{-\infty}^{\infty} f(x) dx &= 1 \\
		\int_{0}^{\infty}\lambda e^{\frac{-x}{100}}  &= 1 \\
		&= 0 - (100\lambda \cdot -1) \\
		&= 100\lambda 
	.\end{align*}
	such that \( \lambda =\frac{1}{100} \). Then,
	\begin{align*}
		\mathbb{P}(50 < X < 150) &= e^{-\frac{1}{2}} - e^{-\frac{3}{2}}   \\
		\mathbb{P}(X < 100) &= 1-e^{-1} 
	.\end{align*}
\end{explanation}

\begin{definition}
	The \textbf{uniform distribution} is given by \[
		f(x) = \begin{cases}
			\frac{1}{b-a}, &\text{ if }a<x<b\\
			0 &\text{ otherwise}
		\end{cases}
	.\]. The \textbf{exponential distribution} with parameter \( \lambda >0 \) has \[
		f(x) = \begin{cases}
			\lambda e^{-\lambda x} &\text{ if } x>0  \\
			0 &\text{ otherwise}
		\end{cases}
	.\] The \textbf{normal distribution} with parameters \( \mu  \) and \( \sigma ^{2}  \) has density function \[
		f(x) = \frac{1}{\sqrt{2\pi \sigma ^{2} } }\exp \left( -\frac{1}{2\sigma ^{2} } (x-\mu )^{2}  \right) 
	.\] 
\end{definition}

\begin{theorem}
	Memoryless + continuous implies \( \exp(\lambda ) \).
\end{theorem}

\begin{eg}
	Consider a post office with two clerks, each of which have a client (Ms. Jones and Mr. Brown). Given that Mr. Smith will receive service as soon as a client leaves, and each time a client spends with the clerk is exponentially distributed with paramenter \( \lambda  \), what is the probability that Mr. Smith leaves last?
\end{eg}
\begin{explanation}
	Let \( X_J, X_B \) be the amount of time \( J, B \) wait after \( S \) walks in. Let \( E_J \) be the event that \( J \) leaves first, and same with \( E_B \). Note that Mr. S leaves 2nd or 3rd. We will calculate the probability that \( S \) leaves second.

	This is just 
	\begin{align*}
		&\sum_{i = J,B} \mathbb{P}(S \text{ leaves second} \mid E_i) \cdot \mathbb{P}(E_i) 
	.\end{align*}
	Then, 
	\begin{align*}
		&\mathbb{P}(S \text{ leaves second} \mid E_i)  \\
		&= \mathbb{P}(X_B > X_S + X_J \mid  X_B > X_S) \\
		&= \mathbb{P}(X_B > X_S) \tag{Memoryless}\\
		&= \frac{1}{2} \tag{Memoryless}
	.\end{align*}
	Such that the answer is \( \frac{1}{2} \).
\end{explanation}
