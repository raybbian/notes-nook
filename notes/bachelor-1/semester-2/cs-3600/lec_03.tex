\lecture{3}{Tue 16 Jan 2024 11:05}{Search Techniques}

\section{Uninformed Search}

\begin{definition}
	\textbf{Offline} problem solving is an agent that, given a sequence, state, goal, and problem, returns an action.
\end{definition}

\begin{definition}
	A deterministic problem is a \textbf{single-state} problem.
\end{definition}

\begin{definition}
	A non-observable problem is a \textbf{conformant} problem.
\end{definition}

\begin{definition}
	A partially observable problem is a \textbf{contingency} problem.
\end{definition}

\begin{definition}
	An unknown state space is an \textbf{exploration} problem.
\end{definition}

How do we select a state space? The world is very, very complex such that the state space must be abstracted for problem solving. These abstract solutions should be easier than the real problem.

\subsection{State Space Graphs and Search Trees}

\begin{definition}
	A \textbf{state space graph} is a mathematical representation of a search problem. Arcs on this graph represent an action that causes a change from one state to another.
\end{definition}

Note that we can rarely build this graph in memory in full. Also, a state tree for a certain graph could be infinitely large (with a cyclic state space graph).

The difference between states and nodes is that the node includes the state, but also information about its children, parents, etc.

\begin{definition}
	\textbf{Completeness}: does it always find a solution?
	\textbf{Complexity}: number of nodes expanded, memory used?
	\textbf{Optimality}: does it always find a least-cost solution?
\end{definition}

Time and space complexity are measured in \( b \), \( d,  \) and \( m \), where \( b \) is the maximum branching factor, \( d \) is the depth of the least-cost solution, and \( m \) is the maximum depth of the state space.

\begin{eg}
	What is the completeness, complexity, and optimality of DFS?
\end{eg}
\begin{explanation}
	DFS fails in infinite depth spaces and spaces with loops, so it is not complete. It has \( b^{m}  \) time complexity and \( bm \) space complexity.
\end{explanation}

\begin{eg}
	What about BFS?
\end{eg}
\begin{explanation}
	Yes, time complexity of \( b^{d+1}  \), in \( b^{d+1}  \) space, etc.
\end{explanation}

Sometimes, the shortest path is not the fastest path, so a uniform cost search could be used.
