\lecture{6}{Tue 30 Jan 2024 12:32}{Graphs, DFS, Topological Order}

\section{Graphs}

Insert every class' graph definitions here.

Two common representations of graphs are adjacency matrices, adjacency lists, and edge lists.

\begin{definition}
	An \textbf{adjacency matrix} is a matrix where \[
		A_{ij} = \begin{cases}
			1 &\text{ if there is an edge between \( i,j \)}\\
			0 &\text{ otherwise}
		\end{cases}
	.\] 
\end{definition}

However, an adjacency list uses too much space, so an adjacency list can be used, which is at most \( O(V + E) \) space. This list stores only direct neighbors.

\subsection{DFS}

Yeah, it's just DFS. At a current node, go to the next node. After you've finished exploring everything from that node, come back to this node and go to next node after. Keep a visited array to not visit vertices more than once.

\begin{lemma}
	DFS(\( v \)) will visit every vertex that can be reached from \( v \) via some path.
\end{lemma}
\begin{proof}
	By contradiction. Assume that there is a path \( v, v_{1}, v_{2}, \ldots , v_k \). Assume \( v_k \) is not visited by the call to DFS(\( v \)). Let \( v_i \) be the last vertex that did get visited. That means DFS(\( v_i \)) must have been called. However, that means the edge between \( v_{i} \) and \( v_{i+1} \) must have been looked at. The only way then that DFS(\( v_{i+1} \)) doesn't get called, is if it is already visited. However, that means that DFS(\( v_{i+1} \)) must have been called. This is a contradiction, as then \( v_{i+1} \) must be the last visited vertex.
\end{proof}

\begin{definition}
	A graph \( G \) is a \textbf{directed acyclic graph} if it is a directed graph without cycles.
\end{definition}

You can think of directed acyclic graphs as dependency graphs. What happens if we have a cycle? Bad things. How can we find these cycles? We can run DFS, and if we explore a vertex that is already in the current path, then we know there is a cycle. We can use backtracking to extract this cycle, if necessary.

\subsection{Topological Sort}
We can order the vertices of our graph topologically, so that all dependencies are processed before the nodes that require them. This is just a dfs where we only recurse once all vertex indegrees have been processed already.
