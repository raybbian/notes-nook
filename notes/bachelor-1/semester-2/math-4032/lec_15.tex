\lecture{15}{Mon 12 Feb 2024 14:06}{}

\begin{definition}
	A graph \( G \) is \textbf{bipartite} if \( G \subseteq T(n, 2) \) for some \( n \). Equivalently, \( G \) is \textbf{bipartite} if its vertex set can be paritioned into two sets \( A \) and \( B \) such that every edge of \( G \) has one end in \( A \) and one end in \( B \). The two sets \( A \) and \( B \) are called a \textbf{bipartition} of the graph.
\end{definition}

\begin{theorem}
	(Kovori-Sos-Turan) Let \( G'=(V',E') \) be a bipartite graph. Then there exists \( \epsilon >0 \) (which depends on \( G' \)) such that every \( n \)-vertex graph with no subgraph isomorphic \( \cong G' \) has \( O(n^{2-\epsilon } ) \) edges. In fact, we can take \[
		\epsilon = \frac{1}{\text{size of larger part of bipartition}}
	.\] 
\end{theorem}

\begin{theorem}
	(Special case of KST) If \( G' \) is the complete bipartite graph with bipartition \( (A,B) \) such that \( |A|=2 \) and \( |B|=2 \), then every \( n \)-vertex graph with no subgraph \( \cong G' \) has \[
		O(n^{\frac{3}{2}} )
	.\] Note that \( G' \) is also the 4-cycle \( C_4 \).
\end{theorem}

\begin{note}
	The same for \( K_{3,3} \) is an open problem, known as the \textbf{Zarankeiwicz Problem}.
\end{note}

We will now look at a construction for the lower bound.

\begin{definition}
	In its essence, a \textbf{projective plane} is a system of \( q^{2} + q + 1  \) lines and ponts such that every pair of points determines a line, every pair of lines determines a unique point, and every line contains \( q+1 \) points.
\end{definition}

\begin{tmpexplanation}
	Consider the (biparite) incidence graph of a projective plane such that each point is connected to the lines that it is in. Formally, \[
		V = \{\text{points}\} \cup \{\text{lines}\} 
	\] and \[ 
		\quad E = \{\{x,y\} : x \text{ is a point in line } y  \}	
	.\] 
	Observe that \( G \) has no subgraph isomorphic to \( C_4 \). Also note that the number of edges is the number of lines times the number of points in each line, which is \[
		(q^{2}+q+1)(q+1) \ge q^{3}  
	.\] Also note that the number of vertices is \( 2(q^{2}+q+1) =O(q^{2})  \). Therefore, \[
		\text{number edges} = \Omega(\text{number vertices}^{1.5} )
	.\] 
\end{tmpexplanation}
