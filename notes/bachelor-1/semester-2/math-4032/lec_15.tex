\lecture{15}{Mon 12 Feb 2024 14:06}{}

\begin{observation}
	Since \( T(n, 2) \) has \( \approx \frac{n^{2}}{4}  \) edges, \( G' \) must have a of \( T(n, 2) \). This way, we can block many more edges.
\end{observation}

\begin{definition}
	A graph \( G \) is \textbf{bipartite} if \( G \subseteq T(n, 2) \) for some \( n \). Equivalently, \( G \) is \textbf{bipartite} if its vertex set can be paritioned into two sets \( A \) and \( B \) such that every edge of \( G \) has one end in \( A \) and one end in \( B \).
\end{definition}
