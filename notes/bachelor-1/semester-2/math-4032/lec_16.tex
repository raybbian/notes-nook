\lecture{16}{Wed 14 Feb 2024 14:02}{}

\begin{theorem}
	(Cauchy-Schwarz) Given \( x_{1}, x_{2}, \ldots  ,x_n \in \mathbb{R}  \) and \( y_{1}, y_{2}, \ldots , y_n \in \mathbb{R} \), \[
		\sum_{i=1}^{n}x_iy_i \le \sqrt{\sum_{i=1}^{n} x_{i}^{2} } \cdot \sqrt{\sum_{y=1}^{n} y_{i}^{2} } 
	.\] 
\end{theorem}

We will now prove the special case of the Kovori-Sos-Turan theorem.

\begin{replacementproof}
	We count the number of cherries such that \( v \) is red and the two green star vertices are \( \in N_G(v) \). One way to count such cherries is choose every two vertices in the neighborhood of \( G \), which is \( \binom{d_G(V)}{2} \).
\end{replacementproof}
