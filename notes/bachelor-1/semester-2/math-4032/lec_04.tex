\lecture{4}{Wed 17 Jan 2024 14:02}{Estimates for \( n! \)}

\begin{lemma}
	(Simplest) For any positive integer \( n \in \mathbb{Z}_{>0} \), \[
		2^{n-1} \le n! \le n^{n-1}  
	.\] 
\end{lemma}
\begin{proof}
	We have for the lower bound
	\[
		n! = \prod_{i=2}^{n}i \ge \prod_{i=2}^{n}2 = 2^{n-1}  
	.\] 
	And for the upper bound,
	\[
		n! = \prod_{i=2}^{n}i \le \prod_{i=2}^{n}n = n^{n-1}  
	.\] 
\end{proof}

Note that these bounds are very far off. Here is a motivating example.

\begin{eg}
	Suppose \( n \) students draw a card from a deck of \( n \) cards, replacing the card afterwards. What is the likelihood that all \( n \) cards drawn are distinct?
\end{eg}
\begin{explanation}
	The probability is the number of desireable outcomes over the total number of outcomes. This is just \[
		\frac{n!}{n^{n} }
	.\] 
	Note that if we use the upper bound from this lemma, we would get that the probability is at most \( \frac{1}{n} \). In reality however, the true probability is much, much smaller.
\end{explanation}

\begin{lemma}
	A better set of bounds are the following:
	\[
		\left( \frac{n}{2} \right) ^{\frac{n}{2}} \le n! \le \frac{(n+1)^{n} }{2^{\frac{n}{2}} }
	.\] 
\end{lemma}
\begin{proof}
	Left as an exercise!
\end{proof}

\begin{lemma}
	For any two \( a,b \ge 2 \), we have \( a\cdot b\ge a+b \).
\end{lemma}
\begin{lemma}
	(Arithmetic-Geometric Mean Inequality) For any two \( a,b\ge 0 \), we have \[
		\sqrt{ab} \le \frac{a+b}{2}
	.\] 
\end{lemma}

With these last two lemmas, we can show the following:

\begin{theorem}
	(Gauss) For any \( n \in \mathbb{Z}_{>0} \), \[
		n^{\frac{n}{2}} \le n! \le \frac{(n+1)^{n} }{2^{n} }
	.\] 
\end{theorem}
\begin{proof}
	We instead look at \( (n!)^{2}  \). Pairing 1 with \( n \), 2 with \( n-1 \), etc, we have for the lower bound
	\begin{align*}
		n! &= \left( \prod_{i=1}^{n}i  \right) \left( \prod_{i=1}^{n}i  \right) \\
		&= \prod_{i=1}^{n}i(n + 1 - i)  \\
		&= \prod_{i=1}^{n}\sqrt{i(n+1-i)}   \\
		&\ge \prod_{i=1}^{n}\sqrt{n} \tag{Lemma 7}  \\
		&\ge n^{\frac{n}{2}} 
	.\end{align*}
	And for the upper bound, we have 
	\begin{align*}
		n! &= \left( \prod_{i=1}^{n}i  \right) \left( \prod_{i=1}^{n}i  \right) \\
		&= \prod_{i=1}^{n}i(n + 1 - i)  \\
		&= \prod_{i=1}^{n}\sqrt{i(n+1-i)}   \\
		&\le \prod_{i=1}^{n} \frac{i + n + 1 - i}{2}  \\
		&=  \frac{(n+1)^{n} }{2^{n} }
	.\end{align*}
\end{proof}

\begin{theorem}
	(Even better bound) For any \( n \in \mathbb{Z}_{>0} \), we have \[
		e \left( \frac{n}{e} \right) ^{n}  \le n! \le en \left( \frac{n}{e} \right) ^{n} 
	.\] 
\end{theorem}
\begin{proof}
	The lower bound will be given as a homework problem. The upper bound is as follows. Note that \( \ln (n!) = \sum_{i=1}^{n} \ln (i) \). Then, we can take the integral of \( \ln (x) \), which is greater than this sum.
	\begin{align*}
		\sum_{i=1}^{n} \ln (i) &\le \int_{1}^{n+1}\ln (x) dx  \\
		&= (n+1)\ln (n+1)-n
	.\end{align*}
	Thus 
	\begin{align*}
		n! &\le e^{(n+1)\ln (n+1)-n}  \\
		&= \frac{e^{(n+1)\ln (n+1)}}{e^{n} }  \\
		&= \frac{\left(e^{\ln (n+1)}\right)^{n+1}    }{e^{n} } \\
		&= \frac{\left( n+1 \right) ^{n+1}}{e^{n} } 
	.\end{align*}
	Applying this for \( n(n-1)! \) gives the bound.
\end{proof}
