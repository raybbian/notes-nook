\lecture{2}{Wed 10 Jan 2024 13:56}{Review of Proofs}

We will now review the types of proofs covered in Math-3012, as well as guidelines for writing them in this class.

\begin{notation}
	If \( F \) is a mapping from \( N \) to \( M \), we write \( F: N \to M \).
\end{notation}

\begin{notation}
	Sometimes, \( N \setminus \{a\}   \) will be instead written as \( N-\{a\}   \).
\end{notation}

\begin{prop}
	Let \( N \) be an \( n \)-element set and \( M \) be an \( m \)-element set. Then, there are \( m^n \) mappings (or functions) from \( N \) to \( M \).
\end{prop}
\begin{proof}
	(Inductive) We go by induction on \( n \). 
	\begin{description}
		\item[Base case.] For the base case \( n=0 \), we consider the empty set \( \varnothing \) to be a mapping from the empty set to \( M \). So \( m^0=1 \) and the base case holds. 
		\item[Inductive step.] Now, let \( n\ge 1 \) and assume that the proposition holds for \( n-1 \) by induction. So, let \( a \in N \). There are \( m^{n-1}  \) mappings \( F' : N \setminus \{a\} \to M  \). For each such \( F' \), we have \( m \) choices for where to send \( a \). These mappings are all distinct, and every \( F:N \to M \) can be obtained in this way. So, the number of mappings \( F : N \to M \) is \( m^{n-1}\cdot m  = m^n\), as desired.
	\end{description}
\end{proof}

\begin{definition}
	A \textbf{bijection} is a function \( f : X \to Y \) such that \( f \) is one-to-one and onto.
\end{definition}

\begin{corollary}
	An \( n \)-element set \( X \) has \( 2^n \) many subsets.
\end{corollary}
\begin{proof}
	(Bijective) For each \( A \subseteq X \), let \( F_A : X \to \{0,1\}   \) such that for each \( x \in X \), 
	\[ F_A(x)=
		\begin{cases}
			0 &\text{ if }x \not\in A\\
			1 &\text{ if }x \in A
		\end{cases} 
	.\] These mappings \( F_A,F_{A'} \) are distinct for distinct subsets \( A,A' \subseteq X \), and every mapping \( F:X \to \{0,1\}   \) is equal to \( F_A \)  for some \( A \subseteq X \). So by proposition 1, the corollary holds.
\end{proof}

\begin{lemma}
	For any non-negative integers \( n,k \) (\( n,k \in Z_{\ge 0} \)), we have \( \binom{n}{k} = \binom{n}{n-k} \).
\end{lemma}
\begin{proof}
	(Algebraic) We have 
	\begin{align*}
		\binom{n}{k} &= \frac{n!}{k!(n-k)!} \\
								 &= \frac{n!}{(n-(n-k))!(n-k)!} \\
								 &= \binom{n}{n-k}
	,\end{align*} as desired.
\end{proof}

\begin{theorem}
	(Binomial Theorem) Let \( n \in \mathbb{Z}_{\ge 0}\). Then \[
		(x+y)^n = \sum_{k=0}^{n} \binom{n}{k} x^k y^{n-k}
	.\] 
\end{theorem}
\begin{proof}
	Consider \[ 
		\underbrace{(x+y)(x+y)\ldots (x+y)}_{n \text{ times } }
	.\] For each \( (x+y) \) term, we select either the \( x \) or the \( y \), and there are \( \binom{n}{k} \) ways to select \( k \) \( x \)'s and \( n-k \) \( y \)'s. The formula follows.
\end{proof}

\begin{corollary}
	For any \( n \in \mathbb{Z}_{\ge 0} \), we have \[ 
		2^n=\sum_{k=0}^{n} \binom{n}{k} \text{ and } 0=\sum_{k=0}^{n} \binom{n}{k} (-1)^k 
	.\]
\end{corollary}
\begin{proof}
	Apply the binomial theorem with \( x=y=1 \) to yield the first result, and with \( x=-1, y=1 \) to yield the second.
\end{proof}

\subsection{Counting Review}

\begin{definition}
	A \textbf{permutation} is a bijection from a finite set to itself.
\end{definition}

\begin{eg}
	One such bijection could be \( 1 \mapsto 2, 2 \mapsto 1, 3 \mapsto 4, 4 \mapsto 5, 5 \mapsto 3 \).
\end{eg}

\begin{lemma}
	The number of such bijections is \( n! \).
\end{lemma}
\begin{proof}
	Exercise to the student!
\end{proof}
