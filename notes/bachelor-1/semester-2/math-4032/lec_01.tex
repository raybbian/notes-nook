\lecture{1}{Mon 08 Jan 2024 12:36}{Syllabus and Review}

\section{Introduction}

This course is basically just a second course in Combinatorics, and will cover a range of topics.

\begin{definition}
	\textbf{Matroids} are the structures that capture whether or not the greedy algorithm works. They will be covered later in the course.
\end{definition}

Now, for some examples and review:

\begin{definition}
	We say points are in \textbf{convex position} if no point is inside a triangle made by 3 other points.
\end{definition}

\begin{eg}
	Given a finite set of points on the plane, what is the maxmimum number of points such that no 3 are on a line, and no 4 are in convex position.
\end{eg}
\begin{explanation}
	Informally, we know that the ``outside'' of our points has at most 3 points in the shape of a triangle. We can then place a point in the middle. However, if we try to add another point, then we find that 4 points are in convex position, which is a contradiction. Therefore, 4 points is the maximum size of such a set.
\end{explanation}

This example is actually part of a more general problem, shown below.

\begin{theorem}
	(ES, 1935) The maxmimum number of points such that no 3 are on a line and no \( n \) are in convex position is \( \le 4^n \) and \( \ge 2^{n-2}  \).
\end{theorem}

\begin{theorem}
	(Suk, 2017) This number is actually \( \le 2^{n + o(1)}  \)
\end{theorem}

\begin{notation}
	Think of \( o(1) \) as standing for a function \( f(n) \) such that \( \lim_{n \to \infty} f(n) = 0 \). In other words, for every \( \epsilon >0 \), there exists \( n_0 \) such that \( |f(n)| < \epsilon  \) for every \( n\ge n_0 \).
\end{notation}

\begin{eg}
	How many distinct 5-letter words are there on the 26-letter english alphabet?
\end{eg}
\begin{explanation}
	There are 26 options for each of the 5 slots, so there are \( 26^5 \) words.
\end{explanation}

\begin{eg}
	What if repetitions aren't allowed?
\end{eg}
\begin{explanation}
	Each slot you lose an option, so there are \( 26 \cdot 25 \cdot 24 \cdot 23 \cdot 22 = \frac{26!}{21!}\) words.
\end{explanation}

\begin{eg}
	How many ways are there to choose 5 students out of 35 to present?
\end{eg}
\begin{explanation}
	There are \( \binom{35}{5} = \frac{35!}{5! \cdot 30!} \) ways.
\end{explanation}
