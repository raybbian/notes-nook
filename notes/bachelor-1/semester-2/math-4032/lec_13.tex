\lecture{13}{Wed 07 Feb 2024 14:01}{Extremal Graph Theory}

\begin{definition}
	If we have that \( G \) and \( G' \) are \textbf{isomorphic}, we write that \[
		G \cong G'
	.\] 
\end{definition}

\begin{definition}
	If \( e \in E \), then the \textbf{ends} of \( e \) are the two vertices \( v \in e \).
\end{definition}

\begin{definition}
	We say that a graph \( G' \) is a \textbf{subgraph} of \( G \) if \( V' \subseteq V \) and \( E' \subseteq E \).
\end{definition}

If we fix a graph \( G' \), what is the maximum number of edges in an \( n \)-vertex graph such that \( G \) does not contain a subgraph \( \cong G' \)?

\begin{eg}
	If \( G' \) is the single edge, then the maximum number of edges in a graph that does not contain \( G' \) is 0.
\end{eg}

\begin{eg}
	If \( G' \) is the line graph with 2 edges, then for every vertex \( v \) we have that \( d_G(v) \le 1 \). By the handshaking lemma, we have that \[
		2|E| \le \sum_{v \in V}d_G(v) \le |V|
	.\] Therefore, \[
		|E| \le \left\lfloor \frac{n}{2} \right\rfloor
	.\] 
\end{eg}

\begin{observation}
	Note that if \( G'' \subseteq G' \), and \( G'\subseteq G \), then \( G'' \subseteq G \).
\end{observation}

\begin{definition}
	For any integer \( r\ge 1 \), the \textbf{complete graph} on \( r \) vertices, denoted \( K_r \), has \( r \) vertices and \( \binom{r}{2} \) edges.
\end{definition}

We now look at the case where \( G'=K_3 \) is the triangle graph. One such example is the complete bipartite graph with \( \left\lfloor \frac{n}{2} \right\rfloor \) vertices on one side and \( \left\lceil \frac{n}{2} \right\rceil  \) vertices on the other. this graph has \( \left\lfloor \frac{n}{2} \right\rfloor \left\lceil \frac{n}{2} \right\rceil \) edges, which is actually the unique graph to the extremal problem.

\begin{theorem}
	(Turan) Every \( n \)-vertex graph with no subgraph isomorphic to \( K_3 \) has at most \[
		\left\lfloor \frac{n}{2} \right\rfloor \left\lceil \frac{n}{2} \right\rceil 
	.\] edges.
\end{theorem}
\begin{proof}
	(Erdos) Let \( G = (V,E) \) be an \( n \)-vertex graph with no subgraph \( \cong K_3 \). Let \( v \in V \) be a vertex of maximum degree. Let \( N_G(v) = \{x \in V : (v, x) \in E\}   \). Let this set be called the \textbf{neighborhood} of \( v \) in \( G \). 

	Let \( \hat{G} = (\hat{V}, \hat{E})\) be the graph with vertex set \( \hat{V}=V \).
\end{proof}
