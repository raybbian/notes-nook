\lecture{3}{Fri 12 Jan 2024 14:01}{Permutations and Cycles}

\begin{notation}
	\( \tau : X \to X \) is a permutation \textbf{on} \( X \). Can also be denoted by \( \sigma : X \to X \).
\end{notation}

We will show that all permutations \( \tau  \) can be ``decomposed'' into ``cycles''.

\begin{eg}
	From the example earlier, \( (1,2) \) is a cycle, and \( (3,4,5) \) is another cycle.
\end{eg}

For the following, let \( \tau : X \to X \).

\begin{definition}
	A \textbf{cycle} of \( \tau  \) is a tuple (ordered set of elements) \( (x_{1},x_{2},\ldots ,x_k) \) such that \( x_{1},x_{2},\ldots ,x_k \) are distinct elements of \( X \), and \( \tau (x_{1})=x_{2}, \tau (x_{2})=x_{3},\ldots,\tau (x_{k-1})=x_k, \tau (x_k)=x_{1}  \). We call \( x_{1},x_{2},\ldots x_k \) the \textbf{elements} of the cycle.
\end{definition}

\begin{lemma}
	If \( (x_{1},x_{2}\ldots ,x_k) \) and \( (y_{1},y_{2},\ldots y_r) \) have an element in common, then \( \{x_{1},x_{2},\ldots x_k\} =\{y_{1},y_{2},\ldots y_r\}    \).
\end{lemma}
\begin{proof}
	Note that since \( (x_{1},x_{2},\ldots ,x_k) \) is a cycle, \( (x_{2},x_{3},\ldots ,x_k,x_1) \) is also a cycle. Because of this, we can assume that \( x_{1}=y_{1} \). So \( x_{2}=\tau (x_{1})=\tau (y_{1})=y_{2} \).  Then, we have that \( x_{2}=y_{2} \). We can repeat this process until \( x_k=y_k \) (swap \( x,y \) if \( k > r \)). Then, we have \( x_{1}=\tau (x_k)=\tau (y_k)=y_{1} \), which means that \( r=k \). Therefore, all cycles are pairwise disjoint.
\end{proof}

\begin{lemma}
	For every \( x \in X \), there exists a cycle of \( \tau  \) which has \( x \) as an element.
\end{lemma}
\begin{proof}
	Consider visiting each element \( x,\tau (x), \tau (\tau (x)), \ldots  \), until the first time we re-visit any element. This will eventually happen, because \( X \) is finite. Then, let's suppoose that we have visited elements \( x_{1},x_{2},\ldots x_k \) so far, such that \( x_{1},x_{2},\ldots ,x_k \) are distinct, and that \( \tau (x_k)=x_i \) for some \( i \in \{1,2,\ldots ,k\}   \). We cannot have \( i\ge 2 \) because then both \( x_{i-1} \) and \( x_k \) would both map to \( x_i \), which is a contradiction because a permutation is a bijection. Therefore, \( i=1 \) and we have established our cycle.
\end{proof}

\begin{corollary}
	There exists cycles \( C_{1},C_{2},\ldots ,C_t \), so that every element of \( X \) is an element in exactly one such cycle.
\end{corollary}

\begin{definition}
	The \textbf{cycle notation} for \( \tau  \) is written as \[
		\tau =C_{1}C_{2}\ldots C_t
	.\] 
\end{definition}

\begin{eg}
	Find the cycle notation for the permutation \( \tau  \) of \( \{1,2,3,4,5,6\}   \) where 
	\begin{align*}
		\tau (1)&=4 \\
		\tau (2)&=6 \\ 
		\tau (3)&= 2 \\
		\tau (4)&= 5 \\
		\tau (5)&= 1 \\
		\tau (6)&= 3
	.\end{align*}
\end{eg}
\begin{explanation}
	By inspection, we have a cycle \( (1,4,5) \) and another cycle \( (2,3,6) \). Therefore, \( \tau = (1,4,5)(2,3,6) \).
\end{explanation}

\begin{definition}
	A \textbf{transposition} is a cycle with exactly two elements.
\end{definition}

\begin{problem}
	How quickly does \( n! \) grow as \( n \) gets large?
\end{problem}
