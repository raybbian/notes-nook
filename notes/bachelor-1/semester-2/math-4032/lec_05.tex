\lecture{5}{Fri 19 Jan 2024 14:01}{Asymptotic Analysis}

\begin{theorem}
	\textbf{Stirling's Formula} says that \[
		n! \sim \sqrt{2\pi n} \left( \frac{n}{e} \right) ^{n} 
	.\] 
\end{theorem}

\begin{definition}
	For two functions \( f,g:\mathbb{Z}_{>0} \to \mathbb{R} \), we write \( f \sim g \) and say \( f \) is asymptotic to \( g \) if \[
		\lim_{n \to \infty} \frac{f(n)}{g(n)} = 1
	.\] Also note that \( f\sim g \iff g \sim f \).
\end{definition}

\begin{eg}
	\( 2n + \sqrt{n} \sim 2n \).
\end{eg}
\begin{explanation}
	This is because \[
		\lim_{n \to \infty} \frac{2n+\sqrt{n} }{2n} = 1
	.\] 
\end{explanation}

\begin{eg}
	(Informal) How many digits are in \( 100! \)?
\end{eg}
\begin{explanation}
	Using Stirling's Formula, we have that \[
		100! \sim \sqrt{2\pi 100} \left( \frac{100}{e} \right) ^{100} = 9.324\ldots \times 10^{157}  
	.\], whereas \( 100! = 9.332\ldots  \times 10^{157}  \) (very close approximation).
\end{explanation}

\begin{definition}
	The \( n \)-th harmonic number \[
		H_n=\sum_{i=1}^{n} \frac{1}{i}
	.\] 
\end{definition}

\begin{theorem}
	(Euler-Mascheroni) \( H_n \sim \ln (n) \).
\end{theorem}
\begin{proof}
	Omitted.
\end{proof}

\begin{lemma}
	For any positive integer \( n \in \mathbb{Z}_{>0} \), we have \[
		\frac{\left\lfloor \log _2(n) \right\rfloor}{2} \le H_n \le \left\lfloor \log _2(n) \right\rfloor + 1
	.\] 
\end{lemma}
\begin{proof}
	We can break up the proof into parts of size \( 2,4,8,16\ldots  \). Let \( S_k=\{i \in \mathbb{Z}_{>0} : 2^{k-1} \le i \le 2^{k}-1   \}   \) for any \( k \in \mathbb{Z}_{>0} \). Note that \( |S_k| = 2^{k-1}  \). Also, for every \( x \in  S_k \), we have \[
		\frac{1}{2^{k} } < \frac{1}{x} \le \frac{1}{2^{k-1} }
	.\] 
	Therefore, we have 
	\begin{align*}
		H_n &= \sum_{i=1}^{n} \frac{1}{i} = \sum_{k=1}^{\left\lfloor \log _2(n) \right\rfloor} \sum_{x \in S_k} \frac{1}{x} \\
				&\ge \sum_{k=1}^{\left\lfloor \log _2(n) \right\rfloor} \sum_{x \in S_k} \frac{1}{2^{k} } \\
		&= \sum_{k=1}^{\left\lfloor \log _2(n) \right\rfloor} 2^{k-1}/2^{k}   \\
		&= \sum_{k=1}^{\left\lfloor \log _2n \right\rfloor} \frac{1}{2} \\
		&= \frac{\left\lfloor \log _2(n) \right\rfloor}{2}
	.\end{align*}
	In the other direction, we have
	\begin{align*}
		H_n &\le \sum_{k=1}^{\left\lfloor \log _2(n) \right\rfloor+1} \sum_{x \in S_k} \frac{1}{x} \\
		&\le \sum_{k=1}^{\left\lfloor \log _2(n) \right\rfloor + 1} \frac{|S_k|}{2^{k-1} }  \\
		&= \left\lfloor \log _2(n) \right\rfloor + 1
	.\end{align*}
\end{proof}

\begin{definition}
	Let \( f,g : \mathbb{Z}_{>0}\to \mathbb{R} \). We say \( f=O(g) \) or \( f \) is big-O of \( g \) if there exists \( n_0,C \), such that \[
		|f(n)| \le C\cdot g(n) \quad \forall n\ge n_{0}
	.\] 
\end{definition}

\begin{note}
	If \( f,g :\mathbb{Z}_{>0}\to \mathbb{R} \) and \( f\sim g \), we have \[
		f=O(g) \quad \text{and} \quad g=O(f)
	.\] If \( \epsilon =1 \) for all significantly large \( n \), \( \frac{f(n)}{g(n)}\le 2 \).
\end{note}

\begin{eg}
	\( \sum_{i=1}^{n} \frac{1}{i} = O(\log n) \).
\end{eg}

\begin{lemma}
	Let \( a, \alpha ,\beta >0 \) be fixed. Then as \( n\to \infty \), 
	\begin{itemize}
		\item \( n^{\alpha } = O(n^{\beta } )  \) if \( \alpha <\beta  \).
		\item \( n^{\alpha } = O(a^{n} ) \) if \( a>1 \).
		\item \( (\ln (n))^{\alpha } =O(n^{\beta } ) \) .
	\end{itemize}
\end{lemma}
