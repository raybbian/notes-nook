\lecture{6}{Mon 22 Jan 2024 14:02}{Binomial Coefficients and Counting Primes}

\begin{note}
	Note that we can also write, for functions \( f, g, h \) \( f = g + O(h) \), which means that \( |f - g| + O(h) \).
\end{note}

\begin{eg}
	\[ \binom{n}{2} = \frac{n(n-1)}{2} = \frac{n^{2}}{2} - \frac{n}{2} = \frac{n^{2}}{2} + O(n) .\]
\end{eg}

\begin{definition}
	\( f(n) = \Theta (g(n)) \) if \( f=O(g) \) and \( g=O(f) \).
\end{definition}

\begin{definition}
	\( f = o(g) \) if \( \lim_{n \to \infty} \frac{f}{g} = 0 \).
\end{definition}

\begin{eg}
	What are all primes less than 20?
\end{eg}
\begin{explanation}
	2, 3, 5, 7, 11, 13, 17, 19
\end{explanation}

\begin{definition}
	Let \( \pi (n) \) be the number of primes that are \( \le n \).
\end{definition}

\begin{theorem}
	The \textbf{prime number theorem} states that \[
		\pi (n) \sim \frac{n}{\ln (n)}
	.\] 
\end{theorem}

The \textbf{Riemann Hypothesis} states that \[
	\pi (n) = \int_{1}^{n}\frac{1}{\ln (x)} dx + O(\sqrt{n} \ln (n)) 
.\] It's called a hypothesis because it is often used in other mathematical proofs, even if not proved yet. For example, determining whether a knot could be un-knotted is in NP if the RH is true.

\begin{lemma}
	For any \( k\ge 1 \), we have that \[
		\binom{2k+1}{k} \le 4^{k} 
	.\] 
\end{lemma}
\begin{proof}
	Later.
\end{proof}

\begin{lemma}
	For any \( n\ge 2 \), the product of all primes \( \le n \) is at most \( 16^{n}  \).
\end{lemma}
\begin{proof}
	We wish to prove that \[
		\prod_{i=1}^{\pi (n)}p_i \le 16^{n}  
	.\] where \( p_i \) denotes the \( i \)-th prime. We proceed with induction on \( n \).
	\begin{description}
		\item[Base case] \( n=2,3 \). Holds trivially.
		\item[Step case 1] \( n \) is even. Note that \( n \) cannot be prime, such that by induction, \[
			\prod_{i=1}^{\pi (n)} p_i =  \prod_{i=1}^{\pi (n-1)} p_i \le  16^{n-1} \le 16^{n} 
		.\] 
		\item[Step case 2] \( n \) is odd. We write \( n=2k+1 \) for some \( k \ge 1 \). Note that every prime \( p \) such that \( k+2 \le p \le 2k + 1 \) divides \( \binom{2k+1}{k} \). This is because \[
				\binom{2k+1}{k} = \frac{(2k+1)!}{k!(k+1)!}
		.\] such that \( p \) divides the numberator but not the denominator. 

		By induction the product of primes \( p \) such that \( 0 \le p \le k+1 \) is bound by \[
			\prod_{i=1}^{\pi (k+1)}p_i \le 16^{k+1}  
		.\] Combining our bounds and lemma, we have
		\begin{align*}
			\prod_{i=1}^{\pi (n)} p_i &= \left(\prod_{i=1}^{\pi (k+1)} p_i \right)\left( \prod_{i=\pi (k+1)+1}^{\pi (n)}p_i  \right) \\
													&\le 16^{k+1} \binom{2k+1}{k} \\
													&\le 16^{k+1} \cdot 4^{k}  \\
													&\le 16^{2k+1} 
		.\end{align*}
		Note that bounding by \( 4^{n}  \) may work here, but \( 16^{n}  \) is presented due to a mistake in the lecture notes.
	\end{description}
\end{proof}

\begin{theorem}
	The weak prime number theorem states that \[ \pi (n) = \Theta \left(\frac{n}{\ln n}\right) .\]
\end{theorem}
\begin{proof}
	We will show the upper bound, i.e. \[
		\pi (n) = O\left(\frac{n}{\ln n}\right)
	.\] 
	Let \( p_{1},p_{2},\ldots \) be the sequence of primes. Then, \[
		\pi (n)! \le \prod_{i=1}^{\pi (n)}p_i  \le 16^{n} 
	.\] because \( p_{1} \ge 1, p_{2} \ge 2 \), etc. We have also shown that \[
		e\left(\frac{\pi (n)}{e}\right)^{\pi (n)} \le \pi (n)!
	.\] As such, \[
		e\left(\frac{\pi (n)}{e}\right)^{\pi (n)} \le 16^{n} 
	.\] Taking the \( \ln  \) of both sides, we have
	\begin{align*}
		\ln \left( \correct{e}{1} \left( \frac{\pi (n)}{e} \right)^{\pi (n)}  \right) &\le \ln \left( 16^{n}  \right) \\
		\pi (n)\cdot \ln \left( \frac{\pi (n)}{e} \right) &\le n\ln (16) \\
	.\end{align*}

	Assume towards a contradiction that \[ \pi (n) \ge \frac{100n}{\ln n} \]. Then, 
	\begin{align*}
		\frac{100n}{\ln n} \cdot \ln \left( \frac{100n}{e\ln n} \right) &\le n\ln 16 \\
		\frac{100}{\ln 16 \cdot \ln n}\ln \left( \frac{100n}{e\ln n} \right) &\le 1 \\
		\frac{100}{\ln 16\cdot \ln n} (\ln (100n) - \ln (e\ln n)) &\le 1 \\
	.\end{align*}
	which is a contradiction (after many calculations). Therefore, there exists some \( C<100 \) such that \( \pi (n) = O(\frac{n}{\ln n}) \).
\end{proof}
