\lecture{12}{Mon 05 Feb 2024 13:58}{}

\begin{note}
	All normal graphs are Sperner, as all edges have the same size.
\end{note}

\begin{lemma}
	(LMY-Inequality) If \( H=(V,E) \) is a Sperner hypergraph, and \( n=|V| \), then \[
		\sum_{A \in E} \frac{1}{\binom{n}{|A|}} \le 1
	.\] 
\end{lemma}
\begin{proof}
	Without loss of generality, let \( V = \{1, 2, \ldots , n\}   \). We proceed with a double counting argument. We let \( A \subseteq \{1, 2, \ldots ,n \}   \) be \textbf{initial} for \( \tau  \) if \( \tau (\{1, 2, \ldots , |A|\}  ) =A\). In other words, the first \( |A| \) elements map to the set \( A \). Note that \( \tau (\varnothing) \subseteq \tau (\{1\} \subseteq \tau (\{1,2\}  ) \subseteq \ldots \subseteq \tau (\{1, 2, \ldots , n\}  ) )\) are all initial sets.

	\begin{observation}
		For every permutation \( \tau  \), there is at most 1 edge \( A \in E \) such that \( A \) is initial for \( \tau  \). This is because in a Sperner hypergraph, no edge is contained in the other.
	\end{observation}

	\begin{observation}
		Every \( A \in E \) is initial for \( |A|!(n-|A|)! \) permutations. This is because there are \( |A|! \) many ways to map \( \{1, 2, \ldots , |A|\}   \) to \( A \). There are then \( (n-|A|)! \) ways to map the rest.
	\end{observation}

	Let \[
		\chi_{A, \tau }= \begin{cases}
			1, &\text{ if \( A \) initial for \( \tau  \)}\\
			0, &\text{ otherwise}
		\end{cases}
	.\] Then, \[
		\underbrace{\sum_{\tau }\sum_{A \in E} \chi_{A,\tau }}_{\le \sum_{\tau } = n!} = \underbrace{\sum_{A \in E}\sum_{\tau } \chi_{A,\tau }}_{\sum_{A \in E}|A|!(n-|A|)!}
	.\] Together, we have that \[
		n! \ge \sum_{A \in E}|A|!(n-|A|)!
	.\] such that \[
		1 \ge \sum_{A \in E} \frac{|A|!(n-|A|)!}{n!} = \frac{1}{\binom{n}{|A|}}
	.\] 
\end{proof}

And now with our proof for Sperner's theorem:

\begin{theorem}
	Sperner's
\end{theorem}
\begin{proof}
	Recall that \[
		\binom{n}{\left\lfloor \frac{n}{2} \right\rfloor} \ge \binom{n}{k}
	.\] for any \( 1\le k\le n \). Therefore, \[
		\frac{|E|}{\binom{n}{\left\lfloor \frac{n}{2} \right\rfloor}} = \sum_{A \in E} \frac{1}{\binom{n}{\left\lfloor \frac{n}{2} \right\rfloor}} \le \sum_{A \in E}\frac{1}{\binom{n}{|A|}}
	.\] which must be at most 1 from LMY. Therefore, \( |E| \le \binom{n}{\left\lfloor \frac{n}{2} \right\rfloor} \).
\end{proof}
