\lecture{14}{Fri 09 Feb 2024 14:00}{Turan's Theorem}

This graph is called the \textbf{Turan Graph} \( T(n, 2) \).

\begin{definition}
	\( T(n, r) \) is the unique (up to isomorphism) \( n \)-vertex graph whose vertex set \( V \) can be partitioned into \( r \) disjoint parts \( V_{1},V_{2},\ldots,V_r \) such that each part has roughly equal size \( \left\lfloor \frac{n}{r} \right\rfloor \) or \( \left\lceil \frac{n}{r} \right\rceil  \), and whose edge set is defined by \[
		E = \{(u, v) : u \in V_i \text{ and } v \in V_j \text{ for some \( i\neq j \).}\}  
	.\]  
\end{definition}

\begin{theorem}
	For any \( n,r\in \mathbb{Z}_{>0} \), any \( n \)-vertex graph \( G=(V,E) \) with no subgraph isomorphic to \( K_{r+1} \) has at most as many edges as the Turan graph \( T(n, r) \). Furthermore, if this graph has the same amount of edges as the Turan graph, \( T(n, r) \), then it is isomorphic to the Turan graph.
\end{theorem}
\begin{proof}
	On HW, but can use the following:
\end{proof}

\begin{definition}
	A \textbf{complete multipartite graph} is a graph \( G=(V,E) \) such that \( V \) has a partition into sets of vertices \( V_{1},\ldots V_n \) such that \[
		E = \{(u, v) : u \in V_i \text{ and } v \in V_j \text{ for } i\neq j\}  
	.\] Denote each set \( V_{1},\ldots V_n \) as a \textbf{part}.
\end{definition}

\begin{lemma}
	Let \( n,r \in Z_{>0}\), and let \( G=(V,E) \) be an \( n \)-vertex complete multipartite graph with \( \le r \) parts. Then \( G \) has at most as many edges as \( T(n, r) \), and is \( T(n, r) \) if \( = \) holds.
\end{lemma}
\begin{proof}
	Suppose \( G \) has \( \ge 2 \) vertices in its largest part than in its smallest part. We will show that \( |E| <  \) the number of edges in \( T(n, r) \). 

	While there are \( \ge 2 \) more vertices in the largest part of \( G_i \) than in its smallest part, let \( G_{i+1} \) be the complete multipartite graph formed from moving one such vertex from the smallest part to the largest part of \( G_i \). 
\end{proof}

\begin{lemma}
	This process will end, and when it does, that graph is isomorphic to \( T(n, r) \).
\end{lemma}
\begin{proof}
	HW.
\end{proof}

\begin{lemma}
	The number of edges in \( G_i \) is strictly less than the number of edges in \( G_{i+1} \).
\end{lemma}
\begin{proof}
	Let \( v \) be the vertex of \( G_i \) which was moved. Then the number of edges of \( G_i \), which are not incident to \( v \), is the number of edges in \( G_{i+1}  \) that are not incident to \( v \). Also, we have that \[
		d_{G_i}(v) = n - \text{vertices in part with \( v \)}
	.\] and \[
		d_{G_{i+1})}(v) = n - \text{vertices in new part with \( v \)}
	.\] Therefore, the number of edges in \( G_{i+1} \) is strictly greater than that of \( G_i \).
\end{proof}
