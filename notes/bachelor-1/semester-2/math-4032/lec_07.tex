\lecture{7}{Wed 24 Jan 2024 14:00}{More Bounds, Inclusion Exclusion}

The proof from previous lecture will not be given, as it is on the homework.

\begin{lemma}
	For any \( n\ge k\ge 1 \), \[
		\binom{n}{k} \le \left( \frac{en}{k} \right) ^{k} 
	.\] 
\end{lemma}

\begin{observation}
	If \( n=2k \), then \[
		\binom{2k}{k} \le \left( \frac{e\cdot 2k}{k} \right)^{k}  \le (2e)^{k} 
	.\] 
\end{observation}

\begin{lemma}
	(Bernoulli-type inequality) For any real number \( x\ge 0 \), we have \[
		1 + x \le e^{x} 
	.\] 
\end{lemma}
\begin{proof}
	This is true for \( x=0 \) because \( 1 + 0 \le e^{0} = 1 \). Therefore, we just need to show that \[
		f(x) = e^{x} - (1+x) 
	.\] is increasing. This holds because \( f'(x) = e^{x}-1\ge 0  \).
\end{proof}

\begin{lemma}
	For any \( n\ge k\ge 1 \), we have \[
		\sum_{i=0}^{k} \binom{n}{i}\le \left( \frac{en}{k}  \right) ^{k} 
	.\]
\end{lemma}
\begin{proof}
	Recall that for any \( x,y \), we have \[
		(x+y)^{n} = \sum_{i=0}^{n} \binom{n}{i}x^{i}y^{n-i}  
	.\] When \( y = 1 \), \[
		(1+x)^{n} = \sum_{i=0}^{n} \binom{n}{i}x^{i}  
	.\] Choosing \( x=\frac{k}{n} \) with the intuition that \( x < 1 \) allows us to continue:
	\[
		\sum_{i=0}^{n} \binom{n}{i}x^{i}\ge \sum_{i=0}^{k} \binom{n}{i}x^{i}  
	.\] Dividing everything by \( x^{k}  \), \[
		\frac{(1+x)^{n} }{x^{k} } = \sum_{i=0}^{n} \binom{n}{i}x^{i-k} \ge \sum_{i=0}^{k} \binom{n}{i}x^{i-k} 
	.\] Since \( x \le 1 \), \[
	\frac{(e^{x})^{n}  }{x^{k}  } \ge  \frac{(1+x)^{n} }{x^k } \ge \sum_{i=0}^{k} \binom{n}{i}x^{i-k} \ge \sum_{i=0}^{k} \binom{n}{i} 
	.\] Plugging in \( x = \frac{k}{n} \) gets our result.
\end{proof}

\begin{lemma}
	For any \( n\ge k\ge 1 \), we have \[
		\binom{n}{k} \ge \frac{n^{k}}{k^{k} } 
	.\] 
\end{lemma}
\begin{proof}
	\begin{align*}
		\binom{n}{k} &= \frac{n!}{k!(n-k)!} \\
		&= \frac{n(n-1)(n-2)\ldots (n-k+1)}{k!} \\
		&= \frac{n}{k}\cdot \frac{n-1}{k-1} \cdot \frac{n-2}{k-2} \cdot \ldots \cdot  \frac{n-k+1}{1} \\
		&\ge \frac{n^{k}}{k^{k} }
	.\end{align*}
\end{proof}

\subsection{Inclusion Exclusion}

\begin{eg}
	Say there are 20 math majors, 15 CS majors, and 5 who are majoring in both in one class. How many people are in the class?
\end{eg}
\begin{explanation}
	\( 20 + 15 - 5 = 30 \).
\end{explanation}

\begin{lemma}
	If \( A \) and \( B \) are finite sets, then \( |A \cup B| = |A| + |B| - |A \cap B| \).
\end{lemma}
\begin{proof}
	Count.
\end{proof}

\begin{lemma}
	If \( A, B, C \) are finite sets, then
	\begin{align*}
		|A \cup B \cup  C| &= |A| + |B| + |C| \\ &- |A \cap B| - |B \cap C| - | C \cap A| \\ &+ |A \cap B \cap C|
	.\end{align*}
\end{lemma}

\begin{definition}
	Given a set \( S \), and a positive integer \( k \le |S|\), we write \[
		\binom{S}{k}
	.\] to denote the set of subsets of \( S \) whose size is exactly \( k \).
\end{definition}

\begin{note}
	\[
		\left| \binom{S}{k} \right| = \binom{|S|}{k}
	.\] 
\end{note}

\begin{theorem}
	(Inclusion Exclusion Principle) If \( A_{1},A_{2},\ldots ,A_n \) are finite sets, then \[
		\left| \bigcup_{i = 1}^{n} A_i \right| = \sum_{k=1}^{n}	(-1)^{k-1} \sum_{I \in \binom{\{1,2,\ldots ,n\}  }{k}} \left| \bigcap_{i \in I} A_i \right| 
	.\] 
\end{theorem}
