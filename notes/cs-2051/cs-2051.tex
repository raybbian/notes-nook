% updated 2023-11-15 13:12:47
\documentclass[a4paper]{article}
% basics
\usepackage[utf8]{inputenc}
\usepackage[T1]{fontenc}
\usepackage[a4paper, margin=1in]{geometry}
\usepackage{textcomp}
% \usepackage[dutch]{babel}
\usepackage{cmbright}
\usepackage{url}
% \usepackage{hyperref}
% \hypersetup{
%     colorlinks,
%     linkcolor={black},
%     citecolor={black},
%     urlcolor={blue!80!black}
% }
\usepackage{graphicx}
\usepackage{float}
\usepackage{booktabs}
\usepackage{enumitem}
% \usepackage{parskip}
\usepackage{emptypage}
\usepackage{subcaption}
\usepackage{multicol}
\usepackage[usenames,dvipsnames]{xcolor}

% \usepackage{cmbright}


\usepackage{amsmath, amsfonts, mathtools, amsthm, amssymb}
\usepackage{mathrsfs}
\usepackage{cancel}
\usepackage{bm}
\newcommand\N{\ensuremath{\mathbb{N}}}
\newcommand\R{\ensuremath{\mathbb{R}}}
\newcommand\Z{\ensuremath{\mathbb{Z}}}
\renewcommand\O{\ensuremath{\emptyset}}
\newcommand\Q{\ensuremath{\mathbb{Q}}}
\newcommand\C{\ensuremath{\mathbb{C}}}
\DeclareMathOperator{\sgn}{sgn}
% \usepackage{systeme} 
% doesn't work for some stupid reason
\let\svlim\lim\def\lim{\svlim\limits}
\let\implies\Rightarrow
\let\impliedby\Leftarrow
\let\iff\Leftrightarrow
\let\epsilon\varepsilon
\usepackage{stmaryrd} % for \lightning
\newcommand\contra{\scalebox{1.1}{$\lightning$}}
% \let\phi\varphi





% correct
\definecolor{correct}{HTML}{009900}
\newcommand\correct[2]{\ensuremath{\:}{\color{red}{#1}}\ensuremath{\to }{\color{correct}{#2}}\ensuremath{\:}}
\newcommand\green[1]{{\color{correct}{#1}}}



% horizontal rule
\newcommand\hr{
    \noindent\rule[0.5ex]{\linewidth}{0.5pt}
}


% hide parts
\newcommand\hide[1]{}



% si unitx
\usepackage{siunitx}
\sisetup{locale = FR}
% \renewcommand\vec[1]{\mathbf{#1}}
\newcommand\mat[1]{\mathbf{#1}}


% tikz
\usepackage{tikz}
\usepackage{tikz-cd}
\usetikzlibrary{intersections, angles, quotes, calc, positioning}
\usetikzlibrary{arrows.meta}
\usepackage{pgfplots}
\pgfplotsset{compat=1.13}


\tikzset{
    force/.style={thick, {Circle[length=2pt]}-stealth, shorten <=-1pt}
}

% theorems
\makeatother
\usepackage{thmtools}
\usepackage[framemethod=TikZ]{mdframed}
\mdfsetup{skipabove=1em,skipbelow=0em}


\theoremstyle{definition}

\definecolor{raybbGreen}{HTML}{9fb549}
\definecolor{raybbTeal}{HTML}{699385}
\definecolor{raybbPink}{HTML}{ffafcc}

\declaretheoremstyle[
    headfont=\bfseries\sffamily\color{raybbPink!70!black}, bodyfont=\normalfont,
    mdframed={
        linewidth=2pt,
        rightline=false, topline=false, bottomline=false,
        linecolor=raybbPink, backgroundcolor=raybbPink!5,
    }
]{thmpinkbox}

\declaretheoremstyle[
    headfont=\bfseries\sffamily\color{raybbTeal!70!black}, bodyfont=\normalfont,
    mdframed={
        linewidth=2pt,
        rightline=false, topline=false, bottomline=false,
        linecolor=raybbTeal, backgroundcolor=raybbTeal!5,
    }
]{thmbluebox}

\declaretheoremstyle[
    headfont=\bfseries\sffamily\color{raybbTeal!70!black}, bodyfont=\normalfont,
    mdframed={
        linewidth=2pt,
        rightline=false, topline=false, bottomline=false,
        linecolor=raybbTeal
    }
]{thmblueline}

\declaretheoremstyle[
    headfont=\bfseries\sffamily\color{raybbGreen!70!black}, bodyfont=\normalfont,
    mdframed={
        linewidth=2pt,
        rightline=false, topline=false, bottomline=false,
        linecolor=raybbGreen, backgroundcolor=raybbGreen!5,
    }
]{thmgreenbox}

\declaretheoremstyle[
    headfont=\bfseries\sffamily\color{raybbGreen!70!black}, bodyfont=\normalfont,
    numbered=no,
    mdframed={
        linewidth=2pt,
        rightline=false, topline=false, bottomline=false,
        linecolor=raybbGreen, backgroundcolor=raybbGreen!1,
    },
    qed=\qedsymbol
]{thmproofbox}

\declaretheoremstyle[
    headfont=\bfseries\sffamily\color{raybbTeal!70!black}, bodyfont=\normalfont,
    numbered=no,
    mdframed={
        linewidth=2pt,
        rightline=false, topline=false, bottomline=false,
        linecolor=raybbTeal, backgroundcolor=raybbTeal!1,
    },
]{thmexplanationbox}



% \declaretheoremstyle[headfont=\bfseries\sffamily, bodyfont=\normalfont, mdframed={ nobreak } ]{thmgreenbox}
% \declaretheoremstyle[headfont=\bfseries\sffamily, bodyfont=\normalfont, mdframed={ nobreak } ]{thmredbox}
% \declaretheoremstyle[headfont=\bfseries\sffamily, bodyfont=\normalfont]{thmbluebox}
% \declaretheoremstyle[headfont=\bfseries\sffamily, bodyfont=\normalfont]{thmblueline}
% \declaretheoremstyle[headfont=\bfseries\sffamily, bodyfont=\normalfont, numbered=no, mdframed={ rightline=false, topline=false, bottomline=false, }, qed=\qedsymbol ]{thmproofbox}
% \declaretheoremstyle[headfont=\bfseries\sffamily, bodyfont=\normalfont, numbered=no, mdframed={ nobreak, rightline=false, topline=false, bottomline=false } ]{thmexplanationbox}

\declaretheorem[style=thmpinkbox, name=Definition]{definition}
\declaretheorem[style=thmbluebox, numbered=no, name=Example]{eg}
\declaretheorem[style=thmgreenbox, name=Proposition]{prop}
\declaretheorem[style=thmgreenbox, name=Theorem]{theorem}
\declaretheorem[style=thmgreenbox, name=Lemma]{lemma}
\declaretheorem[style=thmgreenbox, numbered=no, name=Corollary]{corollary}

\declaretheorem[style=thmproofbox, name=Proof]{replacementproof}
\renewenvironment{proof}[1][\proofname]{\vspace{-10pt}\begin{replacementproof}}{\end{replacementproof}}


\declaretheorem[style=thmexplanationbox, name=Explanation]{tmpexplanation}
\newenvironment{explanation}[1][]{\vspace{-10pt}\begin{tmpexplanation}}{\end{tmpexplanation}}

\declaretheorem[style=thmblueline, numbered=no, name=Remark]{remark}
\declaretheorem[style=thmblueline, numbered=no, name=Note]{note}

\newtheorem*{notation}{Notation}
\newtheorem*{previouslyseen}{As previously seen}
\newtheorem*{problem}{Problem}
\newtheorem*{observation}{Observation}
\newtheorem*{property}{Property}
\newtheorem*{intuition}{Intuition}


\usepackage{etoolbox}
\AtEndEnvironment{vb}{\null\hfill$\diamond$}%
\AtEndEnvironment{intermezzo}{\null\hfill$\diamond$}%
% \AtEndEnvironment{opmerking}{\null\hfill$\diamond$}%

% http://tex.stackexchange.com/questions/22119/how-can-i-change-the-spacing-before-theorems-with-amsthm
\makeatletter
% \def\thm@space@setup{%
%   \thm@preskip=\parskip \thm@postskip=0pt
% }

\newcommand{\oefening}[1]{%
    \def\@oefening{#1}%
    \subsection*{Oefening #1}
}

\newcommand{\suboefening}[1]{%
    \subsubsection*{Oefening \@oefening.#1}
}

\newcommand{\exercise}[1]{%
    \def\@exercise{#1}%
    \subsection*{Exercise #1}
}

\newcommand{\subexercise}[1]{%
    \subsubsection*{Exercise \@exercise.#1}
}


\usepackage{xifthen}

\def\testdateparts#1{\dateparts#1\relax}
\def\dateparts#1 #2 #3 #4 #5\relax{
    \marginpar{\small\textsf{\mbox{#1 #2 #3 #5}}}
}

\def\@lecture{}%
\newcommand{\lecture}[3]{
    \ifthenelse{\isempty{#3}}{%
        \def\@lecture{Lecture #1}%
    }{%
        \def\@lecture{Lecture #1: #3}%
    }%
    \subsection*{\@lecture}
    %\testdateparts{#2}
}

% \renewcommand\date[1]{\marginpar{#1}}


% fancy headers
% \usepackage{fancyhdr}
% \pagestyle{fancy}

% \fancyhead[LE,RO]{Gilles Castel}
% \fancyhead[RO,LE]{\@lesson}
% \fancyhead[RE,LO]{}
% \fancyfoot[LE,RO]{\thepage}
% \fancyfoot[C]{\leftmark}

\makeatother




% notes
% \usepackage{todonotes}
% \usepackage{tcolorbox}

% \tcbuselibrary{breakable}

% \newenvironment{verbetering}{\begin{tcolorbox}[
%     arc=0mm,
%     colback=white,
%     colframe=green!60!black,
%     title=Opmerking,
%     fonttitle=\sffamily,
%     breakable
% ]}{\end{tcolorbox}}

% \newenvironment{noot}[1]{\begin{tcolorbox}[
%     arc=0mm,
%     colback=white,
%     colframe=white!60!black,
%     title=#1,
%     fonttitle=\sffamily,
%     breakable
% ]}{\end{tcolorbox}}


% my stuff
\usepackage{algorithm,algorithmicx,algpseudocode}
\DeclareMathOperator{\Span}{span}
\DeclareMathOperator{\Var}{Var}
\DeclareMathOperator{\Dim}{dim}
\usepackage{listings}
\lstset{
	commentstyle=\color{raybbGreen},
	keywordstyle=\color{raybbTeal},
	stringstyle=\color{raybbGreen},
	basicstyle=\ttfamily\footnotesize,
	breaklines=true,
	showspaces=false,
	showstringspaces=false,
	tabsize=4
}
\renewcommand{\descriptionlabel}[1]{%
  \hspace\labelsep \upshape\bfseries #1:%
}
\renewcommand{\qedsymbol}{$\blacksquare$}


% figure support
\usepackage{import}
\usepackage{xifthen}
\pdfminorversion=7
\usepackage{pdfpages}
\usepackage{transparent}
\newcommand{\incfig}[1]{%
    \def\svgwidth{\columnwidth}
    \import{./figures/}{#1.pdf_tex}
}

% %http://tex.stackexchange.com/questions/76273/multiple-pdfs-with-page-group-included-in-a-single-page-warning
\pdfsuppresswarningpagegroup=1

\hfuzz=10pt 

\author{Raymond Bian}

\title{cs-2051}
\begin{document}
    \maketitle
    \tableofcontents\lecture{1}{Wed 24 Aug 2023 12:54}{Logical Equivalencies}

\lecture{2}{Wed 04 Oct 2023 12:58}{Quantifiers}

\lecture{3}{Wed 04 Oct 2023 12:58}{Nested Quantifiers}

\lecture{4}{Wed 04 Oct 2023 12:58}{Rules of Inference}

\lecture{5}{Wed 04 Oct 2023 12:58}{Proof by Contradiction}

\lecture{6}{Wed 04 Oct 2023 12:59}{Proof by Cases, Colorings, and Invariants}

\lecture{7}{Wed 04 Oct 2023 12:59}{Advanced Proof Techniques}

\lecture{8}{Wed 04 Oct 2023 12:59}{Intro to Sets}

\lecture{9}{Wed 04 Oct 2023 12:59}{Set Proofs and Relations}

\lecture{10}{Wed 04 Oct 2023 13:00}{Exam 1 Review}

\lecture{11}{Tue 03 Oct 2023 14:01}{Cardinality; Countable Sets}

We used sets to talk about \textit{relations}. Depending what relations we were looking at, we could determine whether they were \textit{reflextive, symmetric, antisymmetric, or transitive.}

\begin{definition}
	An \textbf{equivalence relation} is one that is reflexive, symmetric, and transitive.
\end{definition}

\begin{definition}
	An \textbf{ordering relation} is one that is reflexive, antisymmetric, and transitive.
\end{definition}

\section{Functions}

Below are some definitions that we use for functions.

\begin{definition}
	A \textbf{function} is a relation such that \( \forall a \exists !b (f(a) = b) \).
\end{definition}

\begin{definition}
	For a particular function \( f \colon A \to B  \), we call \( A \) the \textbf{domain} and \( B \) the \textbf{codomain}.
\end{definition}

\begin{definition}
	The \textbf{range} of \( f \) is \( \{b \in  B ~|~ \exists  a \in  A (f(a) = b)\}   \)
\end{definition}

\begin{definition}
	We say \( f \) is \textbf{one-to-one} if \( \forall x,y \in A(x \neq  y \implies f(x) \neq f(y)) \). Another word for this is \textbf{injective}. We can use the contrapositive to prove a function \( f \) is injective.
\end{definition}

\begin{definition}
	We say \( f \) is \textbf{onto} if \( \forall b \in B (\exists a \in  A(f(a) = b)) \).
\end{definition}

\begin{definition}
	A \textbf{bijection} is a function \( f \) that is one-to-one and onto.
\end{definition}

\begin{definition}
	Given a bijection \( f \), we define the \textbf{inverse function} \( f^{-1} \)
\begin{align*}
	& f^{-1} \colon B \to A \\
	& f^{-1}(b) = a \iff f(a) = b
\end{align*}
\begin{property}
	\( f^{-1} \) is a bijection.
\end{property}
\end{definition}

\begin{eg}
	Consider:
	\begin{align*}
		f \colon & \mathbb{R} \to \mathbb{R}^+ \\
						 & x \to  x^2
	.\end{align*}
	This function by itself is not one-to-one. But note that we can make restrictions in our domain and codomain to make this function a bijection.
\end{eg}
\begin{eg}
	Consider:
	\[
		\sin \colon \mathbb{R} \to \mathbb{R}
	.\] 
	This function is not one-to-one and onto (not a bijection). Therefore, to find the inverse of the function, we restrict the domain and codomain.	\[
		\sin \colon \left[-\frac{\pi}{2}, \frac{\pi}{2}\right] \to [-1,1]
	.\] 
\end{eg}

\begin{definition}
	Let \( S \) be a set. We say \( S \) has \textbf{cardinality} \( n \) if \( S \) has exactly \( n \) elements (there is a bijection from \( S \) to the set \( [n]=\{1, 2, 3, \ldots , n\}   \)). We write \( |S|=n \).
\end{definition}
\begin{remark}
	Cardinality is \textit{well defined} because you have no bijection between \( [n] \) and \( [m] \) for \( n \neq  m \).
\end{remark}
\begin{remark}
	If \( S \) is infinite, then we say \( |S| = \infty \). In other words, there does not exist a biection from \( S \) to any \( [n] \).
\end{remark}

\begin{definition}
	Let \( A \) and \( B \) be sets. We say \( |A| = |B| \) if there is a bijection from \( A \) to \( B \).
\end{definition}

\begin{eg}
	Let \( \sim \) be a relation on \( \mathcal{P}(U) \). \( A \sim B \) if there is a bijection from \( A \) to \( B \). Are all infinite sets the same?
\end{eg}

\section{Countability}
What does it mean for a set to be countable?

\begin{definition}
	A set \( S \) is \textbf{countable} if there is a one-to-one mapping (\( \exists f \colon S \to \mathbb{N} \) that is a bijection) from \( S \) to \( \mathbb{N} \).
\end{definition}
\begin{itemize}
	\item Every finite set is countable.
	\item \( \mathbb{N} \) is countable (you can map \( \mathbb{N} \) to itself).
	\item \( 2 \mathbb{N} = \{2a ~|~ a \in  \mathbb{N}\}   \) is countable.
		\begin{align*}
			f \colon & 2 \mathbb{N} \to  \mathbb{N} \\
							 & 2a \to a
		.\end{align*}
		\begin{remark}
			There is a one-to-one mapping between sets \( 2\mathbb{N} \) and \( \mathbb{N} \). That means they have the same cardinality, even if this goes against our intuition.
		\end{remark}
\end{itemize}


\lecture{12}{Thu 05 Oct 2023 13:35}{Properties of Countable Sets; Cantor's Diagonalization Argument}

Continuing on from last time, we also have that \( \mathbb{Z} \) is countable as well.

\begin{theorem}
	\( \mathbb{Z} \) is countable.
\end{theorem}
\begin{proof}
	Construct \( f \colon \mathbb{Z} \to  \mathbb{N} \). Define:
	\[
		f(n) = \begin{cases}
			0 & n = 0 \\
			2n & n > 0 \\
			-2n + 1 & n < 0
		\end{cases}
	.\] We wish to prove that \( \forall x,y (x \neq  y \implies f(x) \neq  f(y)) \). Proceeding by contraposition, given \( f(n)=f(m) \), we have the following cases:
	\begin{description}
		\item[Case 1:] \( n = 0 \). Then \( f(n) = 0 = f(m) \implies m = 0\).
		\item[Case 2:] \( n > 0, m > 0 \). Then \( f(n) = 2n = f(m) = 2m \implies 2n = 2m \implies n = m \)
		\item[Case 3:] \( n > 0, m < 0 \). Then \( f(n) \) is even and \( f(m) \) is odd \contra. Vacuously true.
	\end{description}
	The other cases are analogous!
\end{proof}

\begin{property}
	Let \( A,B \) be countable sets. There are the following properties of countable sets:
	\begin{itemize}
		\item \( \{a\} \cup A  \) is countable.
			\begin{proof}
				We wish to find \( g \colon \{a\}  \cup  A \to \mathbb{N}  \). Construct \[
					g(x) = \begin{cases}
						0 & x = a \\
						f(x) + 1 & x \in A
					\end{cases}
				.\] 		
			\end{proof}
		\item \( F \cup A \) is also countable for any finite set \( F \).
		\item \( A \cup  B \) is also countable.
		\item If \( f \colon S \to A \) is one-to-one and \( A \) is countable, then \( S \) is countable.
		\item \( A \times B \) is countable.
		\item \( S \subseteq A \) is countable.
			\begin{proof}
				Given \( f \colon A \to  \mathbb{N} \) is one-to-one, its restriction to \( S \) is also one-to-one.
			\end{proof}
		\item \( \mathbb{Q} \) is countable.
			\begin{proof}
				This is because \( \mathbb{Q} \subseteq \mathbb{Z} \times \mathbb{Z} \). 
			\end{proof}
	\end{itemize}
\end{property}

\subsection{Cantor's Diagonalization Argument}
What about the real numbers?
\begin{theorem}
	\( \mathbb{R} \) is not countable (Cantor).
\end{theorem}
\begin{proof}
	Prove instead that \( (0,1) \) (a subset of \( \mathbb{R} \)) is \textbf{not} countable. We will argue by contradiction. We assume that \( (0,1) \) is countable. Hence, there exists \[
		f \colon (0,1) \to \mathbb{N} 
	.\] shown below.

	\begin{table}[H]
		\caption{Our one-to-one mapping (pages of our book)}\label{tab:}
		\begin{center}
			\begin{tabular}[c]{|l|l|}
				\hline
				\multicolumn{1}{|c|}{\textbf{\( \mathbb{N} \)}} & 
				\multicolumn{1}{c|}{\textbf{\( (0,1) \)}} \\
				\hline
				0 & \( 0.d_{11} d_{12} d_{13} d_{14} \ldots  \) \\
				1 & \( 0.d_{21} d_{22} d_{23} d_{24} \ldots  \) \\
				2 & \( 0.d_{31} d_{32} d_{33} d_{34} \ldots  \) \\
				3 & \( 0.d_{41} d_{42} d_{43} d_{44} \ldots  \) \\
				\vdots & \\
				\hline
			\end{tabular}
		\end{center}
	\end{table}

	Let's define \( b \in \mathbb{R} \) as follows: \[
		b = 0.b_1b_2b_3b_4 \ldots
	.\] where \[
		b_i = \begin{cases}
			3 & d_{ii} \neq 3 \\
			0 & d_{ii} = 3
		\end{cases}
	.\] Then, we claim that \( f(b) \) is not well defined! In other words, there should exist \( k \in \mathbb{N}\) in our book such that \( f(b) = k \). However, we have constructed \( b \) such that \( b_k \neq d_{kk} \): there is no \( k \in \mathbb{N} \) that exists in our book \contra (\( b \) will always have one bit that is different from all entries in our mapping)!
\end{proof}

\exercise{1}
Prove that \( |S| \neq |\mathcal{P}(S)| \).

\lecture{13}{Thu 12 Oct 2023 14:15}{Runtime Complexity}

\section{Analyzing Runtimes}

How many comparisons should we make to choose the best out of \( n \) restuarants?

\subsection{Big O Notation}
Big O Notation does not care about constants.

\begin{definition}
	Given \( f,g \colon \mathbb{R} \to \mathbb{R} \), we say \( f \) is big-O of \( g \) (write \( f=O(g) \)) if \[ \exists C>0, k \in \mathbb{R}(|f(x)| \le  C \cdot |g(x)| \quad \forall  x \ge  k) \] where \( C \) and \( k \) are arbitrary constants (witnesses).
\end{definition}

\begin{eg}
	Given \( f,g \colon \mathbb{N} \to \mathbb{R} \), is \( f=O(g) \):
	\begin{itemize}
		\item \( f(n) = 2n+1 \), \( g(n) = n - 10 \)? Yes.
		\item \( f(n) = 2n+1 \), \( g(n) = n + \sqrt{n}  \)? Yes.
			\begin{proof}
				We have:
				\begin{align*}
					2n & \le 2n && \forall  n \in \mathbb{N} \\
					1 & \le 2\sqrt{n} && \forall n \in  \mathbb{N}, n > 0
				.\end{align*}
				Adding both sides, we have \[
					f(n) \le  2g(n) \quad \forall n\ge 1
				.\] Therefore, \( f = O(g) \) for \( C = 2, k = 1 \).
			\end{proof}
		\item \( f(n) = n\log (n) \), \( g(n) = n + \sqrt{n}  \)? No.
			\begin{proof}
				We proceed by contradiction. Assume \( f = O(g) \). Then, there exists \( C>0 \) and \( k \in \mathbb{N} \) such that \[
					n\log (n) \le  C(n + \sqrt{n} ) \quad \forall n \ge k
				.\] Simplifying, we have: 
				\begin{align*}
					\log (n) & \le  C \left(\frac{n+ \sqrt{n}}{n}\right) \\
										& = C\left(\frac{n}{n} + \frac{\sqrt{n}}{n}\right) \\
										& = C\left(1 + \frac{1}{\sqrt{n} }\right) \\
										& \le 2C \qquad \forall n \ge  1
				.\end{align*}
				This inequality yields a contradiction, as \( \log (n) \) grows to infinity as \( n \to \infty \) and therefore cannot be bounded by a constant \contra.
			\end{proof}
	\end{itemize}
\end{eg}

\begin{notation}
	Let \( \mathcal{F} = \{f \colon \mathbb{N}\to \mathbb{R}\}   \). We say \( f \sim g \) if \( f = O(g) \) and \( g = O(f) \).
	\begin{note}
		This relation between functions \( f\sim g \) is reflextive, symmetric, and transitive.
	\end{note}
\end{notation}

Note that proving the big-O of a function with witnesses is very time-consuming. There are several properties of big-O that we can use to simplify proofs. 
\begin{property}
	Let \( f_i,g_i \colon \mathbb{N} \to  \mathbb{R}\)
	\begin{itemize}
		\item If \( f = O(G) \), then \( f + \alpha = O(g) \) for \( \alpha \in \mathbb{R} \).
			\begin{note}
				As long as \( g \not\to 0 \) as \( x \to \infty \).
			\end{note}
		\item If \( f_{1}=O(g)\), \( f_{2} = O(g) \), then \( f_{1} + f_{2} = O(g) \)
		\item If \(	p(x) = a_n x^n + a_{n-1}x^{n-1} + \ldots + a_1 x + a_0 \), then \( p(x) = O(x^n) \).
		\item If \( f_{1}=O(g_{1}), f_{2}=O(g_{2}) \), then \( f_{1}f_{2} = O(g_{1}g_{2}) \).
		\item \( 1 \le  \log (n) \le  n^{\alpha} \le n^{\beta} \le n^{\beta}\log (n) \le 2^n \) for \( \beta > \alpha > 0 \)
	\end{itemize}
\end{property}

\begin{eg}
	Going back to the restaurants, we have two forming strategies:
	\begin{enumerate}
		\item Compare one restuarant at a time.
		\item Pair the restaurants. Keep the best of each pair. Repeat.
	\end{enumerate}
	Let \( T(n) =  \) the number of meals needed to find the best restaurant out of a list of length \( n \). Then, for the first strategy, we have: \[
		T(n) = T(n - 1) + 2
		.\] For the second strategy, we have: \[
			T(n) = T\left(\frac{n}{2}\right) + n
		.\] 

	We can solve the first strategy's recurrence relation with substitution:
	\begin{align*}
		T(n) & = T(n- 1) + 2 \\
					& = T(n-2) + 2 + 2 \\
					& = T(n-3) + 2 + 2 + 2 \\
					& (\ldots) \\
					& = T(n - (n - 1)) + 2(n - 1) \\
					& = 2(n - 1) \tag{\( T(1) = 0 \)}
	.\end{align*}

	And for the second:
	\begin{align*}
		T(n) &= T\left(\frac{n}{2}\right) + n \\
					&= T\left(\frac{n}{4}\right) + \frac{n}{2} + n \\
					&= T\left(\frac{n}{8}\right) + \frac{n}{4} + \frac{n}{2} + n \\
					&(\ldots ) \\
					&=n\left(1 + \frac{1}{2} + \frac{1}{4} + \ldots + \frac{1}{2^{k-1}}\right) \\
	.\end{align*}
\end{eg}

\lecture{14}{Tue 17 Oct 2023 14:17}{Solving Recurrence Relations}

\section{Induction}

\begin{eg}
	Take from last lecture the following recurrence relation: \[
		T(n) = T(n - 1) + 2, \quad T(1) = 0
	.\] 
	We can "guess" the solution by looking at the relation. We guess \( T(n) = 2 \cdot n \). Then, we have: \[
		T(n) = 2(n - 1) + 2 = 2n - 2 + 2 = 2n, \quad \text{but } T(1) \neq  2
	.\] 
	So instead, we guess \( T(n) = 2 \cdot (n-1) \), which satisfies our base case.
\end{eg}

We use induction to check if a statement \( P(n) \) is true for all \( n \in \mathbb{N} \). We check that 
\begin{enumerate}
	\item \( P(0) \) is true. This is called the base case of induction.
	\item Assume \( P(k) \) is true for some \( k \in \mathbb{N} \). This is called the inductive hypothesis.
	\item Prove that \( P(k) \implies P(k+1) \). This is called the step of induction.
\end{enumerate}

\begin{eg}
	We will prove that for all \( n \ge 1, n \in \mathbb{N} \), \[
		1 + 2 + \ldots + n = \frac{n(n+1)}{2}
	.\] Let \( P(n) \coloneq 1 + 2 + \ldots  + n = \frac{n(n+1)}{2}\). Formally, we wish to prove \( \forall n P(n) \), \( n \ge 1, n \in \mathbb{N} \). We will proceed with induction on \( n \).
	\begin{description}
		\item[Base case:] \( n = 1 \). Then, we have \[
			P(1) \coloneq 1 = \frac{1(1+1)}{2} = 1
		.\]
		\item[Step:] We assume that \( P(k) \) is true, where \( k \ge 1, k \in \mathbb{N} \). We wish to show that \( P(k+1) \) is true. From our inductive hypothesis, we have \[
			1 + 2 + \ldots + k = \frac{k(k+1)}{2}
		.\] Adding \( k + 1 \) to both sides, we have \[
			1 + 2 + \ldots + k + (k+1) = \frac{k(k+1)}{2} + k + 1 = \frac{(k+1)(k+2)}{2}
		.\] The last equality confirms that \( P(k+1) \) is true.
	\end{description}
	It follows by induction that \( P(n) \) is true for any \( n \ge 1, n \in \mathbb{N} \).
\end{eg}

\begin{eg}
	Let \( H_n = 1 + \frac{1}{2} + \frac{1}{3} + \ldots  + \frac{1}{n} \). We wish to prove that \( H_{2^n} \ge  1 + \frac{n}{2} \) for \( n \in \mathbb{N} \).
	\begin{description}
		\item[Base case:] \( n = 0 \). Then, we have \[
				H_{2^0} = H_1 = 1 \ge 1 + \frac{0}{2} = 1 
			\]
		\item[Step:] Let \( P(n) \coloneq H_{2^n} \ge 1 + \frac{n}{2} \). We assume \( P(k) \) is true for some \( k \in \mathbb{N} \). We will prove that \( P(k) \implies P(k+1) \). From our inductive hypothesis, we have \[
			1 + \frac{1}{2} + \frac{1}{3} + \ldots  + \frac{1}{2^k} \ge 1 + \frac{k}{2}
		.\] We add to both sides \(\frac{1}{2^k+1} + \frac{1}{2^k+2} + \ldots + \frac{1}{2^{k+1}}\) such that \[
			H_{2^k+1} \ge  1 + \frac{k}{2} + \frac{1}{2^k+1} + \frac{1}{2^k+2} + \ldots + \frac{1}{2^{k+1}}
		.\] It is enough to show that \( \frac{1}{2^k+1} + \frac{1}{2^k+2} + \ldots + \frac{1}{2^{k+1}} \ge \frac{1}{2} \). Then, we have \[
			\frac{1}{2^k+1} + \frac{1}{2^k+2} + \ldots + \frac{1}{2^{k+1}} \ge \frac{1}{2^{k+1}} + \frac{1}{2^{k+1}} + \ldots + \frac{1}{2^{k+1}} = 2^k \cdot  \frac{1}{2\cdot 2^k} = \frac{1}{2} 
		.\] Combining these inequalities, we have \[
			H_{2^{k+1}} \ge 1 + \frac{k}{2} + \frac{1}{2^k+1} + \ldots + \frac{1}{2^{k+1}} \ge 1 + \frac{k}{2} + \frac{1}{2} = 1 + \frac{k+1}{2} 
		\] Therefore, \( P(k+1) \) is true.
	\end{description}
\end{eg}

\lecture{15}{Thu 19 Oct 2023 14:02}{Induction Continued}

\begin{eg}
	Given \( \alpha  \in \mathbb{R} \), \( \alpha  >  0\), \( \alpha  \neq  1 \). Show that \[
		1 + \alpha  + \alpha ^2 + \ldots + \alpha ^n = \frac{1-\alpha ^{n+1}}{1 - \alpha }
	.\] 
\end{eg}

\begin{proof}
	We will proceed with induction on \( n \).
	\begin{description}
		\item[Base case:] \( n = 0 \). Then, we have \[
			1 = \alpha ^0 = \frac{1-\alpha ^{0+1}}{1-\alpha } = \frac{1-\alpha }{1-\alpha } = 1
		.\] 
		\item[Step:] Assume that \[
			1 + \alpha  + \alpha ^2 + \ldots  + \alpha ^k = \frac{1-\alpha ^{k+1}}{1-\alpha } \qquad \text{for some } k \ge 0
		.\] From the assumption (inductive hypothesis), we have 
		\begin{align*}
			1 + \alpha  + \alpha ^2 + \ldots  + a^{k+1} &= (1 + \alpha  + \alpha^2 + \ldots  + \alpha ^k) + \alpha^{k+1} \\
																									&= \frac{1-\alpha ^{k+1}}{1-\alpha } + \alpha ^{k+1} \\
																									&= \frac{1-\alpha ^{k+1} + (1-\alpha )\alpha ^{k+1}}{1-\alpha } \\
																									&= \frac{1-\alpha ^{k+1} + \alpha ^{k+1} - \alpha ^{k+2}}{1-\alpha } \\
																									&= \frac{1-\alpha ^{(k+1) + 1}}{1-\alpha }
		.\end{align*}
		which was what we wanted.
	\end{description}
\end{proof}

\begin{eg}
	Show that for every \( n \in \mathbb{N} \), \( n \ge 1 \), 21 divides \( 4^{n+1} + 5^{2n-1} \).
\end{eg}

\begin{proof}
	We will proceed with induction on \( n \). Let \( P(n) \) be the statement that \( 21 \mid 4^{n+1} + 5^{2n-1} \). We wish to prove that \( P(n) \) is true for every \( n \in \mathbb{N} \), \( n \ge 1 \).
	\begin{description}
		\item[Base case:] \( n = 1 \). Then, we have \[
				4^1+5^{2 \cdot 1 - 1} = 4^2 + 5 = 9 = 21
		.\] 
		\item[Step:] We assume \( P(k) \) is true for \( k \ge 1 \). We wish to show that \( P(k+1) \) is true as well. In other words, we wish to show that 21 divides \[
			4^{(k+1)+1} + 5^{2(k+1)-1}
		.\] We have:
		\begin{align*}
			4^{(k+1)+1} + 5^{2(k+1)-1} &= 4 \cdot 4^{k+1} + 5^2 \cdot 5^{2k-1} \\
																	&= 4 \cdot 4^{k+1} + 25 \cdot 5^{2k-1} \\
																	&= 4 \cdot 4^{k+1} + (21 + 4) \cdot 5^{2k-1} \\
																	&= 4 \cdot 4^{k+1} + 21 \cdot 5^{2k-1} + 4 \cdot 5^{2k-1} \\
																	&= 4 \cdot \left( 4^{k+1} + 5^{2k-1}\right) + 21 \cdot 5^{2k-1}
		.\end{align*}
		Then, from our assumption, we have 
		\begin{align*}
			4^{(k+1)+1} + 5^{2(k+1)-1}  &= 4 \cdot 21\cdot q + 21 \cdot 5^{2k-1}  \\
																	&= 21 \cdot (4q + 5^{2k-1})
		.\end{align*}
		By definition, this means that 21 divides \( 4^{(k+1)+1} + 5^{2(k+1)-1} \), or \( P(k+1) \) is true, which was what we wanted.
	\end{description}
\end{proof}

\begin{eg}
	Given a complete set of triominoes, we want to tile an \( n\times n \) board. 
	\begin{observe}
		The board cannot be tiled if \( n \) is not divisible by 3, because \( n^2 \) must be divisible by 3 for the board to be tiled.
	\end{observe}
	We want to show that we can tile any \( n \times n \) board with top left corner removed if \( n \) is a power of 2.
\end{eg}

\begin{proof}
	We will proceed with induction on \( n \).
	\begin{description}
		\item[Base case:] \( n = 1 \). It is clear that we can fit a triomino in the 3 squares of the board.	
		\item[Step:] \( n > 1 \). We assume it is possible to tile some \( 2^k \times 2^k \) board with top left corner removed. We wish to show that it is possible to tile the \( 2^{k+1} \times 2^{k+1}   \) board with top left corner removed as well. Here \( k \in \mathbb{N} \), \( k \ge 1 \). By partitioning the board into 4 smaller squares, we can apply our assumption to tile the entire board (see figure below).
	\end{description}
\end{proof}

\begin{figure}[H]
    \centering
    \incfig{triomino-tiling}
    \caption{Triomino Tiling}
    \label{fig:triomino-tiling}
\end{figure}

\begin{eg}
	We have a group of  \( N \) people. One person is a celebrity if everybody knows that person \textbf{and} that person knows no one.
\end{eg}

\exercise{1}
Is \( 4^n - 1 \) a multiple of 3?

\lecture{16}{Tue 24 Oct 2023 14:08}{Strong Induction}

Continuing the example from last time, we want to see for any group if there exists a celebrity. To do so, we can ask any person \( A \) whether or not they know person \( B \). How many questions will we have to ask?

Let \( T(n) \) be the minimum number of questions we must ask to determine if there is a celebrity. Note that if \( A \) knows \( B \), \( A \) cannot be the celebrity. Also note that if \( A \) doesn't know \( B \), then \( B \) cannot be the celebrity. Then, our recurrence relation is defined by \[
	T(n) = 1 + T(n-1)
.\] We wish to show that we can find whether if there is a celebrity (or not) in at most \( 3(n-1) \) questions.

\begin{note}
	The base case is \( T(2) = 2 \). This is because if \( A \) knows \( B \), you still need to check whether or not \( B \) knows \( A \).
\end{note}

\begin{proof}
	We will proceed with induction. 
	\begin{description}
		\item[Base case:] \( n=2 \). We will ask both people, which is sufficient to determine if there is a celebrity. We have \( 2 \le  3 = 3(2-1) \), as desired.
		\item[Step case:] Assume that for any group of \( k \) people, we can determine if there is a celebrity by asking at most \( 3(k-1) \) questions. We wish to show that we can determine the existence of a celebrity in a group of \( k+1 \) people in at most \( 3(k+1-1) =3k \) questions. Pick two people \( A \) and \( B \), and ask if \( A \) knows \( B \). Then, there are two cases:
			\begin{description}
				\item[Case 1:] \( A  \) does not know \( B \). Then, we know that \( B \) cannot be a celebrity. By the inductive hypothesis applied to the original group without \( B \), we can find whether or not there exists a (candidate) celebrity \( C \) in \( 3(k-1) \) steps. To confirm that \( C \) is indeed a celebrity in the group with \( B \), we must check that \( B \) knows \( C \), and \( C \) does not know \( B \). This yields \[
						1 + 3(k-1) + 2 = 3(k+1-1) = 3k
					.\] total questions, as desired.
				\item[Case 2:] \( A \) knows \( B \). Left as an exercise to the reader!
			\end{description}
	\end{description}
	As we have verified both the base and step case of induction, it follows that we can determine the existence of a celebrity in a group of \( n \) people in at most \( 3(n-1) \) moves.
\end{proof}

\subsection{Strong Induction}
Instead of assuming that the previous case is true, we assume that all previous cases are true. In other words, we check that \( P(0) \) is true, and we check that \( P(0) \land P(1) \land P(2) \land \ldots \land P(k) \implies P(k+1) \) is true for all \( k \in \mathbb{N} \). If both are true, we can then similarly conclude that \( P(k) \) is true for all \( k\in \mathbb{N} \).

\begin{eg}
	Show that for all natural numbers \( n\ge 2 \) have a decomposition into prime factors. 
\end{eg}

\begin{proof}
	We will proceed with strong induction on \( n \).
	\begin{description}
		\item[Base case:] \( n=2 \) is prime, as desired.
		\item[Step case:] Assume every natural number of up \( k \) can be decomposed into prime factors. We wish to show that \( k+1 \) can be decomposed into prime factors as well.
			\begin{description}
				\item[Case 1:] \( k+1 \) is prime. Then the decomposition is trivial.
				\item[Case 2:] Otherwise, \( k+1 \) is composite. Then, there exists \( a, b \in \mathbb{N} \) such that \( k+1 = ab \), \( 2\le a,b\le k \). By the inductive hypothesis, \( a \) and \( b \) can be decomposed into prime factors. Then, \( k+1 \) can be decomposed into prime factors as well, as desired.
			\end{description}
			In both cases, \( k+1 \) can be decomposed into prime factors.
	\end{description}
	As we have verified both the base and step case of induction, it follows that for all \( n\in \mathbb{N} \), \( n\ge 2 \), \( n \) can be decomposed into prime factors.
\end{proof}

\begin{note}
	For strong induction, we must carefully choose our base case(s), as shown below.
\end{note}

\begin{eg}
	(Making change/coin change) There are 4 pesos bills and 5 pesos bills in an unknown country. What is the minimum value \( \zeta \) such that one can make change for all \( n \ge \zeta \)?
\end{eg}

\begin{proof}
	We wish to show that \( \zeta=12 \). We will proceed with strong induction on \( n \).
	\begin{description}
		\item[Base case:] \( n=12=4+4+4 \), \( n=13=4+4+5 \), \( n=14=4+5+5 \), \( n=15=5+5+5 \).
		\item[Step case:] Assume one can make change for all values \( [12, k] \) for some natural number \( k \). We wish to show that we can make change for \( k+1 \). To make change for \( k+1 \), we can use the change for \( k-3 \) and add one 4 peso bill.
	\end{description}
\end{proof}

\subsection{Induction and Recursion}

\begin{eg}
	Let \( \{a_n\}_{n \in \mathbb{N}} \) be a sequence of numbers such that \( a_{0}=1, a_{1}=3, a_{2}=9, a_n=a_{n-1} + a_{n-2} + a_{n-3}\) for all \( n \ge 3 \). Prove that \( a_n \le 3^n \).
\end{eg}

\begin{proof}
	We will proceed with strong induction on \( n \).
	\begin{description}
		\item[Base case:] For \( n=0 \), we have \( a_0=1=3^0 \), as desired. For \( n=1 \), we have \( a_1=3=3^1 \), as desired. And for \( n=2, a_2=9=3^2 \), as desired.
		\item[Step case:] Assume \( a_k \le 3^k \) for all values up to \( k \in \mathbb{N} \). We wish to show that \( a_{k+1}\le 3^{k+1}  \). We know that 
			\begin{align*}
				a_{k+1} &= a_k+a_{k-1}+a_{k-2} \\
								&\le 3^k + 3^{k-1} + 3^{k-2} \tag {Follows from inductive hypothesis} \\
								&< 3^k+3^k+3^k \\
								&= 3^{k+1}
			.\end{align*}
			as desired.
	\end{description}
	We have verified the base and step of induction. It follows that \( a_n \le 3^n \) for all \( n \in \mathbb{N} \).
\end{proof}

\lecture{17}{Thu 26 Oct 2023 14:03}{Last of Induction}

\begin{eg}
	The Fibonacci Sequence is defined as follows:
	\begin{align*}
		&f_{0}=f_{1}=1 \\
		&f_n=f_{n-1}+f_{n-2} \qquad \forall n\ge 2
	.\end{align*}
	We can do two things to compute the value of \( f_n \). If \( n=0,1 \), then it will output \( 1 \). Otherwise, we return \( f_{n-1}+f_{n-2} \). Note that we can also accomplish the following (more efficiently) with memoization.

	Then, let \( T(n) \) be the number of operations to get \( f_n \). Note that \( T(n) \ge  c+ T(n-1) + T(n-2) \), which implies that the number of steps to calculate the \( n \)-th Fibonacci number is greater than the \( n \)-th Fibonacci number!

	Show that \( f_n \ge \alpha ^{n-2}  \), where \( \alpha =\frac{1+\sqrt{5} }{2} \).
\end{eg}

\begin{proof}
	Let us proceed by induction. Note that for \( n=0 \), we have \( f_{0}=1>\alpha ^{0-1} \) and for \( n=1 \), \( f=1>\alpha ^{1-2}  \). Then, it suffices to verify the step case. Assume \( f_k \ge \alpha ^{k-2}  \) for some \( k \in \mathbb{N} \ge 0 \). Then, we have:
	\begin{align*}
		f_{k+1}&=f_k+f_{k-1} \tag{Given} \\
					 &\ge \alpha ^{k-2}+\alpha ^{(k-1)-2}   \tag{By I.H.} \\
					 &=\alpha ^{k-2} + \alpha ^{k-3} \\
					 &=\alpha ^{k-3}(\alpha +1) \\
						&=\alpha ^{k-3}(\alpha ^{2}) \tag{\(\alpha ^{2}=\alpha +1\)} \\
						&=\alpha ^{k-1} \\
						&=\alpha ^{(k+1)-2} 
	.\end{align*}
	as desired. Therefore, \( f_n \ge \alpha ^{n-2}  \) for all \( n \in \mathbb{N} \).
\end{proof}

\begin{eg}
	Consider the following matrix \( M = \begin{pmatrix} 1 & 1 \\ 1 & 0 \end{pmatrix}  \). Note that \( M\begin{pmatrix} x \\ y \end{pmatrix} =\begin{pmatrix} x + y \\ x \end{pmatrix}  \).

	Show that for \( f_{0}=0\), \(f_{1}=1 \), and \( f_n=f_{n-1}+f_{n-2} \): \[
		M^n = \begin{pmatrix} f_{n+1} & f_n \\ f_n & f_{n-1} \end{pmatrix} 
	.\] 
\end{eg}

\begin{proof}
	We have for \( n=1 \), \( M^1 = \begin{pmatrix} 1& 1 \\ 1& 0 \end{pmatrix}  \), and for \( n=2 \), \( M^2=\begin{pmatrix} 2 & 1 \\ 1 & 1 \end{pmatrix}  \). Assume that for some \( k \), \[
		M^k=\begin{pmatrix} f_{k+1} & f_k \\ f_k & f_{k-1} \end{pmatrix} 
	.\] Then, we have 
	\begin{align*}
		M^{k+1} = M^k \cdot M &= \begin{pmatrix}f_{k+1} & f_k \\ f_k & f_{k-1} \end{pmatrix} \begin{pmatrix} 1 & 1 \\ 1 & 0 \end{pmatrix} \\
													&= \begin{pmatrix} f_{k+1}+f_k & f_{k+1} \\ f_k+f_{k-1} & f_k \end{pmatrix} \\
													&= \begin{pmatrix} f_{k+2} & f_{k+1} \\ f_{k+1} & f_k \end{pmatrix} 
	.\end{align*}
	as desired.
\end{proof}

\begin{remark}
	Check \( f_{n-1}\cdot f_{n+1}=f_n^2+(-1)^n \)
\end{remark}

This is because
\begin{align*}
	f_{n-1}\cdot f_{n+1} - f_n^2 &= \det \begin{pmatrix} f_{n+1} & f_n \\ f_n & f_{n-1} \end{pmatrix} \\
															 &= \det \left( \begin{pmatrix} 1 & 1 \\ 1 & 0 \end{pmatrix}^n  \right) \\
																&= \det \begin{pmatrix} 1 & 1 \\ 1 & 0 \end{pmatrix} ^n \\
																&= (-1)^n
.\end{align*}

\begin{eg}
	Count the number of bit-sequences of length \( n \) such that there are no two consecutive 1s.
\end{eg}

Let \( a_n \) be the number of bit-sequences. Note that if we place a 0, we can fill the rest with \( a_{n-1} \) sequences. Note that if we place a 1, then the next bit must be a 0, such that we can place the rest with \( a_{n-2} \). In other words, the number of bit-sequences is given by \[
	a_n = a_{n-1} + a_{n-2}
.\] Note that we have 2 bit-sequences of length 1 and 3 bit-sequences of length 2. We quickly realize that \( a_n \) is the \( n+2 \)-nd Fibonacci number.

\begin{eg}
	Let \( s_n \) be the number of bit-sequences such that there are no two 1s at distance 2.
\end{eg}

We have \( s_3=6 \). Note that if we place a 0, we can fill the rest with \( s_{n-1} \) sequences. Note that if we place a 1, then we can place two 0s such that we can fill the rest with \( s_{n-3} \) sequences. Or, we can place two 1s and then two 0s such that we can fill the rest with \( s_{n-4} \) sequences. In other words, our recurrence relation is \[
	s_n = s_{n-1} + s_{n-3} + s_{n-4}
.\] 

\begin{note}
	If we define the empty string to be a bit-sequence, then \( s_0=1 \). This is ok as long as we don't use the empty string in our operation.
\end{note}

\lecture{18}{Tue 31 Oct 2023 14:00}{Modular Arithmetic}

\section{Modular Arithmetic}

Consider the relation \( \equiv_m \) defined by the following: \[
	a \equiv_m b \iff (a - b) \text{ is a multiple of } m
.\] 

\begin{note}
	This is an equivalence relation because it is reflexive, symmetric, and transitive.
\end{note}

We can establish this relation because we know the following theorem (that defines a remainder):

\begin{theorem}
	Let \( b,m \in \mathbb{Z},m>0 \). There exists a unique pair of integers \( q,r \) such that \( b = qm + r \) and \( 0 \le r < m \).
\end{theorem}

\begin{eg}
	13 = 2(5) + 3, such that \( q=2,r=3 \), so \( 13 \equiv_5 3 \). 
\end{eg}

\begin{definition}
	\( \mathbb{Z}_m = \{\overline{0}   ,\overline{1} ,\overline{2} ,\ldots ,\overline{m-1} \}  \) is the set of remainders when \( m \) is divided by a number.
\end{definition}

\begin{eg}
	Let \( m=4 \). \( \mathbb{Z}_4 = \begin{Bmatrix} \overline{0}&=&\{0,\pm 4,\pm 8,\pm 12,\ldots \} \\ \overline{1}&=&\{\ldots ,-7,-3, 1,5,9,\ldots \} \\ \overline{2}&=&\{\pm 2, \pm 6, \pm 10, \ldots \} \\ \overline{3}&=&\{\ldots ,-5,-1,3,7,11,\ldots \}           \end{Bmatrix}   \) 
\end{eg}

\begin{notation}
	We write \( b \pmod m \) to denote the remainder of \( b \) when divided by \( m \).
\end{notation}

\begin{notation}
	We write \( a \equiv b \pmod m \) to denote that \( a \equiv_m b \).
\end{notation}

\subsection{Operations in \( \mathbb{Z}_m \)}

How do we perform operations?

\begin{definition}
	We define \textbf{addition} to be \( \overline{a}+\overline{b} \coloneq \overline{a+b}     \).
\end{definition}

\begin{eg}
	\( \overline{10}+\overline{8}=\overline{10+8}=\overline{18}=\overline{2}    \) when \( m=4 \).
\end{eg}

\begin{definition}
	We define \textbf{multiplication} to be \( \overline{a}\cdot \overline{b}=\overline{ab}    \).
\end{definition}

\begin{eg}
	\( \overline{2}+\overline{13}=\overline{5} \pmod {10}   \) when \( m=10 \).
\end{eg}

\begin{note}
	Integers are not closed under division, so we must be careful. 
\end{note}

As we can see, the zero property and identity property in modular multiplication does not hold. Just like in linear algebra, we must define the inverse of a number in order to perform division.

\begin{definition}
	Let \( m>1 \). We say \( a\in \mathbb{Z} \) is a \textbf{divisor of of zero} if \( \exists b \in \mathbb{Z} \), \( b \not\equiv 0 \pmod m \) such that \( a \cdot b \equiv 0 \pmod m \). In other words, \( \exists \overline{b}\in \mathbb{Z}_m,\overline{b}\neq \overline{0}\land \overline{a}\neq \overline{0}  \land \overline{a}\overline{b}=\overline{0}       \)
\end{definition}

\begin{definition}
	Let \( a \in \mathbb{Z} \). We say \( a \) is \textbf{invertible} if \( \exists b \in \mathbb{Z} \) such that \( a \cdot b \equiv 1 \pmod m \).
\end{definition}

\begin{note}
	These two properties are mutually exclusive.
\end{note}

\begin{definition}
	Let \( a,b \in \mathbb{Z} \). We define their greatest common divisor (\( \gcd \)) of \( a \) and \( b \) as \[
		\gcd(a,b) = \max\{d \in \mathbb{Z} : d|a \land d|b\}  
	.\] 
\end{definition}

\begin{prop}
	Let \( a>b>0 \) be integers. Then, it holds that \[
		\gcd (a,b) = \gcd (a-b,b)
	.\]
\end{prop}

\begin{proof}
	Let \( d=\gcd(a,b) \), \( \hat{d}=\gcd(a,a-b) \). Note that if \( x|y \) and \( x|z \), then \( x|\alpha y+\beta z \) for \( x,y,z,\alpha ,\beta \in \mathbb{Z} \). Then, we know that \[
		d|a \land d|b \text{ by definition of } \gcd(a,b)
	.\] Set \( \alpha =1,\beta =-1 \) to get \( d|a-b \) fron our note. Then, \( d \) is a common divisor of both \( a \) and \( a-b \). However, \( \hat{d}\ge d \) because \( \hat{d} = \gcd(a,a-b) \) (\( \hat{d} \) is the \emph{greatest} common divisor). We also know that \[
		\hat{d}|a \land \hat{d}|a-b \text{ by definition of } \gcd(a,a-b)
	.\] Set \( \alpha =1,\beta =-1 \). then, we have \[
		\hat{d}|\alpha a+\beta (a-b) = a + (b-a) = b
	.\] Now, we know that \( \hat{d} \) is a common divisor of both \( a \) and \( b \). Then, \( d \ge \hat{d} \) because \( d=\gcd(a,b) \). Because \( \ge   \) is antisymmetric, we conclude that \( d=\hat{d} \).
\end{proof}

\begin{algorithm}[H]
	\caption{Euclidean Algorithm}
	\KwIn{ $a \ge b\ge 0$ }
	\KwOut{ $\gcd(a,b)$ }
	\If{b = 0}{
		return \( a \)\;
	}
	return $\gcd(b,a \mod b)$\;
\end{algorithm}

\begin{lemma}
	Let \( d=\gcd(a,m) > 1 \). Then \( a \) is a divisor of zero \( \pmod m \).
\end{lemma}

\begin{proof}
	Let \( b=\frac{m}{d} \). Then, \( a\cdot b=a\cdot \frac{m}{d}=\frac{a}{d}\cdot m \equiv 0 \pmod m \). The last equality follows from the fact that \( d|a \).
\end{proof}

\begin{eg}
	Let \( a=2 \), \( m=10 \). Then, \( b=\frac{10}{2}=5 \), and \( 2 \cdot 5 \equiv 0 \pmod {10} \).
\end{eg}

\begin{theorem}
	Let \( a\ge b \), \( a,b \in \mathbb{N} \). Then, \[
		\gcd(a,b) = \min \{d > 0 : \exists \alpha ,\beta ~ d = \alpha a + \beta b\}  
	.\] 
\end{theorem}

We will prove this next lecture!

\begin{lemma}
	Let \( 1=\gcd(a,m) \). Then, \( \exists ! b \in \mathbb{Z}_m \) such that \( \overline{a}  \cdot \overline{b} = \overline{1} \). In other words, \( a \) is invertible \( \pmod m \).
\end{lemma}

\begin{proof}
	From the theorem, we have \( \exists \alpha ,\beta : 1 = \alpha a + \beta m \). If we take this expression \( \pmod m \), we have \( \overline{1}=\overline{\alpha }\cdot \overline{a}+\cancelto{0}{\overline{\beta }\cdot \overline{m}}      \). So, \( \alpha  \) is the inverse of \( a \) \( \pmod m \)
\end{proof}

\begin{note}
	From these two lemmas, we can classify every integer \( a \) as invertible, or a divisor of 0.
\end{note}


\lecture{19}{Thu 02 Nov 2023 14:10}{Modular Arithmetic Continued}

How do we solve \( \overline{a}\overline{x}=\overline{b}    \) in \( \mathbb{Z}_m \)?

It suffices to solve the cases where \( \overline{b}=\overline{0}   \) and \( \overline{b}=\overline{1}   \). We first check \( \gcd(a,m)=d \).

\begin{eg}
	Find \( \gcd(30,12) \).
\end{eg}

By the Euclidean algorithm, we have
\begin{align*}
	& \gcd(30, 18) \tag{\( 30=18\cdot 1 + 12 \)} \\
	&= \gcd(18, 12) \tag{\( 18=12\cdot 1 + 6 \)} \\
	&= \gcd(12, 6) \tag{\( 12=6\cdot 2 + 0 \)} \\
	&= \gcd(6,0) = 6
.\end{align*}

Then, let us go back to the spot where we have a remainder of 6. Then, we get a linear combination \[
	6 = 18 \cdot 1 + 12 \cdot (-1)
.\] We also get from one step above that: \[
	6 = 18 \cdot 1 + (30 \cdot 1 + 18 \cdot -2) \cdot (-1) = 30 \cdot (-1) + 18 \cdot 3
.\] In other words, we have now found a method for finding the coefficients of the linear combination that represents the gcd in terms of the two numbers we started with. 

\begin{definition}	
	Let \( d=\gcd(a,b) \). Then, there exists numbers \( s,t \in \mathbb{Z} \) such that \( d=as+bt \). These are known as \textbf{Bezout coefficients}.
\end{definition}

We can use the Extended Euclidean Algorithm to find one such coefficient.

\begin{algorithm}
	\caption{Extended Euclidean Algorithm}
	\KwIn{\( (a,b), a \ge b \)}
	\KwOut{\( (d,s,t) \) such that \( d=\gcd(a,b) \), \( d=as+bt \)}
	\( r_0=a \)\;
	\( r_1=b \)\;
	\( s_0=1 \)\;
	\( s_1=0 \)\;
	\( t_0=0 \)\;
	\( t_1=1 \)\;
	\While{\( r_k\neq 0 \)}{
		\( r_{k+1}=r_{k-1} \pmod {r_k} \)\;
		\( s_{k+1}=s_{k-1} - (r_{k-1} \text{ div } r_k)\cdot s_k \)\;
		\( t_{k+1}=t_{k-1} - (r_{k-1} \text{ div } r_k) \cdot t_k \)\;
	}
	return \( (r_{k-1}, s_{k-1}, t_{k-1}) \)\;
\end{algorithm}

\begin{remark}
	If \( d=1 \), then \( s \) is the inverse of \( a \pmod b \).
\end{remark}

We can also solve systems of linear equations in \( \mathbb{Z}_m \). Let's say we wish to find a number such that we can satisfy \[
	\begin{pmatrix}
		x &\equiv a_1 \pmod {m_1} \\
		x &\equiv a_2 \pmod {m_2} \\
			& \hdots \\
		x &\equiv a_k \pmod {m_k} \\
	\end{pmatrix}
.\] 

\begin{theorem}
	(Chinese Remainder Theorem) Let \( m_{1},m_{2},\ldots m_k \in \mathbb{N}\) be numbers such that \( \gcd(m_i,m_j)=1 \) for all \( 1 \le  i < j \le  k \). Then, the system has a unique solution in \( \mathbb{Z}_M \) where \( M=m_{1}m_{2}\ldots m_k \).
\end{theorem}

\begin{proof}
	First, we will prove the solution's existence. Set \( M_i =\frac{M}{m_i}\). Because all \( m \)'s don't share a factor, \( \gcd(m_i, M_i) = 1\). Let \( y_i \) be the inverse of \( M_i \pmod {m_i}\). Consider: \[
		x = M_1 \cdot y_{1} \cdot  a_{1} + M_{2} \cdot y_{2}\cdot a_{2}+ \ldots + M_{k} \cdot y_{k} \cdot a_{k}
	.\] Then, we check: 
	\begin{align*}
		x &\equiv M_{1}\cdot y_{1}\cdot a_{1}+ \cancelto{0}{M_{2}\cdot y_{2}\cdot a_{2}} + \ldots  + \cancelto{0}{M_k \cdot  y_k \cdot a_k} \pmod {m_1} \\
			& \equiv \cancelto{1}{M_{1}\cdot y_{1}}\cdot a_{1} \equiv a_{1} \pmod {m_1}
.\end{align*}
  We can reprove this for each \( 2\le i\le k \).
\end{proof}

\begin{eg}
	Solve \[
		\begin{matrix}
			x & \equiv 3 \pmod 5 \\
			x & \equiv 2 \pmod 7
		\end{matrix}
	.\] 
\end{eg}

We first check that \( \gcd(5,7)=1 \), which it is.
Then, we continue as follows:
\begin{align*}
	M&=5\cdot 7=35 \\
	M_1 &= \frac{35}{5} = 7 \\
	M_2 &= \frac{35}{7} = 5 \\
	y_{1} &= 3 \tag{Inverse of 7 mod 5} \\
	y_{2} &= 3 \tag{Inverse of 5 mod 7} \\
.\end{align*}
Therefore, the solution is as follows: 
\begin{align*} 
	x &= M_{1} \cdot  y_{1} \cdot  a_{1} + M_{2} \cdot  y_{2} \cdot  a_{2} \\
		&= 7 \cdot  3 \cdot 3 + 5 \cdot 3 \cdot  2 = 93\\
		& \equiv 23 \pmod {35}
.\end{align*}

\begin{note}
	This solution (\( x \equiv 93 \pmod {35} \)) is the only solution, which we will prove in our homework.
\end{note}

\lecture{20}{Tue 07 Nov 2023 14:12}{What the Modular Arithmetic}

Fix \( m \in \mathbb{N} \), \( m>1 \).

Let \( a \in \mathbb{N} \). Then, \[
	\{a^n\}_{n\ge 0} \text{ in } \mathbb{Z}_m=\{\overline{0},\overline{1},\overline{2},\ldots ,\overline{m-1}    \}  
.\] 

Note that we can easily calculate \( a^k \) with \[ a^k \pmod m= \left( a^{k-1} \pmod m \right) \left( a \pmod m \right) \pmod m .\] as well as finding repetition. What do we mean by this?

\begin{eg}
	Let \( m=6 \), and \( a=11 \). Then, \( a^n \) has periodicity \( 2 \) when taken mod m.
\end{eg}

\begin{eg}
	What is \( 4^{2023} \pmod 7 \)? Note that \( 2023 \equiv 1 \pmod 3 \) (from the periodicity), so \( 4^{2023}\equiv 4^1 \equiv 4 \pmod 7  \).
\end{eg}

\begin{theorem}
	Let \( a \in \mathbb{N} \). \( a \) is divisible by 3 if and only if the sum of its digits is divisible by 3.
\end{theorem}

\begin{proof}
	All of \( \{10^n\}_{n\ge 0}  \) in \( \mathbb{Z}_3 \) are in the class of \( \overline{1}  \)! Therefore, \( 10^k \equiv 1 \pmod 3 \). Then, \( a \) can be written as \[
		a = (a_k a_{k-1} a_{k-2}\ldots a_{1}a_{0})_{10} \qquad a_i \in \{0,1,2,\ldots 9\}  
	.\] Then, we know that
	\begin{align*}
		a &= 10^k a_k + \ldots  + 10^2 a_2 + 10 a_1 + a_0 \\
			&\equiv \cancelto{1}{\overline{10^k}}\cdot \overline{a_k} + \ldots  + \cancelto{1}{\overline{10^2}}\cdot \overline{a_2}  + \cancelto{1}{\overline{10} }\cdot \overline{a_1} + a_{0} \pmod 3 \\
			&= a_k + \ldots  + a_{1} + a_{0} \pmod 3
	.\end{align*}
\end{proof}

\begin{theorem}
	Let \( a \in \mathbb{N} \). \( a \) is divisible by 4 if and only if the last two digits of \( a \) are divisible by 4.
\end{theorem}

\begin{proof}
	We know that \( \{10^n\} _{n\ge 0}  \) in \( \mathbb{Z}_{4} \) are \( \overline{1}  \) for \( n=0 \), \( \overline{2}  \) for \( n=1 \), and \( \overline{0}  \) for \( n\ge 2   \). Therefore, we only look at \( n=0 \) and \( n=1 \) (tens and ones place).
\end{proof} 

\begin{lemma}
	If \( 1 \le i < j \le p-1 \). Then, \( ai \not\equiv aj \pmod p \).
\end{lemma}

\begin{proof}
	\( ai-aj = a(i-j) \not\equiv 0 \pmod p \). 
\end{proof}

\begin{theorem}
	(Fermat) Let \( p \) be prime. Let \( a \in \mathbb{N} \) such that \( \gcd(a,p)=1\). Then, \( a^{p-1}\equiv 1 \pmod p \).
\end{theorem}

\begin{proof}
	Fix \( p \), prime number. Then we know for \( a^n \) in \( \mathbb{Z}_p \), \( a\in \mathbb{N} \),  \( \gcd(a,p)=1 \). Therefore, all of \[
		1, a, a^2, a^3, \ldots , a^{n}
	.\] are invertible. Then all of \[
		\overline{1}, \overline{2}, \overline{3}, \overline{p-1} 
	.\] and \[
		\overline{1a}, \overline{2a}, \overline{3a}, \overline{(p-1)a} 
	.\] are invertible. Note that the previous two sets are the same sets. Then, we have
	\begin{align*}
		\overline{1}\cdot \overline{2}\cdot \overline{3}\cdot \ldots \cdot \overline{(p-1)} &\equiv \overline{1a}\cdot \overline{2a}\cdot \overline{3a}\cdot \overline{(p-1)a}     \pmod p \\
		(p-1)! &\equiv (p-1)! \cdot a^{p-1}  \pmod p \\
						1 &\equiv a^{p-1} \pmod p
	.\end{align*}
\end{proof}

\begin{eg}
	What is \( 4^{2023} \pmod 7  \)?
\end{eg}

Note that 2023 is prime, and therefore \( 4^{2022}\equiv 1 \pmod 7   \). Then, \( 4\cdot 1 \equiv 4 \pmod 7 \).

\begin{eg}
	What is \( 4^{2023} \pmod {131} \)?
\end{eg}

Note that \( 2023 \pmod {130} = 73 \). Then, by Fermat's theorem, \( 4^{2023} \equiv 4^{73} \pmod {131}   \).

Going even deeper, What about counting the number of invertible elements?
\[
	\left| \{1 \le  x \le m \colon \gcd(x,m)=1\}   \right| = \varphi(m)
.\] is the number of invertible elements in \( \mathbb{Z}_m \). Then, \( \varphi(p)=p-1 \). Note that \( \varphi(p^2)=p(p-1) \).

\begin{theorem}
	\( \varphi(ab)=\varphi(a) \cdot  \varphi(b) \) as long as \( \gcd(a,b)=1 \).
\end{theorem}

\begin{theorem}
	(Euler) Let \( m \in \mathbb{N} \). Let \( p_{1},p_{2},\ldots ,p_k \) be the distinct prime factors of \( m \). Then, \[
		\varphi(m) = m \cdot \left( 1-\frac{1}{p_{1}} \right) \left( 1- \frac{1}{p_{2}} \right)\ldots  \left( 1-\frac{1}{p_k} \right) 
	.\] 
\end{theorem}

\begin{theorem}
	(Euler) Let \( m \in \mathbb{N} \), \( m > 1 \). Let \( a \in \mathbb{N} \) such that \( \gcd(a,m)=1 \). Then \[
		a^{\varphi(m)} \equiv 1 \pmod m
	.\] 
\end{theorem}

\subsection{Cryptography}

From Euler's theorem, we have \( a^{\varphi(m)+1}\equiv a \pmod m  \).

We label our alphabet such that \( A \to 00 \), \( B \to 01 \), \( Z \to  25 \), etc. Then, you can write every message as a sequence of digits. Let this message be \( m \). We release to the world public key \( (N,e) \).

The encryption key is \( m^e \pmod N \). Then, the decryption key is then \( m^{t\cdot \varphi(N)+1}\equiv m \pmod N  \). the decryption strategy is that if we find \( d \in \mathbb{N} \) such that \( ed \equiv 1 \pmod {\varphi(n)} \), then we have \[
	\left( m^e \right) ^d = m^{e\cdot d} = m^{t\cdot \varphi(n)+1} \equiv m \pmod n
.\] such that we can then decrypt the message.

\begin{note}
	For the public key, we need \( \gcd(\varphi(N), e)=1 \).
\end{note}

\begin{remark}
	We usually set \( N=p\cdot q \), both prime. Then, \( \varphi(N) = (p-1)(q-1) \). This is better than \( N \) prime because our enemies need to factorize \( N \) to find our \( p \) and \( q \). By the time they read our message, it is already too late.
\end{remark}

\begin{note}
	If a message is longer than \( N \), we need to break it into pieces because when don't want some number congruent to the message, but the message itself.
\end{note}

\lecture{21}{Thu 09 Nov 2023 14:11}{Counting}

\section{Counting}

We can use the product rule for counting (when counting happens sequentially).

\begin{eg}
	Let a license plate be defined by 3 letters followed by 3 digits. How many possible license plates are there?
\end{eg}

We can choose 26 for the first letter, 26 for the second, and 26 for the third. Then, we can choose 10 for the first digit, 10 for the next, and 10 for the last. Therefore, our answer is \[
	26\cdot 26\cdot 26\cdot 10\cdot 10\cdot 10=(260)^3
.\] 

\begin{eg}
	Let \( |A|=n \), \( |B|=m \). How many functions are there such that \( f:A\to B \)?
\end{eg}

Every element in \( A \) can be mapped to an element in \( B \), such that there are \( m \) choices for all \( n \) elements. Then, the total number of functions is \( m^n\).

\begin{eg}
	Let \( |A|=n \), \( |B|=m \). How many functions \( f:A\to B \) are there such that \( f \) is onto?
\end{eg}

Note that \( a_{1} \) must be mapped to an element in \( B \). There are \( m \) choices to do so. Then, \( a_{2} \) must be mapped to an element in \( B \), not equal to \( f(a_{1}) \). There are \( m-1 \) choices to do so. Then, \( a_{3} \) must be mapped to an element in \( B \), not equal to \( f(a_{1}) \) and \( f(a_{2}) \). There are \( m-2 \) choices to do so. It follows that the total number of functions is \[
	\begin{cases}
		m(m-1)(m-2)\cdots(m-n+1)=(m)_n & m \ge n\\
		0 & m < n
	\end{cases}
.\] 

\begin{notation}
	\( (m)_n = \frac{m!}{(m-n)!} = m(m-1)(m-2)\ldots (m-n+1)\) 
\end{notation}

We can use the sum rule to add together cases in counting:

\begin{eg}
	Count the number of passwords consisting of letters and digits, with length 8-10.
\end{eg}

We can count passwords of length 8, 9, and 10 separately. \[
	36^8+36^9+36^{10}
.\] 

We can also use the subtraction rule! We can count the number of things that do not satisfy our condition, then subtract that number from the total number of things.

\begin{eg}
	Count the number of passwords with at least one digit.
\end{eg}
Note that if we naively fix the position of the digit, make it a digit, and fill the rest of the letters (with 36 options), then we will be over-counting passwords!

Instead, we count the total number of passwords with no constraints, and subtract the number of passwords with no digits, yielding solution \[
	36^8+36^9+36^{10} - (26^8+26^9+26^{10})
.\] 

\begin{notation}
	We will define factorial (!) as \[
		n! \coloneq \begin{cases}
			1 & n=0\\
			n(n-1)! & n>0
		\end{cases}
	.\]
\end{notation}

There is also a division rule, where we count every element exactly \( N \) times.

\begin{definition}
	\textbf{Permutations} count the number of something where order matters! It is written as \[
		P(n,k)=\frac{n!}{(n-k)!}= \text{\# of ordered }k\text{-tuples from } n \text{ elements}
	.\] 
\end{definition}

We derive this formula from the division rule: if we have a permutation of length \( n \) of \( n \) elements, we are counting the number of permutations of length \( k \) of \( n \) elements \( n-k \) times!

\begin{definition}
	\textbf{Combinations} count the number of something where order \textit{order doesn't matter}. It is written as \[
		\binom{n}{k} = \frac{n!}{k!(n-k)!}
	.\] 
\end{definition}

Note that permutations also counts this, but there are \( k! \) ways to order a combination. Therefore, we divide the number of permutations by \( k! \) to yield this amount.

We can also use recursion for counting sets. Let \( 1,2,3,\ldots ,n \) be the elements we can choose from. We can add \( k \) to the number of sets of size \( n-1 \) without \( k \), of which there are \( \binom{n-1}{k-1} \) of them. We can add \( k \) to the number of sets of size \( n-1  \) with \( k \), of which there are \( \binom{n-1}{k} \) of them. Therefore, we have Pascal's identity: \[
	\binom{n}{k}=\binom{n-1}{k-1}+\binom{n-1}{k}
.\]

\lecture{22}{Tue 14 Nov 2023 14:05}{Counting Continued}

\begin{eg}
	How many strings of 10 bits are there such that there are exactly 4 1's?
\end{eg}

We can choose 4 places of 10 to place 1's, and the rest are 0's. So the answer is $\binom{10}{4}$. 

\begin{eg}
	How many strings of 10 bits are there such that there are at most 4 1's?
\end{eg}

We can use the sum rule! \[
	\binom{10}{0} + \binom{10}{1} + \binom{10}{2} + \binom{10}{3} + \binom{10}{4} = 386
.\] 

\begin{eg}
	Show that if \( |S| = n \), then \( |\mathcal{P}(S)| = 2^n \).
\end{eg}

A subset can be formed by choosing whether or not to include every element. Therefore, the number of subsets, the cardinality of the powerset, is \( 2^{|S|} = 2^n \).

\begin{theorem}
	\[
		(a+b)^n = \sum_{k=0}^n \binom{n}{k} a^k b^{n-k}
	.\] 
\end{theorem}

\begin{proof}
	For each term \( (a+b)(a+b)\ldots (a+b) \), we can choose to either multiply the \( a \) or the \( b \). If there are \( k \) number of \( a \)'s, then there are \( n-k \) number of \( b \)'s. Therefore, the number of terms with \( k \) number of \( a \)'s is \( \binom{n}{k} \). Therefore, the coefficient of \( a^k b^{n-k} \) is \( \binom{n}{k} \). Summing over all \( k \) gives the result.
\end{proof}

\exercise{1}
Prove the binomial theorem with induction!

\end{document}